\section{Kiste auf schiefer Ebene}
\subsection{Lösung i}
  Bei dieser Aufgabe muss man sich klar machen, dass die Kisten in zwei Richtungen rutschen können. Als Lösung
  für die Masse $m_2$, für die das System in Ruhe ist, wird sich also ein Intervall ergeben.
  Zunächst soll der Fall des Rutschens nach rechts betrachtet werden. \\
  In Abbildung x sind die Kräfte eingezeichnet, die auf Kiste 1 wirken. Die Haftreibung ist immer so gerichtet, dass
  sie der Bewegungsrichtung entgegen wirkt, hier also in die selbe Richtung wie die Gewichtskraft der Kiste 2. Sie setzt
  sich aus der Normalkraft und dem Haftreibungskeffizienten $\mu_H$ zusammen:
  \begin{equation}
    F_{R,H} = F_N \cdot \mu_H
  \end{equation}
  Die Normalkraft ergibt sich aus der Winkelbeziehung:
  \begin{equation}
    \cos\left(\Theta\right) = \frac{F_N}{F_{G,1}}\quad \Leftrightarrow\quad F_N = \cos\left(\Theta\right) \cdot F_{G,1}
  \end{equation}
  Auf analoge Weise erhält man die Hangabtriebskraft:
  \begin{equation}
     F_H = \sin\left(\Theta\right) \cdot F_{G,1}
  \end{equation}
  Nun betrachtet man den Fall des Kräftgleichgewichts und stellt nach $m_2$ um:
  \begin{align}
    \begin{aligned}
      F_{G,2} + F_{R,H} &= F_H\\
    \Leftrightarrow \quad  F_{G,2} + \cos\left(\Theta\right) \cdot F_{G,1} \cdot \mu_H &= \sin\left(\Theta\right) \cdot F_{G,1} \\
    \Leftrightarrow \quad m_2 g + \cos\left(\Theta\right) \cdot m_1 g \cdot \mu_H &= \sin\left(\Theta\right) \cdot m_1 g \\
    \Leftrightarrow \quad m_2  + \cos\left(\Theta\right) \cdot m_1  \cdot \mu_H &= \sin\left(\Theta\right) \cdot m_1  \\
    \Leftrightarrow \quad m_2 &= m_1 \cdot (\sin\left(\Theta\right) - \cos\left(\Theta\right)  \cdot \mu_H) = \SI{2.8}{\kilo\gram}
  \end{aligned}
  \label{eq: m_2}
  \end{align}
Das heßt für alle Werte $m_2 > \SI{2.8}{\kilo\gram}$ rutscht das System nicht nach rechts. Nun muss die selbe Rechnung noch für den Fall des Rutschens nach
links durchgeführt werden. Hier wirkt die Reibung nun nach rechts, das heißt im Kräftegleichgewicht gilt:
\begin{equation}
  F_{G,2}  = F_H + F_{R,H}
\end{equation}
Die Lösung unterscheidet sich also lediglich um ein Vorzeichen im Vergleich zu \eqref{eq: m_2}.
\begin{equation}
  m_2 = m_1 \cdot (\sin\left(\Theta\right) + \cos\left(\Theta\right)  \cdot \mu_H) = \SI{9.2}{\kilo\gram}
\end{equation}
Das heißt für alle Werte $m < \SI{9.2}{\kilo\gram}$ rutscht das System nicht nach links. Insgesamt ergibt sich:
\begin{equation}
  \text{System in Ruhe} \quad \Leftrightarrow \quad m_2 \in [2.8,9.2]\si{\kilo\gram}
\end{equation}
\subsection{Lösung ii}
Die Masse $m_2 = \SI{10}{\kilo\gram}$ liegt außerhalb des Ruheintervalls. Das System rutscht in diesem Fall nach links. Ansatz:
\begin{equation}
  (m_1 + m_2) \cdot a_{ges} = F_{G,2} - F_{R, G} - F_H
\end{equation}
Hierbei entspricht $F_{R, G}$ nun der Gleitreibungskraft, die der Bewegung entgegen wirkt.
\begin{equation}
  F_{R, G} = \cos\left(\Theta\right) \cdot F_{G,1} \cdot \mu_G
\end{equation}
Damit lässt sich die Beschleunigng berechnen:
\begin{align}
  \begin{aligned}
    a_{ges} &= \frac{m_2 \cdot g - m_1 \cdot g\cdot(\sin\left(\Theta\right) + \cos\left(\Theta\right)\cdot \mu_G)}{(m_1 + m_2)} \\
            &= \SI{0.784}{\meter\per\second^2}
\end{aligned}
\end{align}


\newpage
