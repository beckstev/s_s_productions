\section{Musterlösung Aufgabe 2}
\begin{equation*}
\vec{F}=\begin{pmatrix} -3x^2y^2z+y^3 \\ -2x^3yz+3xy^2-z \\ -x^3y^2-y\end{pmatrix}
\end{equation*}

\subsection{Aufgabenteil 1}

\begin{equation*}
\mathrm{rot}\vec{F}=\vec{0}
\end{equation*}

\subsection{Aufgabenteil 2}

Nutze den Ansatz:

\begin{equation*}
\mathrm{grad}\phi=\begin{pmatrix} \partial_x \phi \\ \partial_y \phi \\ \partial_z \phi \end{pmatrix} =\begin{pmatrix} -F_x \\ -F_y \\ -F_z \end{pmatrix}=-\vec{F}
\end{equation*}

Beginne mit der ersten Komponente:

\begin{align*}
\partial_x \phi&=3x^2y^2z-y^3 \\
\Rightarrow \quad  \phi&=\int 3x^2y^2z-y^3 \mathrm{d}x = x^3y^2z-y^3x+c(y,z)
\end{align*}

Damit haben wir schonmal ein Grundgerüst des Potential gefunden.
Dieses versuchen wir jetzt weiter zu verfeinern.
Dazu machen wir mit der zweiten Komponente weiter:

\begin{align*}
\partial_y \phi= \partial_y\left( x^3y^2z-y^3x+c(y,z)\right)&=2x^3y-3xy^2+\partial_y c(y,z)= 2x^3yz-3xy^2+z \\
\Leftrightarrow \quad \partial_y c(y,z)&=z\\
\Rightarrow \quad c(y,z)=yz+c(z)
\end{align*}

Noch die letzte Zeile

\begin{align*}
\partial_z\phi= \partial_y\left( x^3y^2z-y^3x+yz+c(z)\right)&=y^3y^2+y+\partial_z c(z)=x^3y^2+y\\
\Leftrightarrow \quad \partial_y c(z)&=0\\
\Rightarrow \quad c(z)=d \quad d\in\mathbb{R}
\end{align*}

Damit haben wir das Potential $\phi=x^3y^2z-y^3x+yz+d \quad d\in\mathbb{R}$.
