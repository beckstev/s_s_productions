\section{Musterlösung Aufgabe 3}
Masse der Kugel $M_1=\SI{100}{\gram}$,
Masse der Kugel $M_2=\SI{500}{\gram}$,
Federkonstante $k=\SI{0.2}{\kilo\gram\per\square\second}$ und Auslenkung $x_0=\SI{10}{\centi\meter}$

\subsection{Aufgabenteil 1}
Wir benötigen die folgenden Energien:

\begin{align*}
E_{sp}&=\frac{1}{2}kx^2\\
E_{kin}&=\frac{1}{2}mv^2
\end{align*}

Energien gleichsetzen:

\begin{align*}
\frac{1}{2}kx^2&=\frac{1}{2}mv^2\\
\Leftrightarrow \quad v&=\sqrt{\frac{k}{m}}x
\end{align*}

Für die Auslenkug $x_0$ ergibt sich die Geschwindigkeit

\begin{equation*}
v \sim \SI{0.14}{\meter\per\second}
\end{equation*}

\subsection{Aufgabenteil 2}

Es gilt die Impulserhaltung mit

\begin{equation*}
v_1M_1+v_2M_2=v'\left(M_1+M_2\right)
\end{equation*}

Umstellen liefert für $v'$:

\begin{equation*}
\Leftrightarrow \quad v'=\frac{v_1M_1+v_2M_2}{M_1+M_2} \sim \SI{0.02}{\meter\per\second}
\end{equation*}
