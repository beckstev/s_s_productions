\section{Kurzfragen}
\begin{enumerate}

  \item Nenne die drei Newtonschen Axiome.
  \item Ein Pfeil wird mit der Geschwindigkeit $v_x = 10 m/s$ aus $2\si{\meter}$ Höhe horizontal abgeworfen. Welche
        Entfernung legt er zurück, bevor er den Boden erreicht?
  \item Nenne zwei Kriterien für die Konservativität einer Kraft.
  \item Ein Zug fährt auf der Nordhalbkugel von Süden nach Osten. Welche Scheinkraft wirkt auf ihn und wie ist sie gerichtet?
  \item Was charakterisiert den \emph{unelastischen} Stoß? Nenne ein Beispiel für einen solchen Vorgang.
  \item Du bist Astronaut und möchtest dein Raumschiff fotografieren. Nun schwebst du aber mit der schweren Kamera in der Hand 50m
        neben dem Raumschiff und weißt nicht, wie du zurückkommen sollst. Oder doch?
  \item Du kaufst eine Wäscheschleuder. Bei gleichem Preis stehen zwei zur Auswahl.  Ist es besser eine zu kaufen mit $1.5$-fach
        größerem Radius oder mit $1.5$-fach höherer Umdrehungszahl? Warum?
  \item Ein Schlittschuläufer (Masse $75\si{\kilo\gram}$) wirft einer Schlittschuhläuferin (Masse $62.5\si{\kilo\gram}$) einen
        Medizinball (Masse $5\si{\kilo\gram}$) mit einer Horizontalgeschwindigkeit von $5\si{\meter\per\second}$ zu. Vor dem Wurf
        seien beide in Ruhe. Wie schnell und in welche Richtung bewegen sich beide nach dem Wurf? Reibung und Luftwiderstand sind
        zu vernachlässigen.
  \item Berechne mittels Integration den Flächeninhalt eines Kreises. Tipp: Denk an den guten Jacobi.
\end{enumerate}

\newpage
