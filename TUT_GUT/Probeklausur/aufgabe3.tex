\section{Energie und Impuls}
%Hier noch eine Skizze
Eine Kugel $M_1$ befindet sich vor einer Feder.
Die Feder wird nun um die Strecke $x_0$ eingedrückt.
\begin{enumerate}
\item{ Wie schnell ist die Kugel nach dem Loslassen der Feder. Masse der Kugel $M_1=\SI{100}{\gram}$,
Federkonstante $k=\SI{0.2}{\kilo\gram\per\square\second}$ und Auslenkung $x_0=\SI{10}{\centi\meter}$}
\end{enumerate}
Anschließend stößt die Kugel \emph{unelastisch} (Erinnerung: Der Hund un das Kind bzw. "UND") mit einer zweiten Kugel $M_1=\SI{500}{\gram}$. %ich würde hier noch nen Hinweis geben, falls jemand vergessen hat was unelastisch heißt. Zb: erinnere dich an die Übungsaufgabe mit dem Hund, der auf die schaukel springt oder so arrrsch
\begin{enumerate}
\item {Wie groß ist die Geschwindigkeit $v'$ beider Kugel nach dem Stoß.}
\end{enumerate}
