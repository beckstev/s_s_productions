\section{Das Trägheitsmoment eines Kegels}
Es soll das Trägheitsmoment eines Kegels (homogene Massenverteilung, Radius $R$ und Höhe $H$) bei einer Drehung um die z-Achse bestimmt werden.
Wir starten mit der allgemeinen Formel für das Trägheitsmoment:
\begin{equation}
\label{eq:trag}
I=\int_{V}\rho(\vec{r})r_{\perp}^2\dv{V}
\end{equation}
Es soll eine Drehung um die z-Achse betrachtet werden $\Rightarrow \quad r_{\perp}^2=x^2+y^2$.

Da ein Kegel eine Zylindersymmetrie aufweißt, nutzen wir Zylinderkoordinaten:
\begin{equation*}
\vec{r}=\begin{pmatrix} r\cos(\phi) \\ r\sin(\phi) \\ z \end{pmatrix}
\end{equation*}

Die Koordinaten werden in $r_{\perp}^2$ eingesetz:
\begin{equation*}
r_{\perp}^2=\left(r\cos(\phi)\right)^2+\left(r\sin(\phi)\right)^2=r^2
\end{equation*}
Nun muss noch eine Bedingung für $z(r)$ bzw. für $r(z)$ gefunden werden.

\subsection{Methode A}
Bei Methode A sei z abhängig vom Radius mit $z=\frac{rH}{R} \quad \Leftrightarrow \quad r=\frac{zR}{H}$ (bei $r=0$ ist $z=0$ und bei $r=R$ ist $z=H$). Wenn wir jetzt in \eqref{eq:trag} einsetzen erhalten wir:

\begin{equation*}
I=\rho\int_{V} r^2 \dv{V}=\rho \int_{0}^{2\pi}\int_{0}^{R}\int_{0}^{\frac{rH}{R}} \underset{\text{JD}}{r}r^2\dv{z}\dv{r}\dv{\phi}
\end{equation*}
JD ist die Jacobi Determinante.
Wir müssen jetzt noch $ r=\frac{hR}{H}$ einsetzen (damit der Satz von Fubini nicht verletzt wird):

\begin{equation*}
I=\rho \int_{0}^{2\pi}\int_{0}^{R}\int_{0}^{\frac{rH}{R}} \frac{z^3R^3}{H^3} \dv{z}\dv{r}\dv{\phi}
\end{equation*}
Das integrieren wir jetzt:

\begin{align*}
I=\rho 2\pi \int_{0}^{R}\left[\frac{1}{4}z^4\frac{R^3}{H^3}\right]_{z=0}^{z=\frac{rH}{R}}\dv{r}&=\rho \frac{1}{2}\pi \int_{0}^{R} r^4\frac{H}{R}\dv{r}\\
&=\rho \frac{1}{2}\pi \frac{1}{5}H R^4\\
\intertext{Mit $\rho=\frac{M}{V}=\frac{3M}{\pi R^2H}$ folgt}
&=\frac{3}{10}MR^2
\end{align*}

\subsection{Methode B}
Bei Methode B sei r abhängig von z mit $r=\frac{zR}{H}$. 
Es wird wieder in \eqref{eq:trag} eingesetzt:

\begin{equation*}
I=\rho\int_{V} r^2 \dv{V}=\rho \int_{0}^{2\pi}\int_{0}^{H}\int_{0}^{\frac{zH}{R}} \underset{\text{JD}}{r}r^2\dv{r}\dv{z}\dv{\phi}
\end{equation*}
Dieses Mal muss $r$ nicht ersetzen werden.
Wir lösen das Integral mit:

\begin{align*}
I=\rho 2\pi \int_{0}^{H}\left[\frac{1}{4}r^4\right]_{r=0}^{r=\frac{zR}{H}}\dv{z}&=\rho \frac{1}{2}\pi \int_{0}^{H} z^4\frac{R^4}{H^4}\dv{r}\\
&=\rho \frac{1}{2}\pi \frac{1}{5}H R^4s\\
\intertext{Mit $\rho=\frac{M}{V}=\frac{3M}{\pi R^2H}$ folgt}
&=\frac{3}{10}MR^2
\end{align*}

\begin{center}
Methode A ist, auf Grund der zusätzlichen substitution, nicht zu empfehlen.
Benutzt lieber Methode B.
\end{center}