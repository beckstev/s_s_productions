\section{Diskussion}
Abschließend soll die Aussagekraft der Ergebnisse diskutiert werden.

Die Messergebnisse des verwendeten Aufbaus hängen stark mit
unkontrollierbaren Umweltfaktoren wie z.\, B. Staub ab.
So wurden die Messungen, auf Grund von Bauarbeiten in einem Nachbarraum,
unter erhöhter Staubdichte durchgeführt. Die erhöhte Dichte führte wahrscheinlich zu einer Verringerung des
messbaren Photostroms, denn durch Ablagerung des Staubs an den Brewsterfenstern oder Spiegeln entsteht ein
Intensitätsverlust. Unter normalen Rahmenbedingungen wäre ein
maximaler Photostrom von $I_{p}^{max}\approx\SI{1.5}{\milli\ampere}$ zu erwarten,
bei den Messungen wurde jedoch lediglich ein Maximum von $I_{p}^{max}=\SI{0.76}{\milli\ampere}$
(vgl. Tab. \ref{tab: pola}) gemessen.

Durch den Staub wird die Signifikanz der als Letztes durchgeführten Stabilitätsmessungen besonders beeinträchtigt.
Bei Betrachtung des Plots \ref{fig: konkon} fallen starke Fluktuationen im Photostrom auf,
die vermutlich durch die hohe Staubdichte verursacht werden. Des Weiteren fällt auf, dass
in der Abbildung \ref{fig: konflach} der Photostrom abrupt abbricht, dies lässt sich
auch auf die erhöhte Staubdichte zurückführen. Somit sind die Ergebnisse der Stabilitätsbedingung nicht aussagekräftig und sollten
bei besseren Rahmenbedingungen wiederholt werden.

Jedoch beeinflusst der Staub die Signifikanz bei den anderen Messungen nur ein wenig.
Dies wird deutlich bei der Betrachtung der Abbildungen \ref{fig: T_00}, \ref{fig: T_10} und \ref{fig: pola},
hier fällt auf, dass die Messwerte eine hohe Übereinstimmung mit den Theoriekurven besitzen.
Die Übereinstimmung wird auch durch die kleinen Fehler der Fitparameter deutlich
(vgl. \eqref{eq: fit_t_00}, \eqref{eq: fit_t_10} und \eqref{eq: fit_pola}).
Somit spiegelt hier das Experiment die Theorie wieder.

Der Versuchsaufbau kann durch einen Laser umschließenden Kasten so modifiziert werden,
dass die Staubempfindlichkeit sinkt.

Eine weitere Fehlerquelle sind Ableseungenauigkeiten,
durch diese ist die Abweichung zu Theorie bei der Wellenlängenmessung zu erklären.
Aus der Messung resultiert für die Wellenlänge $\ov{\lambda}=\SI{636\pm9}{\nano\meter}$.
Die Abweichung zum Theoriewert $\lambda\ua{theo}=\SI{632.8}{\nano\meter}$ \cite{anleitung61}
liegt somit bei $\num{0.5\pm 1.5} \%$. Zwar liegt der Theoriewert im Fehlerbereich, dennoch
würde eine genaueres vermessen für eine Verkleinerung des Fehlers von \eqref{eq: mittelwert_wellenlaenge} sorgen.
