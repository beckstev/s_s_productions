\section{Diskussion}
Die gefundenen Ergebnisse zeigen, dass der Versuch durch systematische Fehler beeinflusst ist.
Die Gittervektoren der HOPG Struktur weichen sowohl in Betrag als auch Innenwinkel
deutlich von den Literaturwerten ab. Es ist jedoch anzumerken, dass durch die simple Korrketur der Vektoren
mit einer Diagonalmatrix \eqref{eq: korrekturmatrix} der erwartete Innenwinkel von $\SI{60}{\degree}$ mit dem so
gefunden Wert $\SI{66.4(6)}{\degree}$ noch relativ gut übereinstimmt. HOPG ist damit ein hervorragender Stoff
zur Kallibrierung des Rastertunnelmikroskops. Es fällt ebenfalls auf, dass die Korrektur in $y$-Richtung mit
$s\ua{y} = \num{1.670(8)}$ deutlich höher ist als in $x$-Richtung $s\ua{x} = \num{1.14(1)}$. Dies bestätigt
die Asymmetrie der Verzerrungen, die in Abbildung \ref{fig: up} zu erkennen sind. In $y$-Richtung
wird eine stärkere Streckung deutlich.

Die Verzerrungen der Bilder wie in Abbildung \ref{fig: up} sind auf thermische Drifts und andere Fehlerquellen
zurückzuführen. Da über die gesamte Messzeit keine merkliche Verbesserung in der sichtbaren Differenz zwischen
up und down Bilder festzustellen war, ist auf einen defekten Aufbau (etwa in der Befestigung der Spitze) zu schließen.

Der Gitterabstand von Gold beträgt gemäß \cite{gitterkonstanten} $G\ua{\ce{Au}} = \SI{4.08}{\angstrom}$. Die gefundene Plateaudifferenz
liegt bei $\SI{4.75(9)}{\angstrom}$. Die Höhendifferenz $h\ua{111}$ zwischen zwei $[111]$ Ebenen beträgt
\begin{equation}
  h\ua{111} = \frac{G\ua{\ce{Au}}}{\sqrt{1^2 + 1^2 + 1^2}} \approx \SI{2.35}{\angstrom}.
\end{equation}
Dies lässt darauf schließen, dass die untersuchte Kante durch eine Höhendifferenz von zwei Atomlagen
hervorgerufen wird. Dies deckt sich mit der Beobachtung, dass in Abbildung \ref{fig: au} weitere Kanten zu erkennen
sind, die weniger Kontrast auweisen als die vermessene Kante. Hierbei handelt es sich vermutlich um
einatomige Kanten.
Das Ergebnis ist in anbetracht der Tatsache,
dass die in Abbildung \ref{fig: höhenprofil} eingezeichnete Kante nicht wirklich ausgeprägt ist
und unter Berücksichtigung der genannten Fehlerquellen nicht signifikant.
