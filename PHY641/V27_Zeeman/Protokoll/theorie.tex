\setcounter{page}{1}
\section*{Zielsetzung}
In dem Versuch V27 soll der \emph{Zeeman-Effekt} experiementell beobachtet werden.
Hierzu wird die Aufspaltung der Energieniveaus einer Cadmium Spektrallampe
unter Einwirkung eines äußeren Magnetfeldes untersucht.

\section{Theorie}

\subsection{Magnetisches Moment und Energieaufspaltung}
\subsubsection{Magnetische Momente}
Aus der Quantenmechanik geht hervor das Elektronen neben ihrem Drehimpuls $\vec{l}$ auch einen
Eigendrehimpuls oder auch Spin gennant $\vec{s}$ besitzen. Charakterisiert werden diese beiden Größen durch die
Quantenzahlen $l$ und $s$ mit den aus den Eigenwertgleichung resultierenden Werten:
\begin{align*}
\be{\vec{l}}&=\sqrt{l(l+1)}\hbar & l&=0,1,2,\dots,n-1\\
\be{\vec{s}}&=\sqrt{s(s+1)}\hbar & s&=\frac{1}{2}.
\end{align*}
Auf Grund der Ladung des Elektrons $\map{e}_0$ entstehen magnetische Momente.
Die aus der Theorie resultierenden Momente sind proptotional zum Bohrschen Magneton $\map{\mu}_B$, welches durch
\begin{equation}
  \label{eq:bohrsche_magneton}
  \map{\mu_B}:=-\frac{1}{2}\map{e}\ua{0} \frac{\hbar}{\map{m}\ua{0}}
\end{equation}
definiert ist. Die in Gleichung \eqref{eq:bohrsche_magneton} auftretende Größe $\map{m}_0$ ist die Elektronenmasse.
Für die magnetischen Momente ergibt sich:
\begin{align}
  \vec{\mu}\ua{l}&=-\map{\mu_B}\frac{\vec{l}}{\hbar}=-\map{\mu_B}\sqrt{l(l+1)}\vec{e}\ua{\vec{l}} \\
  \vec{\mu}\ua{s}&=-\map{g_S}\map{\mu_B}\frac{\vec{s}}{\hbar}=-\map{g_S}\map{\mu_B}\sqrt{s(s+1)}\vec{e}\ua{\vec{s}}
\end{align}
Mit $\map{g_S}$ wird der Landé-Faktor des Elektrons bezeichnet und hat den Wert $\map{g_S}\approx 2$.
Es fällt auf, dass das magnetische Spinmoment $\vec{\mu}\ua{s}$ des Elektrons in etwa doppelt
so groß ist wie das magentische Bahnmoment $\vec{\mu}\ua{l}$.

\subsubsection{Magnetische Momente untereinander}





\subsection{Auswahlregel und Polarisation}
