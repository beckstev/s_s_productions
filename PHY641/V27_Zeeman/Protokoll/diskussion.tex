\section{Diskussion}
\begin{table}
  \caption{Zusammenfassung der Ergebnisse.}
  \label{tab: results}
  \begin{tabular}{l S S[table-format=1.3]@{${}\pm{}$} S[table-format=1.3] S}
    \toprule
    {Aufspaltung} & {g (theoretisch)} & \multicolumn{2}{c}{g (experimentell)} & {mittlere Abweichung / \%} \\
    \midrule
    rot $\sigma$  & 1 & 1.045 & 0.005 & 4.5  \\
    blau $\sigma$ & 2 & 2.02  & 0.02  & 1.0    \\
    blau $\pi$    & 1 & 0.581 & 0.002 & 16.2 \\
    \bottomrule
  \end{tabular}
\end{table}
Die gewonnenen experimentellen Werte für die Übergangs-Landé-Faktoren sind in Tabelle \ref{tab: results} zusammengefasst.
Die geringen relativen Fehler der experimentellen Werte sind auf die Tatsache zurückzuführen, dass von einer Betrachtung
der Ableseungenauigkeiten etwa bei der Aufnahme der Hysterese abgesehen wurde. Dennoch zeigt sich, dass die gefundenen Bestwerte
gut mit den theoretischen Werten übereinstimmen. Im Fall der blauen $\sigma$ Aufspaltung konnten lediglich zwei Linien
beobachtet werden. Der bestimmte Landé-Faktor $\num{2.02(2)}$ deutet darauf hin, dass offenbar der zugehörige theoretische Übergang
mit $g = 2$ wesentlich dominanter ist als jener mit $g = 1.5$. Zudem liegen diese so nah beieinander, dass sie mit dem verwendeten
Aufbau nicht zu unterscheiden sind. \\
Weiterhin ist zu beobachten, dass die Helligkeit der aufgespaltenen Linien höher ist als die der unaufgespaltenen (siehe etwa Abbildung
\ref{}). Dies zeigt, dass ein anliegendes Magnetfeld offenbar die Wahrscheinlichkeit der Lichtemission in der Cd-Lampe steigert.
