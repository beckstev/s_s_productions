\begin{table}
\centering
\caption{Aufgenommene Messdaten und berechnete Größen zur Untersuchung des Aluminiumwürfels.
Gemessene Counts $N$, skalierte Counts der Leermessung $N_0$ \eqref{eq: count_skalierung} und
berechnete Werte der Größe $y$ \eqref{eq: y}.}
\label{tab: alu}
\begin{tabular}{S S[table-format=2.0]@{ - } S[table-format=2.0] S S[table-format=4.0]@{${}\pm{}$} S[table-format=2.0] S[table-format=4.0]@{${}\pm{}$} S[table-format=2.0] S[table-format=1.2]@{${}\pm{}$} S[table-format=1.2] }
\toprule
{Projektion} & \multicolumn{2}{c}{Kanal} & {$t / \si{ \second}$} & \multicolumn{2}{c}{$N$} & \multicolumn{2}{c}{$N_0$} & \multicolumn{2}{c}{$y$} \\
\midrule
1 & 58 & 63 & 35.80 & 3245 & 57 & 5715 & 45 & 0.57 & 0.02\\
2 & 57 & 62 & 41.20 & 3420 & 58 & 6577 & 52 & 0.65 & 0.02\\
3 & 57 & 62 & 42.38 & 3424 & 59 & 6766 & 54 & 0.68 & 0.02\\
4 & 56 & 61 & 33.82 & 3624 & 60 & 5260 & 42 & 0.37 & 0.02\\
5 & 55 & 60 & 50.00 & 3661 & 61 & 7903 & 61 & 0.77 & 0.02\\
6 & 55 & 60 & 38.86 & 3492 & 59 & 6044 & 48 & 0.55 & 0.02\\
7 & 55 & 60 & 38.88 & 3392 & 58 & 6047 & 48 & 0.58 & 0.02\\
8 & 55 & 60 & 45.44 & 3367 & 58 & 7182 & 55 & 0.76 & 0.02\\
9 & 55 & 60 & 33.88 & 3659 & 60 & 5270 & 42 & 0.36 & 0.02\\
10 & 55 & 60 & 39.76 & 3289 & 57 & 6347 & 50 & 0.66 & 0.02\\
11 & 55 & 60 & 42.14 & 3468 & 59 & 6727 & 53 & 0.66 & 0.02\\
12 & 55 & 60 & 37.78 & 3408 & 58 & 6031 & 48 & 0.57 & 0.02\\
\bottomrule
\end{tabular}
\end{table}
