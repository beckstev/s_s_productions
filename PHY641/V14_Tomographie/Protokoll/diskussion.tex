\section{Diskussion}
Die gefundenen Werte für die Absorptionskoeffizienten \eqref{eq: exp_mu} weichen im Mittel
um $\SI{-7}{\percent}$ (Aluminium) bzw. $\SI{-25}{\percent}$ (Blei) von den Literaturwerten \eqref{eq: lit_mu} ab.
Die relativen statistischen Fehler der berechneten Mittelwerte liegen in einer Größenordnung von
$\SI{1}{\percent}$. Dies zeigt auf, dass die Diskrepanzen zu den Literaturwerten nur durch
systematische Fehler erklärbar sind.

Eine Fehlerquelle ist die manuelle Einstellung der Würfelposition. Insbesondere bei den Messungen
entlang der Diagonalen kann nicht sicher gestellt werden, dass der Strahl exakt der Projektion folgt,
die in Matrix \eqref{eq: matrix_absoprtionskoeffizienten} angenommen wird. Der Strahl besitzt zudem eine natürliche Breite, weshalb
ein Durchgang durch Elementarwürfel, die bei bestimmten Projektionen theoretisch nicht im Strahlengang liegen, zur
Abschwächung der Intensität beiträgt.

Als weiterer Aspekt ist das verwendete numerische Verfahren zu nennen. Für die Determinante der Matrix
$\left(\symbf{A}^\top \symbf{A}\right)$ gilt
\begin{equation}
  \mathup{det}\left\{\left(\symbf{A}^\top \symbf{A}\right)\right\} \approx \num{2.5e5} \gg 1.
\end{equation}
Dies ist ein Anzeichen für die numerische Stabilität des Verfahrens. Numerische Fluktuationen bei
Lösung der Normalengleichung sind also für die Diskussion der möglichen Fehler irrelevant.

Der Vergleich der gefundenen Werte der Absorptionskoeffizienten des unbekannten Würfels mit den
Litarturwerten (Tabelle~\ref{tab: delta_mu}) erlaubt die Interpretation, dass lediglich der
Würfel in der Mitte (Würfel 5) aus Blei besteht, während die umliegenden signifikant eher dem Wert %signifikant eher klingt nicht so gut
für Aluminium zugeordnet werden können. Gleiches bestätigt sich auch beim Vergleich mit den experimentellen
Werten \eqref{eq: exp_mu}. Jedoch ist hier z.B. bei Würfel 2 mit einer absoluten Abweichung von $\SI{0.34}{\centi\meter^{-1}}$ zu Aluminium
bzw. $\SI{0.4}{\centi\meter^{-1}}$ zu Blei keine signifikante Aussage möglich. Das numerische Verfahren liefert einen unphysikalischen %nicht wieder signifikant verwenden
negativen Wert für Würfel 9. Dies ist auf die bereis genannten Fehlerquellen zurück zu führen. % zurückzuführen, also zusammen geschrieben
Dennoch kann dieser Wert Aluminium zugeordnet werden, da ein Bleiwürfel an der selben Stelle schon während
der Messung durch erhöhte Messzeiten hätte auffallen müssen (vgl. Messzeiten Tabelle~\ref{tab: unb}).

Abschließend ist festzustellen, dass der Aufbau eine gute Möglichkeit bietet Stoffe zu unterscheiden, deren
Absorptionskoeffizienten, wie im Fall von Aluminium und Blei, deutlich voneinander abweichen. Zur Unterscheidung
von Matrialien ähnlicher Dichte ist zumindest eine genauere Justierung der Würfelposition notwendig.
Für eine quantitative Auswertung der Absorptionsspektren (etwa eine Vermessung der Compton-Kante) wäre zudem eine
Kallibrierung des Detektors durchzuführen.
