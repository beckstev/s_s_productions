\section{Auswertung}
Die nachfolgenden Operationen der linearen Algebra werden mit dem \emph{python}-Paket
\emph{numpy}\cite{} durchgeführt. Hierbei ist das Vorgehen bei allen drei Würfeln
analog. Die gemessenen Counts $N\ua{i}$ werden mit einem statistischen Fehler
von $\sigma_{N\ua{i}} = \sqrt{N\ua{i}}$ belegt. Mit den Ergebnissen aus
der Leermessung wird für jeden Wert $N\ua{i}$ die Größe
\begin{equation}
  y\ua{i} = \ln\left(\frac{N_0}{N\ua{i}} \right)
\end{equation}
berechnet, wobei $N_0$ der auf die entsprechende Messzeit normierte Wert der Leermessung ist.
Der Fehler ergibt sich gemäß Gaußscher Fehlerfortpflanzung zu
\begin{equation}
  \sigma_{y_i} = \sqrt{\left(\frac{\sigma_{N_0}}{N_0} \right)^2 +    \left(\frac{\sigma_{N\ua{i}}}{N\ua{i}} \right)^2}.
\end{equation}
Die Kovarianzmatrix $\symbf{V}\left[\vec{y}\right]$ des Vektors $\vec{y} = \left(y_1, \dots , y\ua{n} \right)^\top$ hat aufgrund der statistischen Unabhängigkeit der
Einzelmessungen eine diagonale Gestalt
\begin{equation}
  \symbf{V}\left[\vec{y}\right] = \mathup{diag}\left(\sigma_{y_1}^2, \dots , \sigma_{y\ua{n}}^2 \right).
\end{equation}
Der zu bestimmende Vektor $\vec{\mu} = \left(\mu_1, \dots , \mu_9 \right)^\top$ berechnet sich damit gemäß\footnote{Für Herleitung siehe z.B. \dots}
\begin{equation}
  \vec{\mu} = \left(A^\top W A\right)^{-1} W A^\top \vec{y}, \quad W = V^{-1}.
\end{equation}
Und die zugehörige Kovarianzmatrix $V[\vec{\mu}]$ gemäß
\begin{equation}
  \symbf{V}\left[\vec{\mu}\right] = \left( A^\top W  A\right)^{-1}
\end{equation}
Die Fehler $\sigma_{\mu\ua{i}}$der Absorptionskoeffizienten berechnen sich aus den Wurzeln der Diagonalemente der Kovarianzmatrix
\begin{equation}
 \sigma_{\mu\ua{i}} = \sqrt{\left(V[\vec{\mu}]\right)\ua{ii}}.
\end{equation}





\subsection{Leermessung}
Die gemessenen Daten der Leermessung sind in Tabelle~\ref{tab: leermessung} aufgeführt.
%\input{}


\subsection{Untersuchung des Aluminium-Würfels}
Die aufgenommenen Daten und berechneten Größen sind in Tabelle~\ref{tab: alu} einzusehen.
Als gewichteter Mittelwert des Absorptionskoeffizienten $\mu_{\ce{Al}}$ ergibt sich
\begin{equation}
  \mu_{\ce{Al}} = \dots .
\end{equation}
Der Literaturwert $\mu_{\ce{Al}, \mathup{lit}}$ beträgt gemäß Quelle \cite{}
\begin{equation}
  \mu_{\ce{Al}, \mathup{lit}} = \dots .
\end{equation}

\subsection{Untersuchung des Blei-Würfels}
Die aufgenommenen Daten und berechneten Größen sind in Tabelle~\ref{tab: blei} einzusehen.
%\input{}
Als gewichteter Mittelwert des Absorptionskoeffizienten $\mu_{\ce{Pb}}$ ergibt sich
\begin{equation}
  \mu_{\ce{Pb}} = \dots .
\end{equation}
Der Literaturwert $\mu_{\ce{Pb}, \mathup{lit}}$ beträgt gemäß Quelle \cite{}
\begin{equation}
  \mu_{\ce{Pb}, \mathup{lit}} = \dots .
\end{equation}


\subsection{Untersuchung des Unbekannten Würfels}
Die aufgenommenen Daten sind in Tabelle~\ref{tab: unb} einzusehen.
