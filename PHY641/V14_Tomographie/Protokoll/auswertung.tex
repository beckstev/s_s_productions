\section{Auswertung}
Die nachfolgenden Operationen der linearen Algebra werden mit dem \emph{python}-Paket
\emph{numpy}\cite{} durchgeführt. Hierbei ist das Vorgehen bei allen drei Würfeln
analog. Die gemessenen Counts $N\ua{i}$ werden mit einem statistischen Fehler
von $\sigma_{N\ua{i}} = \sqrt{N\ua{i}}$ belegt. Mit den Ergebnissen aus
der Leermessung wird für jeden Wert $N\ua{i}$ die Größe
\begin{equation}
  y\ua{i} = \ln\left(\frac{N\ua{0, i}}{N\ua{i}} \right)
  \label{eq: y}
\end{equation}
berechnet, wobei $N\ua{0, i}$ der gemäß
\begin{equation}
  N\ua{0, i} = N\ua{0} \frac{t\ua{i}}{t_0}
  \label{eq: count_skalierung}
\end{equation}
auf die entsprechende Messzeit $t\ua{i}$ normierte Wert der Leermessung $(N\ua{0}, t_0)$ ist.
Der Fehler ergibt sich gemäß Gaußscher Fehlerfortpflanzung zu
\begin{equation}
  \sigma_{y_i} = \sqrt{\left(\frac{\sigma_{N\ua{0, i}}}{N\ua{0, i}} \right)^2 +    \left(\frac{\sigma_{N\ua{i}}}{N\ua{i}} \right)^2}.
\end{equation}
Die Kovarianzmatrix $\symbf{V}\left[\vec{y}\right]$ des Vektors $\vec{y} = \left(y_1, \dots , y\ua{n} \right)^\top$ hat aufgrund der statistischen Unabhängigkeit der
Einzelmessungen eine diagonale Gestalt
\begin{equation}
  \symbf{V}\left[\vec{y}\right] = \mathup{diag}\left(\sigma_{y_1}^2, \dots , \sigma_{y\ua{n}}^2 \right).
\end{equation}
Der zu bestimmende Vektor $\vec{\mu} = \left(\mu_1, \dots , \mu_9 \right)^\top$ berechnet sich damit gemäß\footnote{Für Herleitung siehe z.B. \dots}
\begin{equation}
  \vec{\mu} = \left(\symbf{A}^\top \symbf{W} \symbf{A}\right)^{-1}  \symbf{A}^\top \symbf{W} \, \vec{y}, \quad \symbf{W} = \symbf{V}^{-1}.
  \label{eq: mu}
\end{equation}
Und die zugehörige Kovarianzmatrix $\symbf{V}[\vec{\mu}]$ gemäß
\begin{equation}
  \symbf{V}\left[\vec{\mu}\right] = \left( \symbf{A}^\top \symbf{W}  \symbf{A}\right)^{-1}
\end{equation}
Die Fehler $\sigma_{\mu\ua{i}}$der Absorptionskoeffizienten berechnen sich aus den Wurzeln der Diagonalemente der Kovarianzmatrix
\begin{equation}
 \sigma_{\mu\ua{i}} = \sqrt{\left(V[\vec{\mu}]\right)\ua{ii}}.
 \label{eq: sigma_mu}
\end{equation}

\subsection{Leermessung}
Die Daten der Leermessung sind in Tabelle~\ref{tab: leermessung} aufgeführt. Hierbei sind zu jeder Projektion
die Kanäle angegeben, über die integriert wird. Die Messung von Projektion $1$ wird representativ für die
Projektionen 2, 3, 10, 11, 12 verwendet, die von $4$ für 6, 7, 9 und die von $8$ für 5.
\begin{table}
\centering
\caption{Aufgenommene Daten der Leermessung. Messzeit $t$, Counts $N_0$.}
\label{tab: leer}
\begin{tabular}{S S[table-format=2.0]@{ - } S[table-format=2.0] S S[table-format=5.0]@{${}\pm{}$} S[table-format=3.0] }
\toprule
{Projektion} & \multicolumn{2}{c}{Kanal} & {$t / \si{\second}$} & \multicolumn{2}{c}{$N_0$} \\
\midrule
1 & 53 & 58 & 100.00 & 15964 & 126\\
4 & 53 & 58 & 100.64 & 15653 & 125\\
8 & 53 & 58 & 106.80 & 16881 & 130\\
\bottomrule
\end{tabular}
\end{table}


\subsection{Untersuchung des Aluminium- und Blei-Würfels}
Die aufgenommenen Daten und berechneten Größen sind in den Tabellen~\ref{tab: alu} und \ref{tab: blei} einzusehen.
Die gemäß der Gleichungen \eqref{eq: mu} und \eqref{eq: sigma_mu} bestimmten Absorptionskoeffizienten
der einzelnen Elementarwürfel sind in Tabelle~\ref{tab: results_mu} aufgeführt.
\FloatBarrier
\begin{table}
\centering
\caption{Aufgenommene Messdaten und berechnete Größen zur Untersuchung des Aluminiumwürfels.
Gemessene Counts $N$, skalierte Counts der Leermessung $N_0$ \eqref{eq: count_skalierung} und
berechnete Werte der Größe $y$ \eqref{eq: y}.}
\label{tab: alu}
\begin{tabular}{S S[table-format=2.0]@{ - } S[table-format=2.0] S S[table-format=4.0]@{${}\pm{}$} S[table-format=2.0] S[table-format=4.0]@{${}\pm{}$} S[table-format=2.0] S[table-format=1.2]@{${}\pm{}$} S[table-format=1.2] }
\toprule
{Projektion} & \multicolumn{2}{c}{Kanal} & {$t / \si{ \second}$} & \multicolumn{2}{c}{$N$} & \multicolumn{2}{c}{$N_0$} & \multicolumn{2}{c}{$y$} \\
\midrule
1 & 58 & 63 & 35.80 & 3245 & 57 & 5715 & 45 & 0.57 & 0.02\\
2 & 57 & 62 & 41.20 & 3420 & 58 & 6577 & 52 & 0.65 & 0.02\\
3 & 57 & 62 & 42.38 & 3424 & 59 & 6766 & 54 & 0.68 & 0.02\\
4 & 56 & 61 & 33.82 & 3624 & 60 & 5260 & 42 & 0.37 & 0.02\\
5 & 55 & 60 & 50.00 & 3661 & 61 & 7903 & 61 & 0.77 & 0.02\\
6 & 55 & 60 & 38.86 & 3492 & 59 & 6044 & 48 & 0.55 & 0.02\\
7 & 55 & 60 & 38.88 & 3392 & 58 & 6047 & 48 & 0.58 & 0.02\\
8 & 55 & 60 & 45.44 & 3367 & 58 & 7182 & 55 & 0.76 & 0.02\\
9 & 55 & 60 & 33.88 & 3659 & 60 & 5270 & 42 & 0.36 & 0.02\\
10 & 55 & 60 & 39.76 & 3289 & 57 & 6347 & 50 & 0.66 & 0.02\\
11 & 55 & 60 & 42.14 & 3468 & 59 & 6727 & 53 & 0.66 & 0.02\\
12 & 55 & 60 & 37.78 & 3408 & 58 & 6031 & 48 & 0.57 & 0.02\\
\bottomrule
\end{tabular}
\end{table}

\begin{table}
\centering
\caption{Aufgenommene Messdaten und berechnete Größen zur Untersuchung des Bleiwürfels.
Gemessene Counts $N$, skalierte Counts der Leermessung $N_0$ \eqref{eq: count_skalierung} und
berechnete Werte der Größe $y$ \eqref{eq: y}.}
\label{tab: blei}
\begin{tabular}{S S[table-format=2.0]@{ - } S[table-format=2.0] S S[table-format=4.0]@{${}\pm{}$} S[table-format=2.0] S[table-format=6.0]@{${}\pm{}$} S[table-format=3.0] S[table-format=1.2]@{${}\pm{}$} S[table-format=1.2] }
\toprule
{Projektion} & \multicolumn{2}{c}{Kanal} & {$t / \si{ \second}$} & \multicolumn{2}{c}{$N$} & \multicolumn{2}{c}{$N_0$} & \multicolumn{2}{c}{$y$} \\
\midrule
1 & 54 & 59 & 171.60 & 2153 & 46 & 27394 & 217 & 2.54 & 0.02\\
2 & 54 & 59 & 359.30 & 2031 & 45 & 57359 & 454 & 3.34 & 0.02\\
3 & 54 & 59 & 338.58 & 2058 & 45 & 54051 & 428 & 3.27 & 0.02\\
4 & 54 & 59 & 91.94 & 2076 & 46 & 14300 & 114 & 1.93 & 0.02\\
5 & 54 & 59 & 680.72 & 2014 & 45 & 107596 & 828 & 3.98 & 0.02\\
6 & 54 & 59 & 233.70 & 2103 & 46 & 36348 & 291 & 2.85 & 0.02\\
7 & 54 & 59 & 203.34 & 2066 & 45 & 31626 & 253 & 2.73 & 0.02\\
8 & 54 & 59 & 699.96 & 2038 & 45 & 110637 & 852 & 3.99 & 0.02\\
9 & 54 & 59 & 99.48 & 2193 & 47 & 15473 & 124 & 1.95 & 0.02\\
10 & 54 & 59 & 344.66 & 2175 & 47 & 55022 & 435 & 3.23 & 0.02\\
11 & 54 & 59 & 365.08 & 2128 & 46 & 58281 & 461 & 3.31 & 0.02\\
12 & 53 & 58 & 141.28 & 2198 & 47 & 22554 & 179 & 2.33 & 0.02\\
\bottomrule
\end{tabular}
\end{table}


Als gewichtete Mittelwerte der Absorptionskoeffizienten für Aluminium $\mu_{\ce{Al}}$ und Blei $\mu_{\ce{Pb}}$
ergeben sich
\begin{align}
  \begin{aligned}
  \mu_{\ce{Al}} &= \SI{0.188(3)}{\per\centi\meter}\\
  \mu_{\ce{Pb}} &= \SI{0.934(4)}{\per\centi\meter}.
  \label{eq: exp_mu}
\end{aligned}
\end{align}
Die Literaturwerte $\mu_{\ce{Al}, \mathup{lit}}$ und $\mu_{\ce{Pb}, \mathup{lit}}$ betragen gemäß Quelle \cite{mu}
\begin{align}
  \begin{aligned}
  \mu_{\ce{Al}, \mathup{lit}} &= \SI{0.202}{\per\centi\meter} \\
  \mu_{\ce{Pb}, \mathup{lit}} &= \SI{1.245}{\per\centi\meter}.
  \label{eq: lit_mu}
\end{aligned}
\end{align}
\begin{table}
\centering
\caption{Bestimmte Absorptionskoeffizienten für Aluminium und Blei.}
\label{tab: results_mu}
\begin{tabular}{S S[table-format=1.2]@{${}\pm{}$} S[table-format=1.2] S[table-format=1.2]@{${}\pm{}$} S[table-format=1.2] }
\toprule
{Würfel} & \multicolumn{2}{c}{$\mu_{\ce{Al}} \:/\: \si{ \centi\meter^{-1}}$} & \multicolumn{2}{c}{$\mu_{\ce{Pb}} \:/\: \si{ \centi\meter^{-1}}$} \\
\midrule
1 & 0.22 & 0.01 & 1.10 & 0.02\\
2 & 0.12 & 0.01 & 0.61 & 0.01\\
3 & 0.18 & 0.01 & 0.62 & 0.01\\
4 & 0.19 & 0.01 & 0.86 & 0.01\\
5 & 0.20 & 0.01 & 1.20 & 0.01\\
6 & 0.17 & 0.01 & 0.90 & 0.01\\
7 & 0.21 & 0.01 & 1.16 & 0.02\\
8 & 0.26 & 0.01 & 1.22 & 0.01\\
9 & 0.17 & 0.01 & 0.69 & 0.02\\
\bottomrule
\end{tabular}
\end{table}

\FloatBarrier

\subsection{Untersuchung des Unbekannten Würfels}
Die aufgenommenen Daten sind in Tabelle~\ref{tab: unb} einzusehen. Zur Quantitativen Diskussion der Zusammensetzung
des unbekannten Würfels sind in Tabelle~\ref{tab: delta_mu} neben den bestimmten Absorptionskoeffizienten die mittleren Absoluten
Abweichungen zu den Literaturwerten \eqref{eq: lit_mu} und experimentell bestimmten Werten \eqref{eq: exp_mu} eingetragen.
\FloatBarrier
\begin{table}
\centering
\caption{Aufgenommene Messdaten und berechnete Größen zur Untersuchung des Unbekannten Würfels.
Gemessene Counts $N$, skalierte Counts der Leermessung $N_0$ \eqref{eq: count_skalierung} und
berechnete Werte der Größe $y$ \eqref{eq: y}.}
\label{tab: unb}
\begin{tabular}{S S[table-format=2.0]@{${}\pm{}$} S[table-format=2.0] S S[table-format=5.0]@{${}\pm{}$} S[table-format=3.0] S[table-format=6.0]@{${}\pm{}$} S[table-format=3.0] S[table-format=1.2]@{${}\pm{}$} S[table-format=1.2] }
\toprule
{Projektion} & \multicolumn{2}{c}{Kanal} & {$t / \si{ \second}$} & \multicolumn{2}{c}{$N$} & \multicolumn{2}{c}{$N_0$} & \multicolumn{2}{c}{$y$} \\
\midrule
1 & 53 & 58 & 258.66 & 10231 & 101 & 41292 & 327 & 1.40 & 0.01\\
2 & 53 & 58 & 306.36 & 10203 & 101 & 48907 & 387 & 1.57 & 0.01\\
3 & 54 & 59 & 149.74 & 11765 & 108 & 23904 & 189 & 0.71 & 0.01\\
4 & 54 & 59 & 121.24 & 10242 & 101 & 18857 & 151 & 0.61 & 0.01\\
5 & 54 & 59 & 592.14 & 10592 & 103 & 93595 & 720 & 2.18 & 0.01\\
6 & 54 & 59 & 139.20 & 10277 & 101 & 21650 & 173 & 0.75 & 0.01\\
7 & 54 & 59 & 138.62 & 11821 & 109 & 21560 & 172 & 0.60 & 0.01\\
8 & 53 & 58 & 516.40 & 11083 & 105 & 81623 & 628 & 2.00 & 0.01\\
9 & 53 & 58 & 158.08 & 10175 & 101 & 24587 & 197 & 0.88 & 0.01\\
10 & 53 & 58 & 127.00 & 10735 & 104 & 20274 & 160 & 0.64 & 0.01\\
11 & 53 & 58 & 700.76 & 10195 & 101 & 111869 & 885 & 2.40 & 0.01\\
12 & 53 & 58 & 124.32 & 10874 & 104 & 19846 & 157 & 0.60 & 0.01\\
\bottomrule
\end{tabular}
\end{table}

\begin{table}
\centering
\caption{Bestimmte Absorptionskoeffizienten des unbekannten Würfels mit absoluten Abweichungen zu den theoretischen, bzw. experimentell bestimmten Werten.}
\label{tab: delta_mu}
\begin{tabular}{S S[table-format=1.2]@{${}\pm{}$} S[table-format=1.2] S S S S }
\toprule
{Würfel} & \multicolumn{2}{c}{$\mu \:/\: \si{\centi\meter^{-1}}$} & {$\Delta\mu_{\ce{Al}, exp} \:/\: \si{\centi\meter^{-1}}$} & {$\Delta\mu_{\ce{Al}, lit} \:/\: \si{\centi\meter^{-1}}$} & {$\Delta\mu_{\ce{Pb}, exp} \:/\: \si{\centi\meter^{-1}}$} & {$\Delta\mu_{\ce{Pb}, lit} \:/\: \si{\centi\meter^{-1}}$} \\
\midrule
1 & 0.38 & 0.01 & 0.19 & 0.18 & 0.55 & 0.86\\
2 & 0.53 & 0.01 & 0.34 & 0.33 & 0.40 & 0.71\\
3 & 0.36 & 0.01 & 0.17 & 0.15 & 0.58 & 0.89\\
4 & 0.01 & 0.01 & 0.18 & 0.19 & 0.92 & 1.24\\
5 & 1.16 & 0.01 & 0.97 & 0.96 & 0.22 & 0.09\\
6 & 0.16 & 0.01 & 0.03 & 0.04 & 0.77 & 1.09\\
7 & 0.14 & 0.01 & 0.05 & 0.06 & 0.80 & 1.11\\
8 & 0.48 & 0.01 & 0.29 & 0.28 & 0.46 & 0.77\\
9 & -0.02 & 0.01 & 0.21 & 0.22 & 0.95 & 1.26\\
\bottomrule
\end{tabular}
\end{table}

\FloatBarrier
