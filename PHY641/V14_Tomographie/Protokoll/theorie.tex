\setcounter{page}{1}
\section*{Zielsetzung}

In dem Versuch B14 soll mit Hilfe von $\gamma$-Strahlung die
Materialzusammensetzung von Würfeln bestimmt werden.
Hiermit sollen die Grundlagen der Tomographie erlernt werden.

\section{Theorie}

Die Tomographie beruht Grundlegend auf der Tatsache das Strahlung die
durch ein Material propegiert an Intensität verliert.
Wie viel Intensität beim durchqueren des Materials verloren geht wird
mit dem materialspezifischen \emph{Absorptionskoeffizient} $\mu$ angegeben.
Dringt die Strahlung durch mehrere Materialien hintereinander so
folgt für die Intensität der folgende Zusammenhang:

\begin{equation}
  \label{eq: Intensitaet}
  N=I_0 \mathrm{exp}\left( - \sum_i \mu_i d_i \right).
\end{equation}
\textbf{Indicierung mit Stefans vergleichen}
Hierbei ist $I_0$ die Intensität vor dem Eindringe, $\mu_i$ der
Absorptionskoeffizient des jeweiligen Material und $d_i$ die
Wegstrecke die im Material zurückgelegt wird.
Gleichung \eqref{eq: Intensitaet} kann umgestellt werden, um so den Absorptionskoeffizient zu bestimmen

\begin{equation}
  \label{eq: I_umge}
  \eqref{eq: Intensitaet} \quad \Leftrightarrow \quad  \sum_i \mu_i d_i = \ln\left( \frac{ I_0}{N_j} \right).
\end{equation}

Auf die Bedeutung des Index $j$ bei $N_j$ wird in dem Kapitel \emph{Durchführung} erläutert.
