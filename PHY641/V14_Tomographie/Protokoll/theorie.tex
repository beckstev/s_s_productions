\setcounter{page}{1}
\section*{Zielsetzung}

In dem Versuch B14 soll mit Hilfe von $\gamma$-Strahlung die
Materialzusammensetzung von verschiedenen Würfeln bestimmt werden.
Hiermit sollen die Grundlagen der Tomographie erlernt werden.

\section{Theorie}

Die Tomographie beruht grundlegend auf der Tatsache das Strahlung, die %, dass
durch ein Material propagiert an Intensität verliert. %komma
Wie viel Intensität beim Durchqueren des Materials verloren geht, wird
mit dem materialspezifischen \emph{Absorptionskoeffizienten} $\mu$ angegeben.
Dringt die Strahlung durch mehrere Materialien hintereinander, so
folgt für die Intensität der folgende Zusammenhang:

\begin{equation}
  \label{eq: Intensitaet}
  N=I_0 \mathrm{exp}\left( - \sum_i \mu_i d_i \right).
\end{equation}
Hierbei ist $I_0$ die Intensität vor dem Eindringen, $\mu_i$ der
Absorptionskoeffizient des jeweiligen Materials und $d_i$ die
Wegstrecke, die im Material zurückgelegt wird.
Gleichung \eqref{eq: Intensitaet} kann umgestellt werden, um so den Absorptionskoeffizienten zu bestimmen

\begin{equation}
  \label{eq: I_umge}
  \eqref{eq: Intensitaet} \quad \Leftrightarrow \quad  \sum_k \mu_k d_k = \ln\left( \frac{ I_0}{N_i} \right)=y_i.
\end{equation}

Auf die Bedeutung des Index $i$ bei $N_i$ wird in dem Kapitel \emph{Versuchsaufbau/-durchführung} erläutert. %eingegangen
