\begin{table} 
\centering 
\caption{Gemessene Drücke bei der Leckkratenmethode für die Drehschieberpumpe mit $p_{\mathrm{g}}=\SI{0.1\pm0.02}{\milli\bar}$. Messung bei Raumtemperatur.} 
\label{tab: leck_dreh_leck_0.1.pdf} 
\begin{tabular}{S[table-format=1.2]@{${}\pm{}$} S[table-format=1.2] S S S S[table-format=1.2]@{${}\pm{}$} S[table-format=1.2] } 
\toprule  
\multicolumn{2}{c}{$p \:/\: \si{ \milli\bar}$} & {$t_1 / \si{ \second}$} & {$t_2 / \si{ \second}$} & {$t_3 / \si{ \second}$} & \multicolumn{2}{c}{$\overline{t} \:/\: \si{ \second}$} \\ 
\midrule  
0.10 & 0.02 & 0.0 & 0.0 & 0.0 & 0.00 & 0.00\\ 
0.20 & 0.04 & 10.3 & 10.8 & 11.5 & 10.87 & 0.34\\ 
0.40 & 0.08 & 50.4 & 50.9 & 51.9 & 51.04 & 0.42\\ 
0.60 & 0.12 & 105.3 & 104.5 & 107.3 & 105.74 & 0.83\\ 
0.80 & 0.16 & 155.4 & 156.3 & 156.7 & 156.12 & 0.36\\ 
1.00 & 0.20 & 199.9 & 199.9 & 201.0 & 200.28 & 0.37\\ 
\bottomrule 
\end{tabular} 
\end{table}
