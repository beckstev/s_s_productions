\begin{table} 
\centering 
\caption{Gemessene Drücke bei der Leckkratenmethode für die Drehschieberpumpe mit $p_{\mathrm{l}}=0.1$. Messung bei Raumtemperatur.} 
\label{tab: leck_dreh_leck_0.1.pdf} 
\begin{tabular}{S[table-format=1.1]@{${}\pm{}$} S[table-format=1.1] S S S S[table-format=1.1]@{${}\pm{}$} S[table-format=1.1] } 
\toprule  
\multicolumn{2}{c}{$p \:/\: \si{ \milli\bar}$} & {$t_1 / \si{ \second}$} & {$t_2 / \si{ \second}$} & {$t_3 / \si{ \second}$} & \multicolumn{2}{c}{$\overline{t} \:/\: \si{ \second}$} \\ 
\midrule  
0.1 & 0.0 & 0.0 & 0.0 & 0.0 & 0.0 & 0.0\\ 
0.2 & 0.0 & 10.3 & 10.8 & 11.5 & 10.9 & 0.3\\ 
0.4 & 0.1 & 50.4 & 50.9 & 51.9 & 51.0 & 0.4\\ 
0.6 & 0.1 & 105.3 & 104.5 & 107.3 & 105.7 & 0.8\\ 
0.8 & 0.2 & 155.4 & 156.3 & 156.7 & 156.1 & 0.4\\ 
1.0 & 0.2 & 199.9 & 199.9 & 201.0 & 200.3 & 0.4\\ 
\bottomrule 
\end{tabular} 
\end{table}