\section{Diskussion}

Abschließend sollen die Aussagekraft der aus den Versuch resultierenden Ergebnisse diskutiert werden.

Zu Beginn ist die Volumenmessung zu bemägeln, denn insbesondere bei der Vermessung des Tankes
konnten die Radien nicht genau vermessen werden. Damit kann erklärt werden, warum sich das experimentell
bestimmte Volumen \eqref{eq:volumen Tank} von dem, in der Anleitung \cite{} angegebene Volumen
$V\ua{tank}=\SI{9.5\pm0,8}{\litre}$ unterscheidet. Lediglich im Fehlerbeich beider Volumina
kommt es zu einer Überschneidung.
Weiterhin wird die Signifikanz der Druckmessung durch zwei Faktoren vermindert.
Zum Einem besitzen die Messgeräte einen Messfehler, das Penning-Messgerät (in Abb. \ref{fig: aufbau} die \circled{10}) verursacht einen Fehler von $20\%$ und das
Glühkathodemessgerät (in Abb. \ref{fig: aufbau} die \circled{4}) besitzt einen Fehler von $10\%$. Die Auswirkung der großen Messfehler
ist insbesondere in den Abbildung \ref{fig: leck_dreh_1}, \ref{fig: leck_dreh_2}, \ref{fig: leck_turbo_1_2} und \ref{fig: leck_turbo_2_2}.
Durch eine Verringerung der Messungungenauigkeit, würde die Signifikanz der Ergebnisse erheblich verbessert werden.
Zum Anderen ist die Zeitmessung mit einem Fehler verbunden. Dieser kann durch eine Automatisierung der Zeitmessung verringert werden.

Der Unterschied zwischen den experimentell bestimmten Saugvermögen und der Herstellerangabe (vgl. den Plot \ref{fig: dreh_druck_leck_mit}),
lässt sich damti erklären das die Hersteller unter idealen Bedingung die Pumpen vermessen. Hiezu gehöhrt das das Rohr welches an die Pumpe angeflanscht wird,
einen genauso großen Durchmesser wie die Pumpe besitzt. Wie in der Abbildung \ref{fig: aufbau} zu erkennen ist, wird an der Turbopumpe \circled{8} direkt der
Rohrdurchmesser verkleinert. Im Gegensatz dazu, ist das bei der Drehschieberpumpe nicht der Fall (vgl. Foto \ref{fig: aufbaudrehschieber} die \circled{17}),
hier ist die Differenz zwischen dem experimentell errechneten Wert und der Herstellerangabe klein (vgl. Plot \ref{fig: dreh_druck_leck}).

Obendrein wird mit dem Versuch deutlich, dass das Saugvermögen eine größe ist die vom Druck abhängt, dies wird insbesondere
bei der Drehschieberpumpe \ref{fig: druck_dreh} deutlich. 

Der verwendete Versuchsaufbau bietet eine gute Möglichkeit das Saugvermögen der Drehschieberpumpe zu untersuchen.
Für eine genauere Untersuchung der Turbopumpe muss der Aufbau modifiziert werden, das heißt es sollten Rohre mit
 dem selben Rohrdurchmesser wie die Turbopumpe verwendet werden.
