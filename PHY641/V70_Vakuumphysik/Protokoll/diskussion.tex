\section{Diskussion}

Abschließend soll die Aussagekraft der Ergebnisse diskutiert werden. %dem, vielleicht einfach nur 'der Ergebnisse'

Zu Beginn ist die Volumenmessung zu bemängeln, denn insbesondere bei der Vermessung des Tankes
konnten die Radien nicht genau vermessen werden. Damit kann erklärt werden, warum sich das experimentell
bestimmte Volumen \eqref{eq:volumen Tank} von dem, in der Anleitung \cite{anleitung70} angegebene Volumen %cite
$V\ua{tank}=\SI{9.5\pm0,8}{\litre}$ unterscheidet. Lediglich im Fehlerbereich beider Volumina
kommt es zu einer Überschneidung.
Weiterhin wird die Signifikanz der Druckmessung durch zwei Faktoren vermindert.
Zum einen besitzen die Messgeräte einen Messfehler, das Penning-Messgerät (in Abb. \ref{fig: aufbau}) verursacht einen Fehler von $20\%$ und das %einen
Glühkathodenmessgerät (in Abb. \ref{fig: aufbau} ) besitzt einen Fehler von $10\%$. Die Auswirkung der großen Messfehler %kathoden
ist insbesondere in den Abbildungen \ref{fig: leck_dreh_1}, \ref{fig: leck_dreh_2}, \ref{fig: leck_turbo_1_2} und \ref{fig: leck_turbo_2_2} deutlich erkennbar. %abbildungen
Durch eine Verringerung der Messungenauigkeit, würde die Signifikanz der Ergebnisse erheblich verbessert werden.
Zum anderen ist die Zeitmessung mit einem Fehler verbunden. Dieser kann durch eine Automatisierung der Zeitmessung verbessert werden.

Der Unterschied zwischen den experimentell bestimmten Saugvermögen und der Herstellerangabe (vgl. den Plot \ref{fig: dreh_druck_leck_mit}),
lässt sich damit erklären, dass die Hersteller unter idealen Bedingungen Pumpen vermessen. Hiezu gehöhrt unter anderem, dass das Rohr welches an die Pumpe angeflanscht wird,
einen genauso großen Durchmesser wie die Pumpe besitzt. Wie in der Abbildung \ref{fig: aufbau} zu erkennen ist, wird an der Turbopumpe direkt der
Rohrdurchmesser verkleinert. Im Gegensatz zur Drehschieberpumpe (vgl. Foto \ref{fig: aufbaudrehschieber}),
hier ist Differenz der beiden Durchmesser klein. Somit wird die Drehschieberpumpe unter idealen Bedingungen betrieben, was sich auch in den %ideal ist absolut
Unterschieden zwischen dem experimentellen Wert für das Saugvermögen und der Herstellerangabe wieder spiegelt (vgl. Plot \ref{fig: dreh_druck_leck}). %experimentellen

Außerdem wird mit dem Versuch deutlich, dass das Saugvermögen eine druckabhängige Größe ist, dies wird insbesondere %obendrein umgspr.
bei der Drehschieberpumpe (vgl. \ref{fig: druck_dreh}) deutlich.

Der verwendete Versuchsaufbau bietet eine gute Möglichkeit das Saugvermögen der Drehschieberpumpe zu untersuchen.
Für eine genauere Untersuchung der Turbopumpe muss der Aufbau modifiziert werden, das heißt, es sollten Rohre mit
demselben Rohrdurchmesser wie die Turbopumpe verwendet werden. 
Jedoch beitet der Versuch einne gute Möglichkeit die Grundlagen der Vakuumphysik kennenzulernen. 
%war ja eigentlich nicht ziel des versuchs das saugvermögen genauso zu bestimmen, damit es mit der herstelerangabe zusammen passt; daher der Leitwert, vielleicht hier nochmal erwähnen, dass wir grundlagen der vakuumtechnik kennengelernt haben und die theoretischen erwartungen bestätigt wurden

%aus pdf: in den mehrfachplots vielleicht die Schriftgröße anpassen, da die sonst zu klein werden. die \circled zahlen würde ich weglassen, absätze überprüfen
