\documentclass[parskip=half, bibliography=totoc, captions=tableheading, titlepage=firstiscover]{scrartcl} %Ich habe [parskip=half] hinzugefügt
\usepackage{scrpage2}
\def\preambleloaded{Precompiled preamble loaded.}

\usepackage{fixltx2e}

%\pagestyle{headings}

%\pagestyle{headings}
\usepackage{scrpage2}
\clearscrheadings
\pagestyle{scrheadings}
\cfoot[\pagemark]{\pagemark}
\ofoot[]{}
\ifoot[]{}

%\usepackage{polyglossia}
%\setmainlanguage{german}

\usepackage{caption}
\usepackage{amsmath}
\usepackage{amssymb}
\usepackage{mathtools}

%\usepackage{fontspec}
%\defaultfontfeatures{Ligatures=TeX}

%\usepackage[
%  math-style=ISO,
%  bold-style=ISO,
%  sans-style=italic,
%  nabla=upright,
%]{unicode-math}
%\setmathfont{Latin Modern Math}

\usepackage[autostyle]{csquotes}

\usepackage[
  locale=DE,                   % deutsche Einstellungen
  separate-uncertainty=true,   % Immer Fehler mit \pm
  per-mode=symbol-or-fraction, % m/s im Text, sonst Brüche
]{siunitx}

\usepackage{xfrac}

\usepackage[section, below]{placeins}
\usepackage[
  labelfont=bf,
  font=small,
  width=0.9\textwidth,
]{caption}

\usepackage{subcaption}
%\usepackage{subfig}

\usepackage{graphicx}

\usepackage{float}
\floatplacement{figure}{h}
\floatplacement{table}{h}

\usepackage{booktabs}



\usepackage{bookmark}

\usepackage[shortcuts]{extdash}

\usepackage[math]{blindtext}

%\usepackage{microtype}

\usepackage[
  backend=biber,
]{biblatex}
% Quellendatenbank
\addbibresource{lit.bib}

\usepackage{hyperref}

\usepackage{color} % Das ist Geschmacksfrage

\usepackage{makeidx} %Ich habe makeidx hinzugefügt + makeindex
\makeindex

\usepackage[version=3]{mhchem} % für Thermodynamik-chemische Elemente
\usepackage{enumitem} %Ich habe enumitem hinzugefügt
\usepackage{expl3}
\usepackage{xparse}

\RequirePackage{etoolbox}
\AtEndPreamble{
    \usepackage{fontspec}
    \usepackage{microtype}
    \usepackage{fontspec}
    \defaultfontfeatures{Ligatures=TeX}
    \usepackage{polyglossia}
    \setmainlanguage{german}
    \usepackage[
      math-style=ISO,
      bold-style=ISO,
      sans-style=italic,
      nabla=upright,
      ]{unicode-math}
    \setmathfont{Latin Modern Math}

}


\NewDocumentCommand \dif {m}
{
\mathinner{\symup{d} #1}
}
%\usepackage{subcaption}
\usepackage{yfonts}
\def\preambleloaded{Precompiled preamble loaded.}

%\usepackage{showframe}
\newcommand{\versuch}{Grundlagen der Ultraschalltechnik}
\newcommand{\vnr}{US1}
\newcommand{\vd}{Tag der Durchführung: 06.06.17}
\newcommand{\va}{Tag der Abgabe: 13.06.17}
\newcommand{\map}[1]{\mathup{#1}}
\newcommand{\ua}[1]{_{\mathup{#1}}}
\newcommand{\be}[1]{\left|\,#1\,\right|}
\newcommand{\norm}[1]{\|\,#1\,\|}
\newcommand{\ov}[1]{\overline{#1}}
\newcommand{\dv}[1]{\,\mathup{d}#1}

\author{Steven Becker \\
steven.becker@tu-dortmund.de \\
und \\
Stefan Grisard \\
stefan.grisard@tu-dortmund.de}

\title{\versuch}
\subtitle{Versuch \vnr}

\date{\vd \\
\va}
