\newpage
\section{Diskussion}
Die gefundenen Werte für die Schallgeschwindigkeit $c\ua{E} = \SI{+2.79(2)e+03}{\meter\per\second}$ und
$c\ua{D} = \SI{+2.74(3)e+03}{\meter\per\second}$ weichen nur geringfügig von dem Literaturwert
$c\ua{lit} = \SI{2.73e3}{\meter\per\second}$ ab (prozentuale Abweichung des Mittelwertes: $\approx \SI{1}{\percent}$).
Der mit dem Durchschallungsverfahren bestimmte Wert $c\ua{D}$
enthält den Literaturwert im Vertrauensbreich. Als Maß für den systematischen Fehler der Längenmessungen dienen
die Absizzenabschnitte der durchgeführten Regressionsrechnungen ($b\ua{E} = \SI{-1.6(5)}{\milli\meter}$ \&
$b\ua{B} = \SI{-3.7(9)}{\milli\meter}$). Diese liegen in einer Größenordnung die im Rahmen der Präzesion
der durchgeführten Messungen als plausibel erscheinen. \\
Das Cepstrum \ref{fig: cepstrum} zeigt deutliche Störungen auf. Das Ablesen der Peaks ist daher nur ungenau möglich.
Dennoch liefert die Methode relativ genaue Ergebnisse für Abmessungen der Acrylplatten [siehe \ref{eq: platten_cepstrum} und \ref{eq: platten_messschieber}].
Eine quantitave Untersuchung des Spektrums \ref{fig: spectrum} ist nicht möglich. Der Verlauf zeigt mehrere
Maxima neben der Frequenz des Eingangssignal (etwa $\SI{2}{\mega\hertz}$) auf. Diese entstehen durch die Überlagerung der
reflektierten Schallwellen. \\
Aufgrund der relativ präzisen Längenmessung der voran gegangenen Acrylkörper ist die Vermessung des Augenmodells als
signifikant einzustufen. In der Literatur \cite{sehen} findet man für die Länge des reduzierten Auges $l\ua{lit} = \SI{24}{\milli\meter}$. Unter
Berücksichtigung des Maßstabes 1:3 des verwendet Modells ergibt sich experimentell der Wert $l\ua{exp} = \SI{18.4}{\milli\meter}$. Das Ergebnis
weicht also in plausibler Weise vom realistischen Wert ab. \\
Allgemein bietet die Ultraschalltechnik auch im hier angewendeten Rahmen eine gute Möglichkeit zur experimentellen Bestimmung
von räumlichen Abständen.
