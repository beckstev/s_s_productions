\setcounter{page}{1}
\section*{Zielsetzung}

\subsection{Schall und Schallausbreitung}
Der Versuch US $1$ dient als Einführung in die Ultraschalltechnik.
Mit ihm sollen grundlegende physikalsische Eigenschaften der
Ultraschallechographie beobachtet und untersucht werden.
\section{Theorie}
Liegt Schall in einem Frequenzbereich von $\SI{20}{\kilo\hertz}$ bis $\SI{1}{\giga\hertz}$,
wird als \emph{Ultraschall} bezeichnet. Er ist vom Menschen nicht wahrnehmbar.
Schall breitet sich als longitudinale Druckwelle $p(x,t)$ der Form
\begin{equation*}
  p(x,t)=p_0+v_0 Z \cos(\omega t - kx) \qquad Z=c\rho
\end{equation*}
aus. Die in der Gleichung auftretende Größe $Z$ bezeichnet die \emph{akustische Impendanz}.
Die Impendanz wird durch die Dichte $\rho$ des durchquerten Materials und der
materialabhängige Schallgeschwindigkeit festgelegt.
In Flüssigkeiten ist die Schallgeschwindigkeit abhängig von der
Kompressibilität $\kappa$ und der Flüssigkeitsdichte $\rho$
\begin{equation*}
  c\ua{Fl}=\sqrt{\frac{1}{\kappa\rho}}.
\end{equation*}
Statt der Kompressibilität hängt die Schallgeschwindigkeit in einem Festkörper von
dem Elastizitätsmodul $E$ ab:
\begin{equation*}
  c\ua{Fe}=\sqrt{\frac{E}{\rho}}.
\end{equation*}
Zusätzlich wird das Untersuchen von Schall im Festkörpern durch Schubspannungen
erschwert. Die Schubspannungen sorgen dafür, dass sich die Schallwelle nicht nur
Longlitudianle (wie zum Beispiel in Flüssigkeiten), sondern auch als
Transversalwellen ausbreiten.
Dazu kommt das die Schallgeschwindigeit in Festkörper eine Richtungsabhängigkeit
besitzt.
Ähnlich wie bei elektromagnetischen Wellen, können bei Schallwellen auch
Phänomene wie zum Beispiel Reflexion oder Brechung auftreten.

Wie bei mechanischen Wellen, gibt es bei Schallwellen dissipative Effekte.
Hierdurch verliert die Welle ein Teil der Energie bei Schallausbreitung:
\begin{equation}
  \label{eq:abfall}
  I(x)=I_0\map{e}^{-\alpha x}.
\end{equation}
Die Größe $\alpha$ ist der Absorptionskoeffizient der Schallamplitude und $I_0$ ist die
Intensität.
Besonders bei Luft ist $\alpha$ groß, deshalb wird immer ein Kontaktpotential
zwischen Schallgeber und zu untersuchendem Material verwendet.

Beim Auftrefen einer Schallwelle auf eine Grenzfläche, wird nur ein Teil der
Welle transmitiert. Der andere Teil wird reflektiert.
Ein Maß für die Reflexion bietet der Reflexionskoeffizient $R$.
Dieser gibt das Intensitätverhältnis zwischen einfallender und reflektierten Welle an.
Er kann mittels der in den anliegenden Materialien vorliegdenden Impendanzen $Z_1$ und $Z_2$
\begin{equation*}
  R=\left(\frac{Z_1-Z_2}{Z_1+Z_2}\right)^2
\end{equation*}
berechnet werden.
Der Transmittierte Anteil $T$ lässt sich mit $T=1-R$ ermitteln.

\subsection{Erzeugung von Ultraschall}

Eine Möglichkeit Ultraschallwellen zu erzeugen bietet der \emph{piezo-elektrischen Efffekt}.
