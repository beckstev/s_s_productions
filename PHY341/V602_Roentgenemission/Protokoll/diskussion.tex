\section{Diskussion}
Im Folgenden sollen die Messergebnisse in Bezug auf die Messgenauigkeit des
Versuchsaufbaus diskutiert werden.

Bei dem Vergleich der Literaturwerte\cite{cu} für die $K_\alpha$ und $K_\beta$ Linie
von Kupfer
\begin{align*}
    K_\alpha&=\SI{8.2}{\kilo\eV} & K_{\alpha,\map{lit}}&=\SI{8.1}{\kilo\eV}  \quad \Delta\approx 1 \% \\
    K_\beta&=\SI{9.1}{\kilo\eV} & K_{\beta,\map{lit}}&=\SI{8.92}{\kilo\eV} \quad \Delta\approx 2 \%
\end{align*}
zeigt sich, dass der Versuchsaufbau genaue Ergebnisse bei der Untersuchung von Emissionsspektren %der Untersuchung
liefert.

Werden die Ergebnisse für $E\ua{K}$ mit der Literatur \cite{k_kante} verglichen (vgl. Tab. \ref{tab: result_comp}),
so wird deutlich, dass der Versuchsaufbau eine gute Möglichkeit bietet, um die Energie $E\ua{K}$ verschiedener %nomen
Elemente zu untersuchen.
\begin{table}
  \centering
  \caption{Vergleich der Messergebnisse von $E\ua{K}$ mit der Literatur \cite{k_kante}.}
  \label{tab: result_comp}
  \begin{tabular}{S S S S}
    \toprule
    {Element}& {$E\ua{k}$ in $\si{\kilo\eV}$} & {$E\ua{k,lit}$ in $\si{\kilo\eV}$} & {Abweichung in $\%$}  \\
    \midrule
    $\ce{Zn}$&  9.85  & 9.65 & 2 \\
    $\ce{Ge}$&  11.4  & 10.4 & 10 \\
    $\ce{Zr}$&  18.7  & 18.0 & 4 \\
    $\ce{Br}$&  13.9  & 13.5 & 4 \\
    $\ce{Sr}$&  16.7  & 16.1 & 4  \\
    \bottomrule
  \end{tabular}
\end{table}

  Zur Analyse der $L_2$ und $L_3$ Kante von Elementen eignet sich der Versuchsaufbau
  ebenso, denn beim Vergleich der Messergebnisse mit der Literatur \cite{l_kante} %kein neuer satz
  offenbaren sich nur kleine Abweichung.
\begin{align*}
  L_2&=\SI{14.3}{\kilo\eV} & L_{2,\map{lit}}&=\SI{13.7}{\kilo\eV}  \quad \Delta\approx 4 \% \\
  L_3&=\SI{12.2}{\kilo\eV} & L_{3,\map{lit}}&=\SI{11.9}{\kilo\eV} \quad \Delta\approx 3 \%
\end{align*}

Die aus dem Versuch resultierende Rydbergenergie $R_\infty=\SI{14.11\pm0.15}{\eV}$ weicht vom Literaturwert \cite{anleitung602} %vom
$R_{\infty,\map{lits}}=\SI{13.6}{\eV}$ um $\approx 4 \%$ ab. Somit nähert sich der
Versuchsaufbau dem Theoriewert gut an.

Ein systematischer Fehler bei der Apparatur ist wahrscheinlich, denn auch bei der %Apperatur
Untersuchung der Bragg-Bedingung gibt es eine Abweichung von $\approx 2\%$.
\begin{equation*}
  \theta\ua{bragg}=\SI{14.25}{\degree}  \qquad \theta\ua{bragg,lit}=\SI{14}{\degree}.
\end{equation*}
Scheinbar ist die Feinjustage der Versuchsapparatur nicht ganz genau und sorgt so bei den Messergebnissen
für eine konstante Abweichung zu Literatur bis zu $10\%$.

Die in der Auswertung bestimmten Halbwertsbreiten liefern Auskunft über das
Auflösungsvermögen der Apparatur. %Apperatur
Das Auflösungsvermögen gibt an, wie klein der Abstand zweier Messpunkte maximal sein kann damit sie von der Messeinrichtung
getrennt regrestiert werden können.
Eine Analogon gibt es dazu in der Optik, dort besagt das \emph{Rayleigh-Kriterium},
wie weit zwei Beugungsmaxima voneinander entfernt liegen müssen, damit sie exakt aufgelöst werden.
Nach diesem werden zwei Punkte erst dann getrennt aufgelöst, wenn sich das erste Minumum vom ersten Bildpunkt
sich in einem Maxima des zweiten Bildpunktes befindet.
Wie in der Abbildung \ref{fig: halbwert} zu erkennen, ist der Abstand der beiden Peaks
größer als $\delta_{\theta,\alpha}$ oder $\delta_{\theta,\beta}$. Dies ist ein Indikator
für ein hohes Auflösungsvermögen. Eine präzisere Aussage könnte erzielt werden,
wenn um die Hochpunkte herum mehr Messwerte aufgenommen werden. An diese
könnte dann eine Gauß-Funktion gefittet werden, die eine bessere Aussage über die
Breite $\delta_\theta$ bzw $\delta\ua{E}$ liefert.

Abschließend ist zu sagen, dass der Versuchsaufbau signifikante Messergebnisse
bei der Untersuchung von Emissions- und Absorptionsspektren liefert und ein
gutes Auflösungsvermögen besitzt.
