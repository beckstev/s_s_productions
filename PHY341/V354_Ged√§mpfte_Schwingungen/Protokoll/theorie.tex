\setcounter{page}{1}
%\section*{Zielsetzung}
%Mit dem Versuch soll das Schwingverhalten eines elektrischen Schwingkreises,
%bei Dämpfung oder äußerer Anregung, untersucht werden.

\section{Theorie}

Für einen elektrischen Schwingkreis benötigt man grundsätzlich
Kondensatoren und Spulen.
Bei einer einmalig äußeren Anregung z.\,B. durch eine Peakimpuls
kann, dann eine Schwingung gemessen werden.
Wird zusätzlich noch ein Widerstand mit angeschlossen, ergibt sich ein
gedämpfte elektrischer Schwingkreis.
Wird der Schwingkreis kontinuierlich von außen angeregt, nennt man
ihn erzwungenen Schwingkreis.

\subsection{Gedämpfter Schwingkreis}

Zunächst sollen mithilfe der zweiten
Kirchhoffschen Regel die Spannungen in einem Schwingreis (vgl. Abb \ref{})
bestimmt werden:

\begin{equation}
  \label{eq:kirch}
U\ua{R}(t)+U\ua{C}+U\ua{L}(t).
\end{equation}

Durch hinzu nahme der bekannten Zusammenhänge

\begin{equation*}
U\ua{R}(t)=RI(t) \quad U\ua{C}(t)=\frac{Q(t)}{C} \quad U\ua{L}(t)=L\frac{\map{d}}{\map{d}t}I
\end{equation*}

und der zeitlichen Ableitung von \eqref{eq:kirch}, auf eine Schwingungsdifferentialgleichung

\begin{equation}
  \label{eq:differntialgleichung_1}
\frac{\map{d}^2}{\map{d}t^2}I+\frac{R}{L}\frac{\map{d}}{\map{d}t}I+\frac{1}{LC}I=0
\end{equation}

geschlossen werden.
Als Lösungsansatz wählt man:

\begin{equation*}
  I(t)=A\,\map{e}^{-\map{i}\hat{\omega} t} \quad A,\omega\in\mathbb{C}.
\end{equation*}
Dabei ist $\omega$ die Frequenz mit der das System schwingen kann.
Durch einsetzen von $I(t)$ in \eqref{eq:differntialgleichung_1} ergibt sich
für diese:

\begin{equation}
  \label{eq:omega_gedaempft}
  \hat{\omega}_{1,2}=\map{i}\frac{R}{2L}\pm\left(\,\underbrace{\frac{1}{LC}}_{:=\omega_0}-\underbrace{\frac{R^2}{4L^2}}_{:=\beta}\,\right)^{\frac{1}{2}}.
\end{equation}

 Die Frequenz $\omega$ kann folglich drei mögliche Werte annehmen.
 Je nachdem, was für Werte $\omega_0$ und $\beta$ besitzen.
\\

Bei dem Fall $\omega_0<\beta$ handelt es sich um den sogenannten \emph{Kriechfall}.
 Hierbei wird die Wurzel in \eqref{eq:omega_gedaempft} negativ. Als Lösung für
 $I(t)$ erhält man

\begin{equation*}
     I(t)=A\,\map{e}^{\left(-\frac{R}{2L}\mp\sqrt{\frac{R^2}{4L^2}-\frac{1}{LC}}\right)\,t}.
\end{equation*}

Die Stromstärke besitzt somit keine periodische Eigenschaft.
\\

Ist $\omega_0>\beta$ so untersucht man den \emph{Schwingfall}.
Bei diesem bleibt die Wurzel in \eqref{eq:omega_gedaempft} positiv. Als
Lösung für $I(t)$ folgt

\begin{equation*}
  I(t)=A\,\map{e}^{\left(-\frac{R}{2L}\pm\sqrt{{\frac{1}{LC}}-\frac{R^2}{4L^2}}\right)t}.
\end{equation*}

Abschließend soll der Fall $\omega_0=\beta$ besprochen werden.
Dieser wird auch als \emph{aperiodischer Grenzfall} bezeichnet.
Die Stromstärke hat hier keine Nullstellen und fällt schneller als
beim Kriechfall.
Für die Amplitude der Stromstärke ergibt sich

\begin{equation}
  \label{eq:ap_grenz}
  I(t)=A\,\map{e}^{-\frac{R}{2L}\,t}=A\,\map{e}^{-\frac{1}{\sqrt{LC}}\,t}.
\end{equation}

\subsection{Erzwungene Schwingung}

Durch das hinzuschalten einer externen Anregung $U(t)$ (vgl. Abb. \ref{})
erhält die Differntialgleichung \eqref{eq:differntialgleichung_1} eine
Inhomogenität:

\begin{align}
  \frac{\map{d}^2}{\map{d}t^2}I+\frac{R}{L}\frac{\map{d}}{\map{d}t}I+\frac{1}{LC}I=U(t)&=U_0\map{e}^{\map{i}\omega\, t}\notag \\
  LC\frac{\map{d}^2}{\map{d}t^2}U\ua{C}+\frac{R}{C}\frac{\map{d}}{\map{d}t}U\ua{C}+U\ua{C}&=U_0\map{e}^{\map{i}\omega\, t}. \label{eq:dgl_2}
\end{align}

Die Differntialgleichung \eqref{eq:dgl_2} wird gelöst durch

\begin{equation}
  \label{eq:uc_erzwung}
  U\ua{C}(\omega)=U_0\left(\left(1-LC\omega\right)1^2+\omega^2R^2C^2\right)^{-\frac{1}{2}}.
\end{equation}

Die Spannung am Kondensator ist also abhängig von der Erregerfrequenz $\omega$.
Die Untersuchung des Randverhaltens von \eqref{eq:uc_erzwung} liefert

\begin{equation*}
  \lim_{\omega \to \infty} U\ua{C}(t)=0 \qquad \lim_{\omega \to 0} U\ua{C}(t)=U_0.
\end{equation*}

Zusätzlich gibt es die sogenannte \emph{Resoanzfrequenz}.
Bei dieser Frequenz erreicht $U_C$ ihren Maximalwert. Dies ist genau dann der Fall, wenn
$\omega$ auf

\begin{equation*}
  \label{eq:resonanzfrequenz}
  \omega\ua{res}=\left(\frac{1}{LC}-\frac{R^2}{2L^2}\right)^{\frac{1}{2}}
\end{equation*}

eingestellt ist.
Die Spannungsamplitude hat folglich den Betrag:

\begin{equation}
  \label{eq:spann_ampli}
  U\ua{C,\, max}=\frac{1}{R}\sqrt{\frac{L}{C}}U_0=qU_0.
\end{equation}
Der Faktor $q$ wird auch als \emph{Güte des Schwingkreises} bezeichnet.
Ist $U\ua{C}=U\ua{C,\, max}$ spicht man von \emph{Resonanz}.
Als Indikator für die Schärfe der Resonanz dienen die beiden
Frequenzen $\omega_+$ und $\omega_-$. An diesen Frequenzen ist
$U\ua{C}$ auf

\begin{equation*}
   U\ua{C}=2^{-\frac{1}{2}} U\ua{C,\, max}
\end{equation*}

abgefallen.
Mit der Gleichung \eqref{eq:spann_ampli} und der Annahme $\frac{R^2}{L^2}\ll \frac{1}{LC}$
folgt

\begin{equation}
  \label{eq:omega_+_-}
  \omega_+-\omega_-\approx \frac{R}{L}.
\end{equation}

Die Folgerung steht durch,

\begin{equation*}
  q=\frac{1}{\sqrt{LC}\left(\omega_+-\omega_-\right)}
\end{equation*}

im direktem Zusammenhang mit der Güte.

Die Phasen zwischen Kondensator- und Errergerspannung kann maximal
einen Unterschied von $\pi$ besitzen. Dies folgt mit der, aus der
Herleitung von \eqref{eq:uc_erzwung} bekannten, Gleichung:

\begin{equation}
  \label{eq:phase}
  \varphi(\omega)=\arctan\left(\frac{-\omega RC}{1-LC\omega^2}\right).
\end{equation}
