\section{Diskussion}
Im Folgenden sollen die Ergebnisse der Messungen interpretiert und in Beziehung zur Präzision des verwendeten Aufbaus gestellt werden. \\
Am Graphen \ref{fig: amplitude}, der die Einhüllende der gedämpften Schwinngung darstellt, kann der in der Theorie beschriebene Verlauf deutlich erkannt werden.
Der aus dieser Kurve bestimmte effektive Widerstand weicht in plausibler Weise vom verbauten Widerstand ab.\\
Die experimentell bestimmten Werte aus der Messung der Resonanzamplitude, etwa die Resonanzfrequenz [prozentuale Abweichung \SI{}{\%}],
stimmen in guter Näherung mit den theoretischen Werten überein. Dies ist auf die hohe Messpräzision des digitalen Oszilloskops zurückzuführen.
Die berechneten Abweichungen lassen sich durch abweichennde Innenwiderstände und sonstige systematische Fehler erklären. Ähnliche Resultate
wurden mittels der Beobachtung der Frequenzabhängigkeit der Phasendifferenz erzielt. Insbesondere die Resonanzfrequenz [Abweichung \SI{−0,26}{\%}] konnte
sehr genau bestimmt werden. 
