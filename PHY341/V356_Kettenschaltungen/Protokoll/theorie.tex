d\setcounter{page}{1}
\section*{Zielsetzung}
Die Untersuchng einer Kettenschaltung von LC-Gliedern lässt sich mit der Analogie  
zu einem eindimensionalen Festköper motivieren.
Der Versuch ermöglicht somit Einblicke in das Schwingverhalten eines 
Festköpers.

\section{Theorie}

\subsection{Bestimmung der Schwingungsgleichungen für eine $LC$-Kette}
Die Schwingungsgleichung werden mit der Kirchhoffschen Regel aufgestellt.
Zunäschst wird mittels der ersten Kirchhoffschen Regel die Stöme für ein Kettenglied $n$ 
bestimmt (vgl. Abbildung \ref{}):

\begin{equation}
\label{eq:kircheins_lc}
I_n-I_{n+1}-I_{n,\map{quer}}=0
\end{equation}

Mit diesem Ansatz erhält man dann die Gleichung:

\begin{equation}
\label{eq:gleichung_lc}
-\omega^2CU\ua{n}+\frac{1}{L}\left(-U\ua{n-1}+2U\ua{n}-U\ua{n+1}\right)
\end{equation}
Gelöst wird die Gleichung mit
\begin{equation}
\label{eq:loesung_gleichung_lc}
U_n(t)=U_0 \map{e}^{j\omega t} \map{e}^{-jn\theta t}.
\end{equation}
Es sei hierbei $\omega$ die Kreisfrequenz und $\theta$ die Phasenverschiebung 
die durch ein Kettenglied verursacht wird.
Für die Kreisfrequenz ergibt sich die Dispersionsrelation:

\begin{equation}
\label{eq:kreisfrequenz_lc_glied}
\omega^2=\frac{2}{LC}\left(1-\cos(\theta)\right)
\end{equation}

Da \eqref{eq:kreisfrequenz_lc_glied} nur für endlich viele Kettenglieder definert ist, kann
die Formel maximal im Intervall:

\begin{equation}
\label{eq:menge_omega_lc_glied}
\omega\in\left[\,0,\frac{2}{\sqrt{LC}}\,\right)
\end{equation}

\subsection{Bestimmung der Schwingungsgleichungen für eine $LC_1C_2$-Kette}
Im Gegensatz zum vorherigen Kapitel, werden bei diesem Aufbau zwei unterschieldiche 
Kondensatoren $C_1$ und $C_2$ verwendet. Dabei werden die Kondensatoren immer
immmer alternierend geschaltet (vgl. Abbidldung \ref{}).
Aufgrund dessen muss \eqref{eq:gleichung_lc} zu dem Gleichungssystem

\begin{align}
\label{eq:lc1c2_gleichungsy}
\begin{aligned}
-\omega^2C_1U_{2n+1}+\frac{1}{L}\left(-U_{2n}+2U_{2n+1}-U_{2n+2}\right)\\
-\omega^2C_2U_{2n}+\frac{1}{L}\left(-U_{2n-1}+2U_{2n}-U_{2n+2}\right)
\end{aligned}
\end{align}
angepasst werden.
Folglich ergeben sich die gleichen Lösungen wie bei \eqref{eq:gleichung_lc}
lediglich einmal mit $2n$ und einmal mit $2n+1$.
Durch anweden der Cramerschen Regel kann auf die Kreisfrequenz bestimmt 
zu:
\begin{equation*}
\omega^4-\omega^2\frac{2}{L}\left(\frac{1}{C_1}+\frac{1}{C_2}\right)+\frac{4}{L^2C_1C_2}\left(1-\cos^2(\theta)\right)=0
\end{equation*}
Diese Gleichung kann dan nach $\omega^2$ aufgelöst werden.
Das Resultat lautet
\begin{equation}
\label{eq:omgea_ceins_czwei}
\omega_{1,2}^{2}=\frac{1}{L}\left(\frac{1}{C_1}+\frac{1}{C_2}\right)\pm\frac{1}{L}\sqrt{\left(\frac{1}{C_1}+\frac{1}{C_2}\right)-\frac{4\sin^2(\theta)}{C_1C_2}}.
\end{equation}

Es gibt also zwei verschiedene Kreisfrequenzen $\omega_1$ und $\omega_2$.
Trägt man beide Frequenz in einem Diagramm auf (in Abhängigkeit von $\theta$) so ergeben sich 
zwei Kurven (vgl. Abbildung \ref{}). 
Zunächst soll die Lösung $\omega_2$ besprochen werden.
Wird diese bei $\theta=0$ betrachtet ist ebenso $\omega=0$.
Mit der Näherung $\sin(\theta)\approx\theta$ ergibt dann

\begin{equation}
\label{eq:omgea_2_lceins_czwei}
\omega_2\approx\sqrt{\frac{2}{L\left(C_1+C_2\right)}}\theta.
\end{equation}
Die Nullstelle von \eqref{eq:omgea_2_lceins_czwei} liegt im Nullpunkt $\theta=0$.
Für $\theta>0$ wächst die kurve monoton an und erreicht bei $\omega_2(\frac{\pi}{2})=\sqrt{\frac{2}{LC_1}}$
ihr Maximum.
Für $\omega_1$ beläuft sich die Lösung bei $\theta=0$ auf
\begin{equation*}
\omega_1(0)\approx\sqrt{\frac{2(C_1+C_2)}{LC_1C_2}}
\end{equation*}
Auch $\omega_1$ besitzt ein Maximum mit

\begin{equation}
\label{eq:max_omega_1_ceins_czwei}
\omega_1(\frac{\pi}{2})=\sqrt{\frac{2}{LC_2}}
\end{equation}
Da die beiden Maxima unterschieldich Groß sind, gibt es einen Frequenzbereich
in dem die Kettenschaltung keinen Strom durchlässt.
\subsection{Ausbreitungsgeschwindigkeit von Wellen in eienr Kettenschaltung}
Bei einer allgeimeinen Welle in der Form \eqref{eq:loesung_gleichung_lc}
ergibt sich für Ausbreitungsgeschwindigkeit gleicher Phasen
\begin{equation*}
v\ua{ph}=\frac{\omega}{\theta}.
\end{equation*}
Es wird $v\ua{ph}$ als Phasengeschwindigkeit bezeichnet.
Die Zeitabhängigkeit der Phasengeschwindigkeit ist durch $\omega=\map{const}$ 
für alle Zeiten $t$ festgelegt. Wellen die sich mit der Phasengeschwindigkeit
ausbreiten sind, aufgrund ihrer Unbeschränktheit in Raum und Zeit, nicht zu
Informationsübermittlung genutzt werden. Dazu benötigt man Wellenpakete die sich 
mit der Gruppengeschwindigkeit ausbreiten. Auf diese wird hier nicht eingegangen.
Speziell für den versuch ergibt sich als Phasengeschwindigkeit:

\begin{equation}
\label{eq:phasen_esc}
v\ua{ph}=\frac{\omega}{\theta}=\frac{\omega}{\arccos\!\left(1-\frac{1}{2}\omega^2LC\right)}
\end{equation}
Dabei geht die Phasengescchwindigkeit im Grenzfall

\begin{equation*}
\lim_{\omega\to 0}v\ua{ph}=\frac{1}{\sqrt{LC}}
\end{equation*}

\subsection{Der Widerstand einer unendlichen langen Kette}
Da es sich bei der Kettenschaltung um eine reale Schaltung handelt
besitzt diese einen elektrischen Widerstand $R$.
Der Widerstand kann mittels der Eingangspannung $U_0$ und dem Strom $I_0$ bestimmit werden.
Dazu wird zunächst mit der Kirchhoffschen Knotregel gearbeitet (vgl. Abbildung \ref{}).
Mit dieser erhält man
\begin{equation*}
I_0 - \map{i}\omega\frac{C}{2}U_0+\frac{U_1-U_0}{\map{i}\omega L}=0
\end{equation*}
Es sei $\map{i}$ das komplexe i.
Mit dem ohmschen Gesetz erhält man letzendlich

\begin{equation}
R(\omega)=Z(\omega)=\sqrt{\frac{L}{C}}\frac{1}{\sqrt{1-\frac{1}{4}\omega^2LC}}
\end{equation}
Es ist üblich $R$ bzw. $Z$ als frequenzabhängigen Wellenwiderstand zu bezeichnen.
Für den Fall einer unendlich langen Kette wäre der Wiederstand rein reell, dass bedeutet 
das es zu keinem Zeiptunkt weder zu einer Phasenverschiebung kommmt noch zu einer Reflexion entsteht.
Durch einen Abschlusswiederstand am Ender der Kette ist dieser Zustand auch bei einer endlichen Kette 
realisierbar. Der Abschlusswiederstand muss dann lediglich den Betrag $Z$ besitzen.
Zwar ist $Z$ frequenzabhängig, doch bei Frequenzen weit unter der Grenzfrequenz, ist 
$Z\approx \sqrt{\frac{L}{C}}$.

\subsection{Eigenschaften einer endlichen Kette}
 
