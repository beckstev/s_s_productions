d\setcounter{page}{1}
\section*{Zielsetzung}
Die Untersuchng einer Kettenschaltung von LC-Gliedern lässt sich mit der Analogie  
zu einem eindimensionalen Festköper motivieren.
Der Versuch ermöglicht somit Einblicke in das Schwingverhalten eines 
Festköpers.

\section{Theorie}

\subsection{Bestimmung der Schwingungsgleichungen für eine $LC$-Kette}
Die Schwingungsgleichung werden mit der Kirchhoffschen Regel aufgestellt.
Zunäschst wird mittels der ersten Kirchhoffschen Regel die Stöme für ein Kettenglied $n$ 
bestimmt (vgl. Abbildung \ref{}):

\begin{equation}
\label{eq:kircheins_lc}
I_n-I_{n+1}-I_{n,\map{quer}}=0
\end{equation}

Mit diesem Ansatz erhält man dann die Gleichung:

\begin{equation}
\label{eq:gleichung_lc}
-\omega^2CU\ua{n}+\frac{1}{L}\left(-U\ua{n-1}+2U\ua{n}-U\ua{n+1}\right)
\end{equation}
Gelöst wird die Gleichung mit
\begin{equation*}
U_n(t)=U_0\map{e}^{j\omgea t}\map{e}^{-jn\theta t}.
\end{equation*}
Es sei hierbei $\omega$ die Kreisfrequenz und $\theta$ die Phasenverschiebung 
die durch ein Kettenglied verursacht wird.
Für die Kreisfrequenz ergibt sich die Dispersionsrelation:

\begin{equation}
\label{eq:kreisfrequenz_lc_glied}
\omega^2=\frac{2}{LC}\left(1-\cos(\theta)\right)
\end{equation}

Da \eqref{eq:kreisfrequenz_lc_glied} nur für endlich viele Kettenglieder definert ist, kann
die Formel maximal im Intervall:

\begin{equation}
\label{eq:menge_omega_lc_glied}
\omega\in\left[\,0,\frac{2}{\sqrt{LC}}\,\right)
\end{equation}

\subsection{Bestimmung der Schwingungsgleichungen für eine $LC_1C_2$-Kette}
Im Gegensatz zum vorherigen Kapitel, werden bei diesem Aufbau zwei unterschieldiche 
Kondensatoren $C_1$ und $C_2$ verwendet. Dabei werden die Kondensatoren immer
immmer alternierend geschaltet (vgl. Abbidldung \ref{}).
Aufgrund dessen muss \eqref{eq:gleichung_lc} zu dem Gleichungssystem

\begin{align}
\label{eq:lc1c2_gleichungsy}
\begin{aligned}
-\omega^2C_1U_{2n+1}+\frac{1}{L}\left(-U_{2n}+2U_{2n+1}-U_{2n+2}\right)\\
-\omega^2C_2U_{2n}+\frac{1}{L}\left(-U_{2n-1}+2U_{2n}-U_{2n+2}\right)
\end{aligned}
\end{align}
angepasst werden.
Folglich ergeben sich die gleichen Lösungen wie bei \eqref{eq:gleichung_lc}
lediglich einmal mit $2n$ und einmal mit $2n+1$.
Durch anweden der Cramerschen Regel kann auf die Kreisfrequenz bestimmt 
zu:
\begin{equation*}
\omega^4-\omega^2\frac{2}{L}\left(\frac{1}{C_1}+\frac{1}{C_2}\right)+\frac{4}{L^2C_1C_2}\left(1-\cos^2(\theta)\right)=0
\end{equation*}
Diese Gleichung kann dan nach $\omega^2$ aufgelöst werden.
Das Resultat lautet
\begin{equation}
\label{eq:omgea_ceins_czwei}
\omega_{1,2}^{2}=\frac{1}{L}\left(\frac{1}{C_1}+\frac{1}{C_2}\right)\pm\frac{1}{L}\sqrt{\left(\frac{1}{C_1}+\fac{1}{C_2}\right)-\frac{4\sin^2(\theta)}{C_1C_2}}.
\end{equation}

Es gibt also zwei verschiedene Kreisfrequenzen $\omega_1$ und $\omega_2$.
Trägt man beide Frequenz in einem Diagramm auf (in Abhängigkeit von $\theta$) so ergeben sich 
zwei Kurven (vgl. Abbildung \ref{}). 


