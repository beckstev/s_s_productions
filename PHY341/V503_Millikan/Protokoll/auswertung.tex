\section{Auswertung}
Die gemessenen Thermowiderstände $R$, Fallzeiten $t$ und Spannungen $U$ sind in Tabelle \ref{} einzusehen.
Die aus der Anleitung entnommenen Werte Paare $(R, T)$ (siehe Tabelle x) wurden benutzt um den Zusammenhang
zwischen dem Thermowiderstand und und der Temperatur mittels Polynominterpolation zu approximieren. Die graphische
Darstellung des gefundenen Polynoms befindet sich in Abbildung \ref{}. Die mit dem Interpolationspolynom berechneten
Temperaturen sind ebenfalls in Tabelle \ref{} eingefügt. Der Zusammenhang zwischen der Viskosität von Luft und
der Temperatur wurde der Anleitung \cite{} entsprechend als linear angenommen. Mit den Wertepaaren
\begin{align}
  \eta_1 &= \SI{1.85e-5}{\newton\second\meter^{-2}}, \quad T_1 = \SI{16}{\celsius} \\
  \eta_2 &= \SI{1.88e-5}{\newton\second\meter^{-2}}, \quad T_2 = \SI{32}{\celsius},
\end{align}
die dem Graphen aus der Anleitung entnommen wurden, wird eine Geradengleichung bestimmt, die den bestimmten Temperaturen
$T$ in $\si{\celsius}$ die Viskositäten $\eta$ in $\si{\newton\second\meter^{-2}}$ zuordnet.
\begin{equation}
  \eta(T) = \SI{0.001875}{\newton\second\meter^{-2} \celsius^{-1} } * T  + \SI{1.82}{\newton\second\meter^{-2}}.
\end{equation}
Die somit berechneten Viskositäten sind in Tabelle \ref{} aufgetragen.
Die Fallgeschwindigkeiten der Tröpfchen berechnen sich gemäß \eqref{} und der Falldistanz $s = \SI{0.5}{\milli\meter}$.
Mit Gleichung \eqref{} ergeben sich somit schließlich die Tröpfchenradien und die die Ladungen $q$. Hierbei wurde
direkt die Korrektur \eqref{} angewandt. Alle Ergebnisse sind in Tabelle \ref{} eingetragen. \\
Um nun



\begin{table} 
\centering 
\caption{Gemessene und brechnete Größen für einzelne beobachtete Tropfen. Thermowiderstand $R$, Temperatur $T$, Luftviskosität $\eta$, Tröpfchenradius $r$, Fallzeit $t$, Fallgeschwindigkeit $v_0$, Schwebespannung $U$ und korrigierte Ladung $q$.} 
\label{tab: data} 
\begin{tabular}{S S S S S S S S } 
\toprule  
{$R/\si{\mega\ohm}$} & {$T/\si{\celsius}$} & {$\eta/10^{-5}\si{\newton\second\meter^{-2}}$} & {$r/\si{\milli\meter}$} & {$t/\si{\second}$} &
 {$v_0/\si{\centi\meter\second^{-1}}$} & {$U/\si{\volt}$}  & {$q/10^{-19}\si{\coulomb}$}  \\ 
\midrule  
 1.89  & 27.26  & 0.00  & 0.00  & 10.525  & 0.005  & 79.60  & 9.19\\ 
1.88  & 27.48  & 0.00  & 0.00  & 11.498  & 0.004  & 68.80  & 9.25\\ 
1.85  & 28.17  & 0.00  & 0.00  & 10.807  & 0.005  & 61.50  & 11.43\\ 
1.83  & 28.63  & 0.00  & 0.00  & 9.504  & 0.005  & 144.00  & 5.98\\ 
1.81  & 29.11  & 0.00  & 0.00  & 9.251  & 0.005  & 143.60  & 6.26\\ 
1.81  & 29.11  & 0.00  & 0.00  & 12.489  & 0.004  & 97.90  & 5.72\\ 
1.81  & 29.11  & 0.00  & 0.00  & 5.147  & 0.010  & 86.60  & 26.02\\ 
1.80  & 29.36  & 0.00  & 0.00  & 28.793  & 0.002  & 22.30  & 6.57\\ 
1.79  & 29.60  & 0.00  & 0.00  & 16.228  & 0.003  & 130.40  & 2.83\\ 
1.79  & 29.60  & 0.00  & 0.00  & 15.588  & 0.003  & 70.20  & 5.61\\ 
1.78  & 29.85  & 0.00  & 0.00  & 4.940  & 0.010  & 184.40  & 13.04\\ 
1.77  & 30.11  & 0.00  & 0.00  & 23.773  & 0.002  & 26.70  & 7.50\\ 
1.77  & 30.11  & 0.00  & 0.00  & 16.099  & 0.003  & 280.00  & 1.34\\ 
1.76  & 30.36  & 0.00  & 0.00  & 10.014  & 0.005  & 98.60  & 8.07\\ 
1.76  & 30.36  & 0.00  & 0.00  & 21.332  & 0.002  & 207.00  & 1.15\\ 
1.76  & 30.36  & 0.00  & 0.00  & 10.561  & 0.005  & 79.70  & 9.18\\ 
1.75  & 30.62  & 0.00  & 0.00  & 13.247  & 0.004  & 53.80  & 9.50\\ 
1.75  & 30.62  & 0.00  & 0.00  & 13.929  & 0.004  & 104.40  & 4.52\\ 
1.74  & 30.89  & 0.00  & 0.00  & 7.828  & 0.006  & 197.00  & 5.95\\ 
1.74  & 30.89  & 0.00  & 0.00  & 13.804  & 0.004  & 301.00  & 1.59\\ 
1.74  & 30.89  & 0.00  & 0.00  & 15.460  & 0.003  & 35.00  & 11.42\\ 
1.74  & 30.89  & 0.00  & 0.00  & 9.605  & 0.005  & 83.40  & 10.19\\ 
1.73  & 31.16  & 0.00  & 0.00  & 16.007  & 0.003  & 60.60  & 6.24\\ 
1.73  & 31.16  & 0.00  & 0.00  & 14.304  & 0.003  & 97.10  & 4.66\\ 
1.73  & 31.16  & 0.00  & 0.00  & 13.946  & 0.004  & 107.90  & 4.37\\ 
\bottomrule 
\end{tabular} 
\end{table}
\begin{table}
\centering
\caption{Wertepaare zur Interpolation des Zusammenhangs zwischen $R$ und $T$.}
\label{tab: thermowiderstand}
\begin{tabular}{S S| S S } 
\toprule
{$R /\si{\mega\ohm}$} & {$T/\si{\celsius}$} & {$R /\si{\mega\ohm}$} & {$T/\si{\celsius}$}  \\
\midrule
 2.30  & 20  & 1.77  & 30\\
2.23  & 21  & 1.74  & 31\\
2.17  & 22  & 1.70  & 32\\
2.11  & 23  & 1.67  & 33\\
2.05  & 24  & 1.63  & 34\\
2.00  & 25  & 1.60  & 35\\
1.95  & 26  & 1.57  & 36\\
1.90  & 27  & 1.55  & 37\\
1.86  & 28  & 1.52  & 38\\
1.81  & 29  & 1.50  & 39\\
\bottomrule
\end{tabular}
\end{table}

