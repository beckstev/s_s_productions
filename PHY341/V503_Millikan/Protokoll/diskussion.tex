\section{Diskussion}
Im Folgenden soll die Signifikanz des gewonnenen Ergebnisses für die Elementarladung interpretiert werden. \\
Der verwendete Aufbau stellt eine gute Möglichkeit zur Anwendung der Schwebemethode zur Verfügung. Mit Hilfe des Mikroskops
kann präzise erkannt werden, wann sich das Tröpfchen im Kräftegleichgewicht befindet. Auch die Aussagekraft der Geschwindigkeitsmessung lässt
sich allgemein als annehmbar einstufen. Durch eine Abdeckung der Einsprühöffnung wird sichergestellt, dass die
Tröpfchen nicht durch Luftverwirbelungen beeinflusst werden. Darüber hinaus erweist es sich als zweckmäßig eher langsamere Tropfen
im Bereich von etwa $\SI{0.005}{\centi\meter \per \second}$ zu verwenden, da diese die Zeitmessung erleichtern. \\
Die bestimmte Elementarladung $\mathup{e}\ua{exp} = \SI{+1.7(1)e-19}{\coulomb}$ liegt sehr nah an dem Theoriewert
$\mathup{e}\ua{0} = \SI{1.6021766208e-19}{\coulomb}$. Die mit einiger Willkür getroffenen Annahmen zur Verwendung der Daten erweisen sich
somit als gut gewählt. Auch ohne diese in der Auswertung beschriebenen Einschränkungen wird mit der verwendeten Methode ein Wert gefunden
$\mathup{e}'\ua{exp} = \SI{+1.2(11)e-19}{\coulomb}$, der in der richtigen Größenordnung liegt und den Theoriewert in der Abweichung enthält. Jedoch ist in
diesem Fall der Fehler und damit die Streuung um die ELementarladung wesenlich höher (relativer Fehler ca. $\SI{92}{\percent}$) und das Ergebnis %EL=El
nicht signifikant. Hierin liegt die Motivation der durchgeführten Datenaussparungen. %Absatz
Der Versuch in der durchgeführten Form liefert insgesamt eine gute Möglichkeit eine grobe Abschätzung der ELementarladung zu bestimmen.

%In Tabelle 3 ist e nicht aufrecht
