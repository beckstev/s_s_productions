\setcounter{page}{1}
\section*{Zielsetzung}
Der Versuch $V503$ soll eine experimentelle Bestimung der Elementarladung $\map{e}_0$ ermöglichen. %Bestimmung; vielleicht noch Schwebemethode erwähnen
\section{Theorie}
Um die Elementarladung $\map{e}_0$ zu ermittelen, wird der nach Robert Andrews Millikan benannte %ermitteln
Milikan-Versuchsaufbau verwendet. Millikan verwendete dazu zerstäubte Öltröpfchen in einem %Millikan
Plattenkondensator. Beim Zerstäubevorgang erhalten die Öltröpfchen eine elektrische
Ladung, hierdurch stehen sie in Wechselwirkung mit dem elektrischen Feld $\vec{E}$ des
Kondensators. Ein Öltröpfchen trägt immer ein ganzzahliges Vielfaches der Elementarladung. %Satz an andere Stelle
Neben der elektrischen Feldkraft des Kondensators, wirkt auch die Gravitaionskraft %Gravitation
$\vec{F}\ua{g}=m\vec{\map{g}}$ auf ein Töpfchen. %Tröpchen
Auf Grund dessen bewegt sich das Tröpfchen zum Boden
des Kondensators. Beim Fallvorgang wirkt die Stoksche Reibungskraft %Stokesche
\begin{equation*}
  \vec{F}\ua{R}=-6\map{\pi}\eta\ua{L}\vec{v}
\end{equation*}
der Bewegung entgegen. Bei der Reibungskraft bezeichnet $r$ den Radius des
Tröpfchens und $\eta\ua{L}$ die Viskosität von Luft. Sobald ein Kräftegleichgewicht
zwischen der Reibungskraft $\vec{F}\ua{R}\propto \vec{v}$ und der Gravitationskraft
herrscht, bewegt sich das Teilchen mit der konstanten Geschwindigkeit $v_0$ weiter.
Beim Kräftegleichgewicht gilt folgender Zusammenhang:
\begin{equation}
  \label{eq:kraeftegleich}
  \frac{4\map{\pi}}{3}r^3\left(\rho\ua{oel}-\rho\ua{L}\right)\map{g}=6\map{\pi}\eta\ua{L}r v_0.
\end{equation}
Die linke Seite von \eqref{q:kraeftegleich} Berücksichtigt zusätzlich den Auftrieb des %berücksichtigt, Prinzip des Archimedes
Tröpfchen in der Luft. Mit der obigen Gleichung lässt sich ein Ausdruck für den Radius %Tröpfchens
des Teilchen %Teilchens finden
\begin{equation}
  \label{eq:radius_teilchen}
  r=\sqrt{\frac{9\eta\ua{L}v_0}{2\map{g}\left(\rho\ua{oel}-\rho\ua{L}\right)}}
\end{equation}

Neben der Gravitations- und Reibungskraft wirkt auf das geladene Teilchen, %kein Komma
die elektrische Feldkraft $\vec{F}\ua{elk}=q\vec{E}$ des Kondensators. %hast du schon erwähnt
Bei einer positiven Spannung ist die elektrische Kraft in Richtung des der Gewichtskraft gerichtet. %-des
Das Teilchen bewegt sich in diesem Fall mit einer gleichförmigen Sinkgeschwindigkeit $\vec{v}\ua{ab}$ ($\be{\vec{v}\ua{ab}}>v_0$).
Unter dieser Einstellung lautet die Kräftegleichung
\begin{equation}
  \label{eq:kraefte_eins}
    \frac{4\map{\pi}}{3}r^3\left(\rho\ua{oel}-\rho\ua{L}\right)\map{g}-\map{\pi}\eta\ua{L}r v\ua{ab}=-qE.
\end{equation}
Im Gegensatz dazu bewirkt eine negative Spannung eine elektrische Feldkraft die der Gravitationskraft
entgegenwirkt. Folglich bremst es das fallende Tröpfchen ab oder
lässt es, bei hinreichend großer Spannung, sogar nach oben beschleungigen. %beschleunigen
Ab einer bestimmten Spannung erreicht das Tröpfchen eine Schwebezustand. An diesem %einen
Punkt gilt
\begin{equation}
  \label{eq:krafte_gleich}
  \vec{F}\ua{g}=\vec{F}\ua{elk} \quad \Leftrightarrow \quad m\map{g}=\rho V\map{g}=qE, \qquad V=\frac{4}{3}\pi r^3.
\end{equation}
Mit der Gleichung \eqref{eq:krafte_gleich} ergibt sich eine Gesetzmäßigkeit für die Ladung $q$:
\begin{equation}%Formel unschön formatiert
  \label{eq:q}
  q_0=\frac{\frac{4}{3}\pi r^3 \map{g}}{E} =\frac{4}{3} \pi \left( \sqrt{\frac{9\eta\ua{L}v_0}{2\map{g}\left(\rho\ua{oel}-\rho\ua{L}\right)}}\right)^3 \map{g} \frac{U}{d}.
\end{equation}
%Absatz entfernen

Bei der Verwendung des Stokschen Gesetz ist zu beachten, dass es nur für Teilchen gilt,%Stokeschen Gesetzes
deren Abmessung größer als die freie Weglänge in Luft $\ov{l}$. %Verb fehlt
Da diese Voraussetzung beim verwendeten Versuchsaufbau nicht erfüllt, muss die Viskosität $\eta\ua{L}$ %Verb fehlt
angepasst zu %angepasst werden
\begin{equation*}
  \eta\ua{eff}=\eta\ua{L}=\left(\frac{1}{1+B\frac{1}{pr}}\right). \quad B=\SI{9.17e-3}\,\si{\centi\meter}\map{Torr}
\end{equation*}
Hierbei ist $p$ der Druck. Mit der Korrektur ergibt sich ein korrigierter Wert für die Ladung %Luftdruck, redundanter Satz
\begin{equation}
  \label{eq:q_korri}
  q=q_0\left(1+\frac{B}{pr}\right)^{\frac{3}{2}}.
\end{equation}
