\setcounter{page}{1}
\section*{Zielsetzung}
Der Versuch $V503$ soll die Elementarladung $\map{e}_0$ experimentell bestimmt werden.
\section{Theorie}

Um die Elemtarladung $\map{e}_0$ zu ermittelen, wird der nach Robert Andrews Millikan benannte
Milikan-Versuchsaufbau verwendet. Millikan verwendete dazu zerstäubte Öltröpfchen in einem
Plattenkondensator. Beim Zerstäubevorgang erhalten die Öltröpfchen eine elektrische
Ladung, hierdurch stehen sie in Wechselwirkung mit dem elektrischen Feld $\vec{E}$ des
Kondensators. Ein Öltröpfchen trägt immer ein ganzzahliges Vielfaches der Elemetarladung.
Neben der elektrischen Feldkraft des Kondensators, wirkt auchd die Gravitaionskraft
$\vec{F}\ua{g}=m\vec{\map{g}}$. Auf Grund dessen bewegt sich das Tröpfchen zum Boden
des Kondensators. Beim Fallvorgang wirkt die Stoksche Reibungskraft
\begin{equation*}
  \vec{F}\ua{R}=-6\map{\pi}\eta\ua{L}\vec{v}
\end{equation*}
der Bewegung entgegen. Bei der Stoksche Reibungskraft bezeichnet $r$ den Radius der
Tröpfchen und $\eta\ua{L}$ die Viskosität von Luft. Sobald ein Kräftegleichgewicht
zwischen der Reibungskraft $\vec{F}\ua{R}\propto \vec{v}$ und der Gravitationskraft
herrscht, bweget sich das Teilchen mit der konstanten Geschwindigkeit $v_0$ weiter.
Beim kräftegleichgewicht gilt folgender Zusammenhang:
\begin{equation}
  \label{eq:kraeftegleich}
  \frac{4\map{\pi}}{3}r^3\left(\rho\ua{oel}-\rho\ua{L}\right)\map{g}=6\map{\pi}\eta\ua{L}r v_0.
\end{equation}
Die linke Seite von \eqref{q:kraeftegleich} Berücksichtigt zusätzlich den Auftrieb des
Tröpfchen in der Luft. Mit der obigen Gleichung lässt sich ein Ausdruck für den Radius
des Teilchen
\begin{equation}
  \label{eq:radius_teilchen}
  r=\sqrt{\frac{9\eta\ua{L}v_0}{2\map{g}\left(\rho\ua{oel}-\rho\ua{L}\right)}}
\end{equation}

Neben der Gravitations- und Reibungskraft wirkt auf das geladene Teilchen
die elektrische Feldkraft $\vec{F}\ua{elk}=q\vec{E}$ des elektrischen Feldes.
Bei positiver Spannung wirkt die elektrische Kraft in Richtung des der Gewichtskraft.
Das Teilchen bewegt sich also mit einer gleichförmigen Sinkgeschwindigkeit $\vec{v}\ua{ab}$ ($\be{\vec{v}\ua{ab}}>v_0$).
Unter diesen Einstellung lautet die Kräftegleichung
\begin{equation}
  \label{eq:kraefte_eins}
    \frac{4\map{\pi}}{3}r^3\left(\rho\ua{oel}-\rho\ua{L}\right)\map{g}-\map{\pi}\eta\ua{L}r v\ua{ab}=-qE.
\end{equation}
Im Gegensatz dazu bewirkt eine negative Spannung eine elektrische Feldkraft die der Gravitationskraft
entgegenwirkt. Folglich bremst es das fallende Tröpfchen ab oder
lässt es, bei hinreichend großer Spannung, sogar nach oben beschleungigen.
In dem letzteren Fall kann das Tröpfchen die maximale Geschwindigkeit $\vec{v}\ua{auf}$ erreichen.
Die Kräftegleichung lautet
\begin{equation}
  \label{eq:kraefte_zwei}
    \frac{4\map{\pi}}{3}r^3\left(\rho\ua{oel}+\rho\ua{L}\right)\map{g}+\map{\pi}\eta\ua{L}r v\ua{auf}=qE.
\end{equation}
Mit Hilfe der Gleichungen \eqref{eq:kraefte_eins} und \eqref{eq:kraefte_zwei} kann eine Gesetzmäßigkeit
für die Ladung $q$ des Öltröpfchens gefunden werden:
\begin{equation}
  \label{eq:ladung}
  q=3\pi\eta\ua{L}r\frac{v\ua{ab}+v\ua{auf}}{E}=3\pi\eta\ua{L}\sqrt{\frac{9\eta\ua{L}v_0}{2\map{g}\left(\rho\ua{oel}-\rho\ua{L}\right)}}\frac{v\ua{ab}+v\ua{auf}}{E}.
\end{equation}
Bei der Verwendung des Stokschen Gesetz ist zu beachten, dass es nur für Teilchen gilt,
deren Abmessung größer als die freie Weglänge in Luft $\ov{l}$.
Da diese Voraussetzung beim verwendeten Versuchsaufbau nicht erfüllt, muss die Viskosität $\eta\ua{L}$
angepasst
