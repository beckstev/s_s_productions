\section{Diskussion}
Im Folgenden sollen die Messergebnisse in Bezug auf die Messgenauigkeit des
Versuchsaufbaus diskutiert werden.

Als eine mögliche Fehlerquelle des Versuchsaufbaues ist die Empfindlichkeit der
Strommessung zu nennen, da im $\si{\nano\ampere}$ Bereich gemessen wurde, sorgten schon kleinste
Veränderung der Lichtintensität (z.\,B. hervorgerufen durch Erschütterungen) zu einer Veränderung der gemessenen Stromstärke.
Eine weitere Unsicherheit entsteht durch das Spektrum der Hg-Lampe, denn
teilweise war dies nicht diskret, sondern kontinuierlich (z.\,B. bei grün-blau).
Dadurch war eine genaue Justierung der Photozelle erschwert.

Beim Vergleich des Messergebnisses für die Konstante $\frac{\map{h}}{\map{e}}$ mit der Theorie
zeigt sich jedoch, dass die oben beschriebenden Mängel kaum ins Gewicht fallen.
Denn die Gegenüberstellung mit dem Theoriewert\cite{scipy}
\begin{equation}
  \label{eq:abweichung}
  \frac{\map{h}}{\map{e}}\ua{exp}=\left(\num{3.8\pm 0.7}\right)\,\num{e-15}\,\si{\eV}, \quad  \frac{\map{h}}{\map{e}}\ua{theo}=\num{4.1 e-15}\,\si{\eV}, \quad \Delta\approx -8\%
\end{equation}
zeigt eine geringe Abweichung zwischen Experiment und Theorie. Dennoch könnte eine Verbesserung des Versuchsaufbaus
für eine Verringerung der Fehler in \eqref{eq:messergebnisse} sorgen.


Zusammenfassend liefert der Versuch eine gute Möglichkeit den Photoeffekt untersuchen,
durch eine Verbesserung des Versuchsaufbaus wäre eine genaue Bestimmung von $\frac{\map{h}}{\map{e}}$
und der Austrittsarbeit denkbar.
