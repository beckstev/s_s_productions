\setcounter{page}{1}
\section*{Zielsetzung}
Als photoelektrischen Effekt wird das Auslösen von Ladungsträgern aus einer
mit Licht bestrahlten Oberfläche bezeichnet. Im Versuch 500 soll der Photoeffekt beobachtet und die
Abhängigkeit der Energie der ausgelösten Teilchen von der Wellenlänge des Lichtes experimentell %Die Energie der ausgelösten Teilchen wird nicht dirket bestimmt, wir suchen eher die Austrittsarbeit und verschen einen WErt für h/e zu finden.
untersucht werden.  
\section{Theorie}
Wird das Licht im Maxwellschen Sinne als elektromagnetische Welle betrachtet, so müsste
der photoelektrische Effekt bei beliebigen Wellenlängen unter hinreichend großer
Lichtintensität zu beobachten sein. Die Teilchen in der Festkörperoberfläche werden in diesem
Modell durch das einfallende Licht zu erzwungenen Schwingungen angeregt, deren größer werdende
Amplitude schließlich zu einem Überwinden der zurückhaltenden elektrostatischen Kräfte führt.
Darüber hinaus müssten bestimmte Frequenzen existieren, die messbare Resonanzeffekte hervorrufen und die kinetsiche
Energie der ausgelösten Teilchen sollte mit der Lichtintensität steigen.\\
Die tatsächlichen Beobachtung lassen sich wie folgt zusammenfassen:
\begin{itemize}
  \item Die kinetische Energie der ausgelösten Teilchen ist unabhägig von der Lichtintensität und
  hängt ausschließlich von der Frequenz des verwendeten Lichtes ab.
  \item Die Zahl der ausgelösten Teilchen steigt mit der Lichtintensität.
  \item Zwischen Einfall des Lichtes und Austritt der Elektronen existiert keine zeitliche Verzögerungen.
  \item Unterhalb einer Grenzfrequenz ist der Photoeffekt nicht mehr zu beobachten.
\end{itemize}
Die Resultate widersprechen also grundlegend dem Wellencharakter des Lichtes und können lediglich im
Rahmen einer Teilchenvorstellung exakt beschrieben werden.\\
In diesem Modell besteht das Licht aus sogenannten Photonen bzw. Lichtquanten, deren Energie $E\ua{ph}$ proportional
zur Frequenz $\nu$ ist
\begin{equation}
E\ua{ph} = h \nu.
\end{equation}
Die Proportionalitätskonstante $h$ entspricht dem Planckschen Wirkungsquantum ($h \approx \SI{6.626e-34}{\joule\second}$). Ein Photon %Setz h\approx in die nächste Zeile
überträgt nun seine Energie instantan und vollständig auf ein Elektron in der Oberfläche. Damit das Elektron
aus dem Festkörper ausgelöst werden kann, muss zunächst eine Arbeit $A\ua{k}$ geleistet werden. Der Photoeffekt bricht
also ab, wenn gilt
\begin{equation}
  h\nu < A\ua{k}.
\end{equation}
Wird das Elektron ausgelöst, gilt für seine kinetische Energie $E\ua{kin}$ nach dem Energieerhaltungssatz
\begin{equation}
  E\ua{kin} = h\nu - A\ua{k}.
\end{equation}
