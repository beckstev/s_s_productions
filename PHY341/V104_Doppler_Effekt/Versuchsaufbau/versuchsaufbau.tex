\documentclass[parskip=half]{scrartcl} %Ich habe [parskip=half] hinzugefügt

%\usepackage[calc]{picture}
\usepackage{fixltx2e}

\usepackage{tcolorbox}

%\pagestyle{headings}
\usepackage{scrpage2}
\pagestyle{scrheadings}
\ifoot[\pagemark]{\pagemark}
\ofoot[]{}
\cfoot[]{}

\usepackage{polyglossia}
\setmainlanguage{german}

\usepackage{amsmath}
\usepackage{amssymb}
\usepackage{mathtools}

\usepackage{fontspec}
\defaultfontfeatures{Ligatures=TeX}

\usepackage[
  math-style=ISO,
  bold-style=ISO,
  sans-style=italic,
  nabla=upright,
]{unicode-math}

\setmathfont{Latin Modern Math}
%\setmathfont[range={\mathscr, \mathbfscr}]{XITS Math}
%\setmathfont[range=\coloneq]{XITS Math}
%\setmathfont[range=\propto]{XITS Math}

\usepackage[autostyle]{csquotes}

\usepackage[
  locale=DE,                   % deutsche Einstellungen
  separate-uncertainty=true,   % Immer Fehler mit \pm
  per-mode=symbol-or-fraction, % m/s im Text, sonst Brüche
]{siunitx}
%\sisetup{math-stylemicro=\text{µ},text-micro=µ}

\usepackage{xfrac}

\usepackage[section, below]{placeins}
\usepackage[
  labelfont=bf,
  font=small,
  width=0.9\textwidth,
]{caption}

\usepackage{subcaption}

\usepackage{graphicx}

\usepackage{float}
\floatplacement{figure}{h}
\floatplacement{table}{h}

\usepackage{booktabs}

\usepackage{biblatex}
\addbibresource{lit.bib}

\usepackage{bookmark}

\usepackage[shortcuts]{extdash}

\usepackage[math]{blindtext}

\usepackage{microtype}



\usepackage{hyperref}

\usepackage{color} % Das ist Geschmacksfrage

\usepackage{makeidx} %Ich habe makeidx hinzugefügt + makeindex
\makeindex

\newcommand{\tens}[1]{\underline{\underline{#1}}} %Für Tensor mit der Ordnung zwei
\newcommand{\map}[1]{\mathup{#1}} %Befehl für mathup

\usepackage[version=3]{mhchem} % für Thermodynamik-chemische Elemente
\usepackage{enumitem} %Ich habe enumitem hinzugefügt



%\usepackage{showframe}

\author{Steven Becker und Stefan Grisad}

\title{Doppler-Effekt}

\date{\today\\WS 2016/2017}

\newcommand{\ud}{\mathup{d}}
\newcommand{\del}{\partial}

\begin{document}
\maketitle
%\input{""}
\section{Versuchsdurchführung}

Der Versuch Doppler-Effekt besteht aus fünf einzelnen Versuchen.
Zwei dienen dazu die Eigentlichen beiden Messungen vorzubereiten, denn 
zum einen wird die Geschwindigkeit des Wagens, auf dem der
Lautsprecher befestigt ist und zum Anderen wird die Grundfrequenz
$\nu_0$ eines 
frequenzstabilen Generators gemessen.

\subsection{Messung der Relativgeschwindigkeit}

%Die Messung der Realtivgeschwindigkeit, wird benötigt um nacher die Frequenz 
%bzw. die Wellenlängenverschiebung des Doppler-Effekts zu berechnen.
%
%Zur Messung benötigt man einen Wagen, auf dem der Lautsprecher montiert ist,
%zwei Lichtschranken, logische Schaltung und ein Antriebsmotor (10 Gang).
%
%Die Lichtschranken sollen Anfangs- und Endpunkt festlegen.
%Erreicht der Wagen, angetrieben durch den Synchronmotor, die erste 
%Lichtschranke, so aktiviert diese.

Der Versuchsaufbau ist in Abbildung $1$ abgebildet. 
Der durch den zehgängigen Synchronmotor angetriebene Wagen, durch läuft eine 
von zwei Lichtschranken festgelegte Strecke $l$. Dabei dienen die Lichtschranken weiter dazu
eine Zeitmessung zu beginnen bzw. zubeenden. Die dadurch resultierenden Messgrößen
werden dann weiter genutzt um die Geschwindigkeit des Wagens abhängig vom eingestellten
Gang zu berechnen.
Die in der Abbildung zu erkennende logische Schaltung soll im Folgenden einmal kurz
erläutert werden.
Zunächst sind die beiden Phototransistoren an sogenannte Schmitt-Trigger angeschlossen.
Diese haben die Aufgaben die vom Photoransistor regrestierten Impulse in ein 
für logische Gatter essenzieles TTL-Signal umzuwandeln. Weiter sind beide Schmitt-Trigger an
eine bistabile Kippstufe angeschlossen, die sobald einer der Transistor einen Impuls sendet.
Entweder auf High (Messung beginnt) oder auf Low (Messung endet) schaltet. Dadurch wird 
das folgende Und-Gatter für Signale durchlässig oder nicht.
Der neben der bistabilenKippstufe angeschlossene, Zeitgenerator liefert mit einer Genauigkeit von
$10^{-5}$ im Abstand von $1\si{\micro\second}$ einen Impuls.
Aufgrund dieser Eigenschaft übernimmt er in diesem Versuch die Rolle einer Uhr.
Der nach dem Zeitgenerator eingebaute Untersetzer dient hier dazu, um 
ein Overflow des Zählwerks zu verhindern. Denn er verringert die Anzahl der Impulse um 
den einstellbaren Faktor $10^n$.

\subsection{Messung der Schallgeschwindigkeit}

Der Versuchsaufbau zur Messung der Schallgeschwindigkeit ist der
Abbildung \textbf{ NUMMER EINFÜGEN} zu entnehmen.

Im Versuch wird der vom Quarzgenerator erzeugte und der vom Mikrofon gemessene 
Ton, jeweils auf die x- und y-Achse eines Ozilliskop aufgetragen.
Dadurch können sogenannte Lissajou-Figuren erzeugt werden.
Diese werden genutzt, um den auf dem Lautsprecher montierten Präzisionsschlitten 
(Genauigkeit von $10\si{\micro\meter}$), so 
einzustellen, dass beide Signale in Phase sind. 
Dies ist genau dann der Fall, wenn die Lissajou-Figur eine Gerade ergibt.
Wenn nun immer der Abstand, zwischen zwei 
Signalen gleicher Phase, gemessen wird, kann auf die Wellenlänge geschlossen werden.
Mithilfe der der Ruhefrequenz des Generators und der gerade gemessenen 
Wellenlängen kann dann die Schallgeschwindigkeit gefolgert werden.



\subsection{Frequenzmessung}
Der Oberbegriff Frequenzmessung fasst jeweils drei verschiedene
Frequenzmessung zusammen.
Die Messung der Grundfrequenz $\nu_0$ 
für die darauf folgende Messung der Frequenz einer bewegten Quelle wichtig. Denn sie wird für eine nachfolgende BErechnung benötigt.
benötigt. Abschließend wird noch die Frequenz einer Schwebung gemessen.

An dieser Stelle soll noch eine kurze Erklärung der logischen Schaltung
für die beiden folgenden Messungen folgen.

Wie im obigen Versuch haben wir auch hier wieder an einen
Schmitt-Trigger angeschlossen Phototransistor. 
Der Schmitt-Trigger bzw. der Phototransistor übernehmen wieder die Aufgabe 
einen eingestellten Messzeitraum zu starten. Denn nach dem der 
Schmitt-Trigger den Impuls des Phototransistors in ein TTA-Signal 
umgewandelt hat, geht dieses in eine Bistabile-Kippstufe.
Das nun von der Kippstufe ausgesendete Signal dient, dazu um zwei
Und-Gatter zu schalten. 
Das eine Und-Gatter ist mit dem Mikrofon und einem Impulsforma (Sinusschwingung zu Rechteckschwingung)
verbunden. Das heißt, sobald dass Und-Gatter vom Flip-Flop ein High Potential erhält, wird es freigeschaltet
und das dahinter liegende Zählwerk kann die vom Mikrofon gesendeten Impulse (nur ganzzahlig) zählen.
Das andere Und-Gatter ist noch mit dem Zählwerk verbunden. Wird es nach dem High Impuls von der
Kippstufe freigeschaltet, steuert das Zählwerk mit regelmäßigen Impulsen (1 pro $1\si{\micro\second}$)
einen Untersetzer. Der Untersetzer hat die Aufgabe ein reproduzierbares Zeitintervall festzulegen.
Denn nach $10^n$ Signalen vom Zeitbasisgenerator sendet der Untersetzer ein Low Impuls an die Bistabilekippstufe.
Dadurch wird dann der Messzeitraum beendet.

\subsubsection{Ruhefrequenz}

Um die Ruhefrequenz zu messen, bewegt sich der auf dem Wagen montierte Lautsprecher nicht. 
Das vom Lautsprecher erzeugte Signal wird vom Mikrofon wahrgenommen und vom Zählwerk gemessen.
Nachdem Ablauf des vorher festgelegten Zeitraums, kann dann die Grundfrequenz bestimmt werden.

\subsection{Bewegte Quelle}

Im Gegensatz zur Messung der Ruhefrequenz bewegen sich hier Wagen und Lautsprecher.
Die Geschwindigkeit ist dabei abhängig von dem beim Motor eingestellten Gang.
Sobald der Wagen jetzt die Lichtschranke durchläuft, beginnt der Messzeitraum und
endet nach der beim Untersetzer eingestellten Zeit. Wichtig, ist noch zu erwähnen, dass hier 
nicht nur eine Bewegungsrichtung gemessen wird. Das heißt, es werden jeweils Messreihen 
aufgestellt, inden sich der Wagen in positiver (auf das Mikrofon) oder in negative (weg vom Mikrofon) Richtung
bewegt.

\subsubsection{Schwebungsmethode}

Eine Schewbung entsteht allgemein, wenn sich zwei Schwingungen mit ähnlicher Frequenz überlagern.
Es kommt zu einer zeitabhängigen Amplitude. Den Effekt der Schwebung macht man sich bei diesem Versuch
zunutze.

Der grundlegende Versuchsaufbau ist in Abbildung \textbf{Abbildung einfügen} zu sehen.
Der Lautsprecher wird neben dem Mikrofon plaztiert, aber so das er in Richtung eines
auf dem Wagen montierten Spigel zeigt. 
Denn dadurch misst das Mikrofon nicht nur die Grundschwingung vom Lautsprecher, sondern auch
die vom Spiegel reflektierte. 
Es sei zu erwähnen, dass sich der Lautsprecher in der Nähe vom Mikrofon befinden muss, da sonst keine Schwebung messbar ist.
Besagter Spiegel befindet sich auf dem Wagen. Dieser witd vom zehngängigen Motor angetrieben.
angetrieben. Dadurch kommt es bei der vom Spiegel reflektierten Schwingung zu einer Frequenzänderung
und eine Schwebung entsteht.
Die vom Mikrofon gemessene Schwingung wird anschließend durch einen Gleichrichter und einen Tiefpass geschickt.
Der Tiefpass filtert nun die Schwebung heraus. Da auch in diesem Versuch wieder ein Untersetzer eingesetzt wird, um
ein Zeitintervall festzulegen. Kann die Frequenz der Schewbung danach bestimmt werden.
Diese Frequenz entspricht dann gerade der Frequenzdifferenz durch den Dopplereffekt.
%2v noch erwähnen? Wie erwähnen ?
 

\printbibliography
\printindex
\end{document}