\section{Diskussion}

Im folgenden sollen die gewonnen Ergebnisse aus dem voran gegagangen Abschnitt, mit Hinblick auf die Präzision
des verwendeten Versuchsaufbaus, auf ihre Plausibilität hin überprüft werden.  \\
Der Literaturwert für die Schallgeschwindigkeit beträgt $v_{Schall} = 330 m/s$ \cite{numpy}. Der in (Bezug) ermittelte Wert weicht
um etwa $8,3 \% $ nach unten hin ab. Als maßgebliche Fehlerquelle ist hierfür das unpräzise Ablesen der Lissajou-Figuren zu nennen. Die
vom Oszilloskop visualisierte verstärkte Signalspannung des Mikrophons wich stark von dem ausgesandten Signal des Generators ab, was es kompliziert
machte die entsprechenden Figuren zu reproduzieren.
In den drei Teilversuchen (Bezug...) wurde mit verschiedenen Methoden jeweils der Wert für die inverse Wellenlänge bestimmt. Hierbei weichen alle drei Werte realtiv stark
voneinander ab, wobei die Abweichung der Messung mittles der Schwebungsmethode von der Direktmessung der Frequenzänderung mit $21.5 \%$ deutlich höher ausfällt,
als die Diskrepanz des Wertes aus der Wellenlängenmessung im Vergleich mit selbiger ($11.9 \% $). Der qualitative Vergleich der drei Werte mit Hilfe des Studentschen t-Tests
(Abschnitt bla) zeigt, dass mit großer Wahrscheinlichkeit systematische Fehler aufgetreten sind. \\
Als Begründung hierfür sind meherere Bestandteile des verwendeten Versuchsaufbaus zu nennen.
Bei der Messung der Geschwindigkeit des durch den Synchrotronmotor angetreieben Wagens zeigte sich, dass die Laufzeiten
durchaus stark variierten, was sich in der Standardabweichung des Mittelwerts (Bezug) wiederspiegelt. In Verbindung mit der bereits erwähnten
unpräzisen Tonmessung sieht sich hier die Ungenauigkeit der Frequenz- und insbesondere Schwebungsmessung begründet. \\
Zum Abschluss sei noch zu erwähnen, dass die verwendeten Lichtschranken ab der Geschwindigkeit in Gang 36 nicht mehr auslösten. Da sich hierdurch
die Größe der Stichprobe in allen Messungen halbierte, ist die Signifikanz der Ergebnisse als relativ gering einzustufen.
