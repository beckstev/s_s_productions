\begin{table}
\centering
\caption{Gemessene Anzahl an Zerfällen bei Indium.}
\label{tab: indium_messwerte}
\begin{tabular}{S S S S S }
\toprule
{$t$ in $\si{\second}$} & {Anzahl $N$} & {$\sigma\ua{N}$} & {$\sigma\ua{ln,+}(N)$} & {$\sigma\ua{ln,-}(N)$}  \\
\midrule
 240  & 1914  & 44  & 0.02  & 0.02\\ 
480  & 1771  & 42  & 0.02  & 0.02\\ 
720  & 1687  & 41  & 0.02  & 0.02\\ 
960  & 1665  & 41  & 0.02  & 0.02\\ 
1200  & 1577  & 40  & 0.02  & 0.03\\ 
1440  & 1461  & 38  & 0.03  & 0.03\\ 
1680  & 1491  & 39  & 0.03  & 0.03\\ 
1920  & 1319  & 36  & 0.03  & 0.03\\ 
2160  & 1361  & 37  & 0.03  & 0.03\\ 
2400  & 1233  & 35  & 0.03  & 0.03\\ 
2640  & 1225  & 35  & 0.03  & 0.03\\ 
2880  & 1141  & 34  & 0.03  & 0.03\\ 
3120  & 1062  & 33  & 0.03  & 0.03\\ 
3360  & 1026  & 32  & 0.03  & 0.03\\ 
3600  & 948  & 31  & 0.03  & 0.03\\ 
\bottomrule
\end{tabular}
\end{table}

Die graphische Darstellung ist in Abbildung \ref{} zufinden, bei dieser wurde
