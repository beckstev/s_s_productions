\begin{table}
\centering
\caption{Gemessene Anzahl an Zerfällen bei Indium.}
\label{tab: indium_messwerte}
\begin{tabular}{S S S S S }
\toprule
{$t$ in $\si{\second}$} & {Anzahl $N$} & {$\sigma\ua{N}$} & {$\ln{\left(N+\sigma\ua{N}\right)}-\ln{N}$} & {$\ln{\left(N-\sigma\ua{N}\right)}-\ln{N}$}  \\
\midrule
 240  & 1914.5  & 43.8  & 0.02  & 0.02\\
480  & 1771.5  & 42.1  & 0.02  & 0.02\\
720  & 1687.5  & 41.1  & 0.02  & 0.02\\
960  & 1665.5  & 40.8  & 0.02  & 0.02\\
1200  & 1577.5  & 39.7  & 0.02  & 0.03\\
1440  & 1461.5  & 38.2  & 0.03  & 0.03\\
1680  & 1491.5  & 38.6  & 0.03  & 0.03\\
1920  & 1319.5  & 36.3  & 0.03  & 0.03\\
2160  & 1361.5  & 36.9  & 0.03  & 0.03\\
2400  & 1233.5  & 35.1  & 0.03  & 0.03\\
2640  & 1225.5  & 35.0  & 0.03  & 0.03\\
2880  & 1141.5  & 33.8  & 0.03  & 0.03\\
3120  & 1062.5  & 32.6  & 0.03  & 0.03\\
3360  & 1026.5  & 32.0  & 0.03  & 0.03\\
3600  & 948.5  & 30.8  & 0.03  & 0.03\\
\bottomrule
\end{tabular}
\end{table}

Die graphische Darstellung ist in Abbildung \ref{} zufinden, bei dieser wurde
