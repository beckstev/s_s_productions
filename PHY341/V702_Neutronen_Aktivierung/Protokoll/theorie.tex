\setcounter{page}{1}
\section*{Zielsetzung}
Im Versuch 702 \textit{Aktivierung durch Neutronen} soll der Zerfall instabiler Atomkerne untersucht werden.
Hierzu werden verschiedene Isotope durch Neutronenbeschuss angeregt und die anschließende Rückkehr in einen
stabilen Zustand mittels der experimentellen Bestimmung der Halbwertszeit charakterisiert.
\section{Theorie}
Absorbiert der Kern \ce{^m_z A} ein Neutron \ce{^1_0 n}, so geht er aufgrund des Energiegewinns
zunächst in einen angeregten Zustand über, der unter Aussendung eines $\gamma$-Quants wieder
auf einen Grundzustand zurück fällt:
\begin{equation}
  \ce{^m_z A + ^1_0 n -> ^{m+1}_z A^* -> ^{m+1}_z A }+ \gamma.
\end{equation}
Dies geschieht etwa innerhalb der ersten $\SI{e-16}{\second}$. Ist der entstandene Kern \ce{^{m+1}_z A}
nicht stabil, sprich ist das Verhältnis aus Neutronen und Protonen nicht entsprechend jenem der Kerne
gleicher Ordnungszahl, so zerfällt das Reaktionsprodukt weiter gemäß
\begin{equation}
  \ce{^{m+1}_z A -> ^{m+1}_{z+1} C }+ \beta^{-} + E\ua{kin} + \bar{\nu}\ua{e}.
\end{equation}
Der Kern wandelt sich also durch Aussendung eines ELektrons und eines Antineutrinos in einen Kern höherer Ordnungszahl $\mathup{z} + 1$, wobei
sich die kinetische Energie der beiden Teilchen aus dem Massendefekt $\Delta m$ berechnet
\begin{equation}
  E\ua{kin} = \Delta m \cdot c^2.
\end{equation}
Ein Maß für die Wahrscheinlichkeit mit der es zur Reaktion zwischen Neutron und Kern kommt ist der Wirkungsquerschnitt
$\sigma$\footnote{$[\sigma] = \SI{e-24}{\centi\meter^2} := \SI{1}{\barn}$}. Er entspricht
der Fläche eines Ziels (hier Kern), das alle ausgesandten Neutronen einfangen würde. Für $N\ua{F}$ Kerne innerhalb einer betrachteten Beschussfläche
$F$ berechnet sich der Wirkungsquerschnitt also gemäß
\begin{equation}
  \sigma = \frac{n\ua{r}}{n\ua{a}}\frac{F}{N\ua{F}} = P \,\frac{F}{n\ua{F}}.
\end{equation}
Mit dem Verhältnis aus Teilchen, die zur Reaktion geführt haben, $n\ua{r}$ und ausgesandten Teilchen $n\ua{a}$. Der Wirkungsquerschnitt steigt also
mit der Wahrscheinlichkeit $P$. Im Fall der Neutronenaktivierung ist anschaulich klar, dass langsame Neutronen wahrscheinlicher in Wechselwirkung mit dem
Atomkern treten asl schnelle. Es gilt für kleine Energien der Neutronen $E$ mit Geschwindigkeit $v$
\begin{equation}
  \sigma \propto \frac{1}{\sqrt{E}} \propto \frac{1}{\sqrt{v}}.
\end{equation}
Für das Experiment eignen sich somit insbesondere langsame Neutronen, da sie die Wahrscheinlichkeit der Kernaktivierung erhöhen.
