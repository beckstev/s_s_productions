\setcounter{page}{1}
\section*{Zielsetzung}
Im Versuch 702 \textit{Aktivierung mit Neutronen} soll der Zerfall instabiler Atomkerne untersucht werden.
Hierzu werden verschiedene Isotope durch Neutronenbeschuss angeregt und die anschließende Rückkehr in einen
stabilen Zustand mittels der experimentellen Bestimmung der Halbwertszeit charakterisiert.
\section{Theorie}
Absorbiert der Kern \ce{^m_z A} ein Neutron \ce{^1_0 n}, so geht er aufgrund des Energiegewinns
zunächst in einen angeregten Zustand (auch Zwischenkern oder Compound-Kern genannt) über, der unter Aussendung eines $\gamma$-Quants wieder
auf einen Grundzustand zurück fällt:
\begin{equation}
  \ce{^m_z A + ^1_0 n -> ^{m+1}_z A^* -> ^{m+1}_z A }+ \gamma.
\end{equation}
Hierbei handelt es sich um Gamma-Strahlung. Der Vorgang geschieht etwa innerhalb der ersten $\SI{e-16}{\second}$.
Ist der entstandene Kern \ce{^{m+1}_z A}
nicht stabil, zerfällt das Reaktionsprodukt weiter gemäß
\begin{equation}
  \ce{^{m+1}_z A -> ^{m+1}_{z+1} C }+ \beta^{-} + E\ua{kin} + \bar{\nu}\ua{e}.
\end{equation}
Dies ist dann der Fall, wenn das Verhältnis aus Neutronen und Protonen nicht jenem der stabilen Kerne gleicher Ordnungszahl entspricht.
Typischerweise liegt die Neutronenzahl 20-50\% höher als die Protonenzahl.
Der Kern wandelt sich durch Aussendung eines Elektrons und eines Antineutrinos in einen Kern höherer Ordnungszahl $\mathup{z} + 1$, wobei
sich die kinetische Energie der beiden Teilchen aus dem Massendefekt $\Delta m$ berechnet
\begin{equation}
  E\ua{kin} = \Delta m \cdot c^2.
\end{equation}
Dieser Prozess wird als $\beta$-Strahlung bezeichnet. Im Versuch werden die Zerfälle von Indium- und Rhodiumisotopen untersucht.
Indium zerfällt nach der Aktivierung in Zinn gemäß folgender Gleichung
\begin{equation}
  \ce{^115_49 In + ^1_0 n ->^116_49 In -> ^116_50 Sn} + \beta^- + \bar{\nu}\ua{e}.
\end{equation}
Für das Rhodiumisotop treten mit verschiedenen Wahrscheinlichkeit zwei mögliche Zerfälle auf:
\begin{equation}
\ce{^103_45 Rh + ^1_0 n} \begin{cases}
\overset{10\%}{\ce{->}} \ce{^{104i}_45 Rh -> ^{104}_45 Rh} + \gamma \ce{-> ^104_46 Pd} +  \beta^- + \bar{\nu}\ua{e}\\
\overset{90\%}{\ce{->}} \ce{^{104}_45 Rh ->  ^104_46 Pd} +  \beta^- + \bar{\nu}\ua{e}.
\end{cases}
\end{equation}
Hierbei ist anzumerken, dass der auftretende Zerfall des aus dem \ce{^{104i}_45 Rh} Isomer entstandenen \ce{^{104}_45 Rh} gegenüber dem
Gamma-Zerfall des \ce{^{104i}_45 Rh} vernachlässigt werden kann. Der Versuch ermöglicht somit eine Untersuchung der Zerfallszeiten
der zueinander isomeren Kerne \ce{^{104i}_45 Rh} und \ce{^{104}_45 Rh}.

\subsection{Wirkungsquerschnitt}
Ein Maß für die Wahrscheinlichkeit mit der es zur Reaktion zwischen Neutron und Kern kommt ist der Wirkungsquerschnitt
$\sigma$\footnote{$[\sigma] = \SI{e-28}{\meter^2} := \SI{1}{\barn}$}. Er entspricht
der Fläche eines Ziels (hier Kern), in welchem alle ausgesandten Neutronen zur Reaktion führen. Für $N\ua{F}$ Kerne innerhalb einer betrachteten Beschussfläche
$F$ berechnet sich der Wirkungsquerschnitt also gemäß
\begin{equation}
  \sigma = \frac{n\ua{r}}{n\ua{a}}\frac{F}{N\ua{F}} = P \,\frac{F}{n\ua{F}}.
\end{equation}
Mit dem Verhältnis aus Teilchen, die zur Reaktion geführt haben $n\ua{r}$ und ausgesandten Teilchen $n\ua{a}$. Der Wirkungsquerschnitt steigt also
mit der Wahrscheinlichkeit $P$. Im Fall der Neutronenaktivierung gilt für kleine Energien der Neutronen $E$ mit Geschwindigkeit $v$
\begin{equation}
  \sigma \propto \frac{1}{\sqrt{E}} \propto \frac{1}{v}.
\end{equation}
Für das Experiment eignen sich somit insbesondere langsame Neutronen, da sie die Wahrscheinlichkeit der Kernaktivierung erhöhen.

\subsection{Erzeugung freier Neutronen}
Um Neutronen freizusetzen, wird der Zerfall des \ce{^226 Ra}-Isotops genutzt.
Die dabei frei werdenen $\alpha$-Teilchen (Helium-Kerne) lösen bei Beschuss auf eine Beryllium-Probe
folgende Kernreaktion aus
\begin{equation}
  \ce{^9_4 Be + ^4_2}\alpha \ce{-> ^12_6C + ^1_0 n}.
\end{equation}
Durch den elastischen Stoß der Neutronen mit den Kernen einer Paraffin Ummantelung wird deren kinetische Energie
auf etwa $\SI{0.025}{\eV}$ herabgesetzt. In diesem Zustand werden sie als thermische Neutronen bezeichnet. Ihre
Energie entspricht jener der umliegenden Moleküle, welche diese durch thermische Bewegung erhalten.

\subsection{Zerfallsgesetz}
Die zeitliche Entwicklung der noch nicht zerfallenen Kerne $N(t)$ in einer Probe wird durch das sogenannte
Zerfallsgesetz beschrieben.
\begin{equation}
  N(t) = N_0 \cdot \exp\left(-\lambda t  \right).
  \label{eq: zerfallsgesetz}
\end{equation}
$N_0$ entspricht hierbei einem Startwert zur Zeit $t = 0$. Wesentlich entscheidender ist jedoch die sogenannte Zerfallskonstante $\lambda$,
aus der die Halbwertszeit $T$ berechnet werden kann.
\begin{align}
  \begin{aligned}
    \frac{1}{2} N_0 &= N_0 \exp\left(-\lambda T \right) \\
    \Rightarrow \, T &= \ln\frac{2}{\lambda}.
  \end{aligned}
  \label{eq: halbwertszeit}
\end{align}
Statt einer kontinuierlichen Messung der vorhandenen Kerne $N(t)$ wird die Anzahl der Zerfälle $N\ua{\Delta t}$ innerhalb eines
konstanten Zeitintervalls $\Delta t$ aufgenommen.
\begin{align}
  N\ua{\Delta t} &= N(t) - N(t + \Delta t) \\
  \Rightarrow \, \ln N\ua{\Delta t} &= \underbrace{\ln\left[N_0\left(1 - e^{-\lambda \Delta t} \right) \right]}_{:=b} - \lambda t \label{eq: n_prozeitintervall}\\
  &= -\lambda t + b.
\end{align}
Der mit $b$ gekenntzeichnete Term ist konstant. Durch lineare Ausgleichsrechnung kann somit die Zerfallskonstante und damit die
Halbwertszeit aus einer Messung der Zerfallsrate ermittelt werden. \\
Auch ohne Anwesenheit einer radioaktiven Probe kann stets eine natürliche Radioaktivität nachgewiesen werden. Dieser sogenannte
Nulleffekt muss bei der Untersuchung der Halbwertszeit berücksichtigt werden.
Bei bekanntem Nulleffekt gilt für $N\ua{\Delta t}$
\begin{equation}
  N\ua{\Delta t} = N\ua{\Delta t, ges} - N\ua{0, \Delta t}.
  \label{eq: nullrate}
\end{equation}
Hierbei entspricht $N\ua{\Delta t, ges}$ der Gesamtzahl an gemessenen Impulsen und $N\ua{0, \Delta t}$ der Nullrate
bezogen auf das Intervall $\Delta t$.
