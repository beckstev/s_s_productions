\section{Theorie}

\subsection{Trägheitsmoment}
Für das Trägheitsmoment eines starren Körpers gilt:
\begin{equation}
   I = \int r^2 \dif{m}
\end{equation}

Hierbei entspricht $r$ dem Abstand eines Massenelements $dm$ zur Rotationsachse. Dies
lässt sich mit Hilfe der Dichte des Körpers $\rho(\vec{r}) = \dif{m}/\dif{V}$ umformulieren zu:
\begin{equation}
  I = \int \rho(\vec{r})\cdot r^2 \dif{V}
\end{equation}

Das Trägheitsmoment ist stets bezüglich einer Achse definiert und ist eine
additive Größe. Der so genannte \textit{Satz von Steiner} macht eine Aussage über die Änderung des Trägheitsmoments
bei paralleler Verschiebung jener Achsen, die durch den Schwerpunkt verlaufen. Ist das
Trägheitsmoment $I_S$ eines Körpers der Masse $m$ bezüglich einer solchen Achse bekannt, so ergibt sich das
Trägheitsmoment bezüglich einer um die Länge $a$ parallel verschobenen Achse zu:

\begin{equation}
    I = I_S + m \cdot a^2
    \label{eq: steiner}
\end{equation}

\subsection{Drehbewegung}
Für den Drehimpuls eines rotierenden Körpers gilt:
\begin{equation}
 \vec{L} = I \cdot \vec{\omega}
\end{equation}

Bei konstantem Trägheitsmoment folgt mit dem Drehwinkel $\phi$ für die zeitliche Änderung, die dem Drehmoment
entspricht:
\begin{equation}
  \dot{\vec{L}} = \vec{M} = I \cdot \frac{d}{dt}\vec{\omega} =
  I \cdot \frac{d^2}{dt^2} \phi \cdot \frac{\vec{\omega}}{|\vec{\omega}|}
  \label{eq: drehmoment}
\end{equation}
In einem schwingunsfähigen Drehsystem wirkt z.B. durch eine Torsionsfeder ein
rücktreibendes Moment $M_{r}$, dessen Stärke durch eine Winkelrichtgröße $D$ gewichtet
werden kann.
\begin{equation}
  M_{r} = - D \cdot \phi
\end{equation}
Mit \eqref{eq: drehmoment} ergibt sich die homogene Differentialgleichung:
\begin{equation}
  \ddot{\phi} + \frac{D}{I}\phi = 0
\end{equation}
Hiebei handelt sich um die charakteristische Differentialgleichung des harmonischen
Oszillators, deren Allgemeine Lösung mit $\omega = \sqrt{\frac{D}{I}}$ folgendermaßen angegeben werden kann:
\begin{equation}
  \phi(t) = A\cdot \cos{\omega t} + B \cdot \sin{\omega t}
\end{equation}
Zwischen Drehfrequenz und Schwingungsdauer $T$ besteht der bekannte Zusammenhang
$T =\omega / 2\pi$. \\
Konkret werden im Versuch die folgenden Formeln zur Berechnung von Trägheitsmomenten benötigt:
\begin{enumerate}
  \item homogene Kugel mit Radius R, Gesamtmasse M, Rotation um Symmetrieachse:
  \begin{equation}
    I = \frac{2}{5}M R^2
  \end{equation}

  \item homogener Zylinder mit Radius R, Höhe h, Gesamtmasse M, Rotation um Symmetrieachse:
  \begin{equation}
    I = \frac{1}{2} M R^2
  \end{equation}

  \item homogener Zylinder mit Radius R, Höhe h, Gesamtmasse M, Rotation um Achse durch den Schwerpunkt senkrecht zur Symmetrieachse:
  \begin{equation}
    I = M (\frac{R^2}{4} + \frac{h^2}{12})
  \end{equation}
\end{enumerate}
