\section{Diskussion}
In diesem Kapitel soll eine Art Plausibilitätsprüfung in Bezug auf die
gewonnen Einblicke, die durch die Experimente erfolgt sind folgen.
Dabei beginnen wir zunächst bei der Messung der Schwingungsdauer.
Hier sei direkt der Mensch und seine Reaktionszeit als Fehlerquelle anzumerken. 
Denn eine exakte Festlegung von Anfangs- und Endpunkt einer jeden Messung ist 
nicht möglich. % Denn es ist, extrem schwierig Anfangs- und Endpunkt einer jeden Zeitmessung exakt 
Doch durch eine Mittlung 
über 5 Messwerte sorgt dafür, dass der so entstehende Fehler vergleichsweise klein wird.

Der enstande Unterschied zwischen der passiv und dynamisch gemessener Winkelrichtgörße,
lässt sich auf einige Fakoren zurückführen.
Zum einen ist das Ablesen der Kraft bei der passiven Methode 
druch die grobe Skalierung ungenau. 
Zweitens ist es das positionieren der Federwaage orthogonal zum Radius
nicht ganz genau. 
Bei der dynamischen Methode ist auch das oben angesprochene messen der Schwingungsdauer 
eine Fehelerquelle.
Dazu kommt die Messungenauigkeit der Waage, bei der Gewichtsmessung. 
Betrachtet man die Messung des Eigenträgheitsmomentes der Drillachse so fällt sofort auf, 
dass experimentelle Größen mit theoretisch bestimmten Größen (Trägheitsmoment der beiden Zylinder) 
gemischt worden sind. 
Dies vermindert die Aussagekraft der Messergebnise.
Hinzu kommmt noch das die  Höhen und Radien der Zylinder, ganz genau bestimmt werden können.
Da die Ausrichtung der Massen auf dem Stab nicht hundertprozentig genau sind, was 
bedeutet das eine vollkom symetrische Ausrichtung  der Zylinder nicht gewährleistet ist. 
Folgt somit eine weitere Unsicherheit die in das Resultat einfließt.
Des weiteren wurde die Masse des Stabes selber in allen theoretischen Berechnung vernachlässigt.
Das Resultat der Berechnung verliert so teils an Aussagekraft.

Die Messung der Trägheitsmoment von Kugel und Zylinder (grau) fiel unterschiedlich aus.
Bei dem Zylinder war eine Abweichung von $+33\%$ zur Theorie festzustellen, 
dies lässt sich wahrscheinlich auf die ungenauen Größe $I_S$ zurückführen.
Ist aber dennoch als ein positives Ergebnis einzustufen.
Hingegen war die Abweichung bei der Kugel $\pm 0\%$, dieser Wert überrascht zeigt,
aber auch wie genau mechanische Messungen von Hand getätigt werden können.
Die wohl größte Ungenauigkeit fand man bei der Messung des Trägheitsmomentes der Puppe. 
Denn hier lag die Abweichung zwischen Theorie und Praxis bei
$+1200\%$ (Position 1). Dieses Ergebnis lässt sich auf die 
geometrisch uneganue theoretische Berrechnung zurückführen.
Denn zum einen wurde die Puppe stark vereinfacht siehe \ref{fig:approx_puppe}, als auch
die Befestigung der Puppe vernachlässigt.  %kann man das so schreiben ??

Abschließend ist zu sagen, dass sich der Zusammenhang zwischen der Theorie und den experimentellen Befunden
gerade bei dem Zylinder und der Kugel bestätigt haben.
Lediglich bei der Puppe ist keine qualitative Aussage möglich.