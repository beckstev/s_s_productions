\section{Theorie}

\subsection{Prinzip der Wärmepuppe}

Der Erfahrung nach fließt Wärme stets
vom wärmeren Medium $T_1$ zum kälteren Medium $T_2$.
Um diesen Effekt umzukehren, nutzt man die Wärmepumpe.
Durch Zufuhr von Energie (z. B. mechanische Arbeit) sagt der erste
Hauptsatz der Thermodynamik mittels,

\begin{equation}
\label{eq:hst_1}
Q_1=Q_2+A
\end{equation}

dass die im Warmen aufgenommene Wärmemenge $Q_1$ gleich
der Summe der aus dem Kalten entnommene Wärmemenge $Q_2$
und der zugeführten Arbeit $A$ ist.

Jede Wärmepumpe besitzt eine sogenannte Güteziffer $\nu$.
Diese gibt das Verhältnis zwischen transportierter Wärmemenge
und der dazu benötigten Arbeit $A$ an.
Aus dem ersten Hauptsatz ergibt sich unter idealen Voraussetzungen:

\begin{equation}
\label{eq:best_kennziffer}
\nu=\frac{Q_1}{A}
\end{equation}

Betrachtet man zusätzlich den zweiten Hauptsatz der Thermodynamik, so
ergibt sich ein weiterer Zusammenhang zwischen den Wärmemengen
$Q_1$ und $Q_2$ und den Temperaturen der Medien $T_1$ und $T_2$.
Ändert sich die Temperatur der beiden Medien während der
Wärmeübertragung nicht, so verschwindet die reduzierte Wärmemenge und es folgt

\begin{equation}
\label{eq:hst_2}
\frac{Q_1}{T_1}-\frac{Q_2}{T_2}=0
\end{equation}

Jedoch ist für \eqref{eq:hst_2} eine Voraussetzung,
dass der Prozess reversibel ist.
Das bedeutet, dass die in einem thermodynamischen Prozess aufgenommene Wärme
und Energie bei Umkehrung des Versuches wieder zurückfließen muss.
In der Realität ist dies durch Verlustwärme und Reibungsprozesse
nicht zu realisieren.
Dadurch stellt sich für die reale Wärmepumpe eine andere
Güterziffer $\nu_{real}$ ein.
Sie lässt sich mittels der idealen Güteziffer
\begin{equation*}
\nu_{id}=\frac{T_1}{T_1-T_2}
\end{equation*}
abschaätzen zu:

\begin{equation*}
\nu_{real}<\frac{T_1}{T_1-T_2}
\end{equation*}

Das bedeutet je geringer die Differenz zwischen $T_1$ und $T_2$,
desto höher ist die Effizienz der Wärmepumpe.

\subsection{Bestimmung der realen Güteziffer}
Aus einer von $t$ abhängigen Messreihe $T_1$ wird der Differenzenquotient $\frac{\Delta T_1}{\Delta t}$ bestimmt.
Damit kann dann die Wärmemenge bestimmt werden:

\begin{equation*}
\frac{\Delta Q_1}{\Delta t}=\left(m_1c_w+m_kc_k\right)\frac{\Delta T_1}{\Delta t}
\end{equation*}

Durch Bestimmung einer Ausgleichskurve, kann der
Differenzenquotient durch einen Differentialquotienten
ersetzt werden.
Dies soll in allen folgenden Rechnungen gelten.
Man erhält:

\begin{equation}
\label{eq:warmemenge_1}
\frac{\mathup{d} Q_1}{\mathup{d} t}=\left(m_1c_w+m_kc_k\right)\frac{\mathup{d} T_1}{\mathup{d} t}
\end{equation}

Wobei $m_1c_w$ die Wärmekapazität des Wassers im Behälter $1$ ist und
$m_kc_k$ die Wärmekapazitäten der Kupferschlange und des Eimers sind.
Mit \eqref{eq:best_kennziffer} und \eqref{eq:warmemenge_1} folgt dann

\begin{equation}
\label{eq:bestimmung_ziffer}
\nu_{real}=\frac{\mathup{d}Q_1}{\mathup{d}tP}
\end{equation}

Hierbei sei $P$ die Leistungsaufnahme des Wattmeter im Zeitraum $\mathup{d}t$.

\subsection{Bestimmung des Massendurchsatzes}

Bei Betrachtung der Messreihe $T_2$ bildet man den Differentialquotient $\frac{\mathup{d}T_2}{\mathup{d}t}$.
Damit kann dann mittels

\begin{equation*}
\frac{\mathup{d} Q_2}{\mathup{d} t}=\left(m_2c_w+m_kc_k\right)\frac{\mathup{d} T_2}{\mathup{d} t}
\end{equation*}Der

die Wärmemenge, die pro Zeit $\mathup{d}t$ entnohmen wird, bestimmt werden.
Da es sich bei diesem Vorgang um eine Aggregatzustandsänderung handelt,
muss die Wärmemenge gleich der Verdampfungswärme $L$ pro Massenzeitstückchen $\frac{\mathup{d}m}{dt}$.
Also:

\begin{equation*}
\frac{\mathup{d} Q_2}{\mathup{d} t}=L\frac{\mathup{d}m}{dt}
\end{equation*}

\subsection{Bestimmung der mechanischen Kompressorleistung}

Der Kompressor benötigt, wenn ein Gas vom Volumen $V_a$ auf das Volumen $V_b$ verringert wird,
die Arbeit:

\begin{equation}
\label{eq:mechanische_arbeit}
A_m=-\int_{V_a}^{V_b}p\mathup{d}V
\end{equation}
Hier sei $p$ der Druck.

Gehen wir davon aus, dass der Kompressor adiabatisch arbeitet,
also keine Wärme an die Umgebung abgibt.
So kann mithilfe der Poisson Gleichung

\begin{equation*}
p_aV^{\kappa}_a=p_bV^{\kappa}_b=pV^{\kappa} \quad \kappa>1
\end{equation*}

der Zusammenhang \eqref{eq:mechanische_arbeit} umgeschrieben werden.
Dabei gibt $\kappa$ das Verhältnis der Molwärme $C_p$ und $C_V$ an.
Es folgt

\begin{equation*}
A_m=-p_aV_a^{\kappa}\int_{V_a}^{V_b}V^{-\kappa}\mathup{d}V=\frac{1}{\kappa-1}\left(p_b\left(\frac{p_a}{p_b}\right)^{\frac{1}{k}}-p_a\right)V_a
\end{equation*}

und damit direkt für die Kompressorleistung

\begin{equation}
N_{mech}=\frac{\mathup{d}A_m}{\mathup{d}t}=\frac{1}{\kappa-1}\left(p_b\left(\frac{p_a}{p_b}\right)^{\frac{1}{\kappa}}-p_a\right)\frac{1}{\rho}\frac{\mathup{d}m}{\mathup{d}t}
\label{eq: Nmech}
\end{equation}

Hierbei ist $\rho$ die Dichte des Transportmediums im gasförmigen Zustand, also beim Druck $p_a$.
