\section{Diskussion}
Abschließend sollen an dieser Stelle die gewonnenen Ergebnisse der Auswertung interpretiert und
in Verbindung mit den in der Theorie aufgeführten Zusammenhängen gesetzt werden.\\
Sowohl die Linsengleichung, als auch das Abbildungsgesetz konnten in guter Überinstimmung
mit der Theorie überprüft werden. Die prozentualen Abweichungen zwischen $V_1 = \frac{b}{g}$ und $V_2 = \frac{B}{G}$ %przentuale groß, oder in der Auswertung auch klein
in Tabelle \ref{tab: methode_1} ebenso wie die Abweichung zwischen der aus dem Graphen \ref{fig: methode_1} abgelesenen Brennweite $f\ua{1, exp} = \SI{+9.7(1)}{\centi\meter}$ und mit \eqref{eq. linsengleichung} %eqref
berechneten Brennweite $f\ua{1, mid} = \SI{+9.68(1)}{\centi\meter}$ sehr gering. Der bestimmte Wert für $f$ weicht lediglich um etwa $\SI{3}{\milli\meter}$ von dem vom Hersteller
angegebenen Wert $f = \SI{10}{\centi\meter}$ ab. Daher kann der Bestimmung der Brennweite einer unbekannten Linse $  f\ua{u, exp} = \SI{13.9(1)}{\centi\meter}$
ein hohes Vertrauen zugeordnet werden. Hierbei ist jedoch zu erwähnen, dass der Wasserdruck und damit die Brechungseigenschaft der %Vertrauen klingt blöd
verwendeten Linse nicht exakt konstant gehalten werden konnte. \\ %-exakt
Die mit der Methode nach Bessel bestimmte Brennweite $f\ua{2} = \SI{+9.90(1)}{\centi\meter}$ liefert einen Wert, der um etwa $\SI{1}{\milli\meter}$ vom Erwartungswert abweicht. Die Methode
liefert darüber hinaus eine genaue Untersuchung der chromatischen Abberation. Die Brennweite für blaues Licht $f\ua{b} = \SI{+9.88(2)}{\centi\meter}$ ist um etwa
$\SI{1}{\milli\meter}$ kleiner als jene für rotes Licht $f\ua{r} = \SI{+9.98(2)}{\centi\meter}$. Dies bestätigt die Erwartung, dass Licht großerer Frequenz im Bereich %größerer
normaler Dispersion stärker gebrochen wird.\\
Die nach der Methode von Abbe aufgenommenen Brennweiten einer Linsenanordnung [$f\ua{a, 1} = \SI{+16.2(3)}{\centi\meter}$, $f\ua{a, 2} = \SI{+15.5(6)}{\centi\meter}$] stellen eine plausible Bestätigung der Formel
\eqref{eq: gleichung_linsensystem} dar [$f\ua{a, t} = \SI{+16.7(3)}{\centi\meter}$]. Die Messdaten in \ref{fig: abbe_g} und \ref{fig: abbe_b} zeigen den erwarteten linearen Verlauf.
Die Methode liefert somit eine gute Möglichkeit zur Bestimmung der Lage von Hauptebenen und Brennweiten eines Linsensystems.\\
Die auftretenden Abweichungen sind hauptsächlich durch ein ungenaues Erkennen der Stellen, die ein scharfes Bild erzeugen, zu erklären.
Insbesondere in Fällen kleiner Bildgröße $B$ kann nur sehr unpräzise die richtige Anordnung bestimmt werden. Die angenommenen Ableseungenauigkeiten
spielen nur eine untergeordnete Rolle, wie sich an den kleinen Fehlern etwa der Vergrößerungen $V$ [siehe etwa Tabelle \ref{tab: methode_1}] zeigt. % \\
Insgesamt ermöglicht der Aufbau jedoch die Aufnahme sehr guter Resultate, die nur geringfügig von den theoretisch erwarteten Werten abweichen. %Aufnahme von Resultaten klingt nicht so gut
