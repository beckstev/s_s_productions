\section{Diskussion}
Im Folgenden sollen die Messergebnisse in Bezug auf die Messgenauigkeit des
Vesuchsaufbaus diskutiert werden.

Zu einer signifikaten Ungenauigkeit in dem Versuchsaufbau sorgt der verwendete
X-Y Schreiber, weil ein Teil der fein Justierung defekt war. Mussten teilweise %$XY$, Satzzeichen
ungenauere Skalierungen verwendet werden. Dadurch wurde zum Beispiel die Aufnahme
der Frank-Hertz-Kurve beeinflusst, da in dieser das erste Maxima auf Grund der groben Skalierung
nicht eindeutig erkennbar ist. %Formulierung

Neben dem X-Y Schreiber sind die Spannungsquelle für $U\ua{A}$ und $U\ua{B}$ %s.o., Plural, singular beachten
eine Fehlerquelle. Die Spannungsquelle verwendet eine analoge Anzeige, hierdurch war das genaue Ablesen
der Spannung erschwert. Eine digitale Vorrichtung verbessert kann die Messergebnisse verbessern. %wort zu viel
Das Problem mit der analogen Anzeige, wirkt sich insbesondere bei der Messung %kein komma
der Energieverteilung negativ aus. Dort war eine Spannung von $U\ua{A}=\SI{11}{\volt}$
verlangt, jedoch reicht die analoge Skala nur bis $\SI{10}{\volt}$, hier musste
die $\SI{11}{\volt}$ somit abgeschätzt werden. %Satz
Hinzu kommt die beschädigte Isolierung des Kabels für das Picoampermeter.
Durch die Beschädigung ist das Messgerät sehr anfällig gegen Induktionsströme
z.\,B. verursacht durch die Heizung.
Die Induktionsströme kann auch der Grund für den unphysikalischen Spannugnsverlauf in Abbildung
\ref{fig: messkurve_ioni} sein.%Plural Singular beachten

Als letzte Fehlerquelle ist die für den Versuch verwendete Heizung
zu erwähnen. Da diese keine Funktion besaß, um die Temperatur konstant zu halten, war
es bei den Messungen kaum möglich eine gleichbleibende Temperatur zu gewährleisten.

 \begin{table}
   \centering
   \caption{Vergleich der Messergbnisse mit Literaturwerten}
   \label{tab: results}
   \begin{tabular} {S S[table-format=1.2] @{${}\pm{}$} S[table-format=1.2] S}
     \toprule
     {Messgröße} & \multicolumn{2}{c}{Exp. Wert}& {Literaturwert} \\
     \midrule
     $\text{E}_{\mathrm{anr}} \,/ \, \si{\eV}\, \, \text{\cite{anreg}}$  &  4.65 & 0.12  & 4.9\\
     $\text{U}_{\mathrm{ion}} \, / \si{\volt } \, \, \text{\cite{ioni}}$ & 9.74 & 0.14 &  10.4\\
     \bottomrule
   \end{tabular}
 \end{table}

Beim Vergleich der Messergebnisse mit der Literatur (vgl. Tabelle  \ref{tab: results})
zeigt sich, dass der Versuch trotz der oben aufgeführten Mängel gute Resultate liefert. %vielleicht noch Abweichungen berechnen
