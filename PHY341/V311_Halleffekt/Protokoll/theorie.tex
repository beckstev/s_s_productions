\section*{Zielsetzung}
Der Versuch möchte den Hall-Effekt durch Mesung der 
Hall-Spannung experiementell bestätigen.

\section{Theorie}

\subsection{Elektrische Leitfähigkeit}
Jedes Atom besitzt nach der Quantenmechanik und dem Pauli-Prinzip 
Energiebände.
Diese geben die möglichen Energiebereiche an die ein gebundenes Elektron 
erreichen kann. Auf den Bändern kann das Elektron weder 
Energie aufnehmen noch abgeben. Es ist dem Elektron also 
nicht gestattet zu beschleunigen bzw. abzubremsen (z.B. durch ein äußeres E-Feld).
Lediglich bei leitenden Elemente (z.B. Metalle) gibt es ein sogenanntes \emph{Leitfähigkeitsband}.
Auf diesem besitzen die Elektronen (\emph{Leitungselektronen}) die Möglichkeit zu Beschleunigen oder abzubremsen.
Bei Isolatoren befinden sich keine freien Elektronen auf dem \emph{Leitfähigkeitsband}.
In Metallen kommt es zu Bildung von Gitterstrukturen. 
Auf diesen verhalten sich die \emph{Leitungselektronen} wie Teilchen eines idealen Gases.
Das bedeuet es kommt untereinander zu Stößen.
Die gemittelte Zeit zwischen zwei Stößen wird als \emph{mittlere Flugzeit} $\overline{\tau}$ bezeichnet.
Wird ein Elektorn mittels $\vec{F}=\map{e}\vec{E}$ ($\map{e}\, \hat{=}$ Elementarladung) beschleunigt 
ergibt sich für \emph{Driftgeschwindigkeit}:

\begin{equation}
\label{eq:drift_v}
\vec{\overline{v}}\ua{d}=\frac{1}{2}\Delta\vec{v}=-\frac{1}{2}\frac{\map{e}}{\map{m}\ua{e}}\vec{E}\overline{\tau}
\end{equation}

Hierbei sei $\map{m}\ua{e}$ die Elektronenmasse.
Mittels der \emph{Driftgeschwindigkeit} kann nun auf die Stromdichte $j$ geschlossen werden. Dazu:

\begin{equation}
\label{eq:stromdicht}
j=-n\vec{\overline{v}}\ua{d}\map{e}=\frac{1}{2}\frac{\map{e^2}}{\map{m}\ua{e}}n\vec{E}\overline{\tau}
\end{equation}
Die Größe $n$ gibt die Elektronen pro Volumeneinheit an.
Mit der idealisierte Annahme das ein homogener Leiter betrachtet wird, kann 
$j=\frac{I}{Q}$ und $E=\frac{U}{L}$ umgeschrieben werden.
Dabei sei $Q$ der Querschnitt und $L$ die Länge des Leiters.
Mittels dem Ohmschen Gesetz folgt ansschließend

\begin{equation}
\label{eq:wider}
R=2\frac{\map{m}\ua{e}}{\map{e}^2}\frac{1}{n\overline{\tau}}\frac{L}{Q}
\end{equation}
\subsection{Der Hall-Effekt}
Der \emph{Hall-Effekt} kommt immer dann zu stande, wenn 
ein Magnetfeld senkrecht auf einer stromdruchflossenden 
homogen Leiterplatte (Dicke $d$ und Breite $b$) steht.
Die Lorentzkraft $\vec{F}\ua{l}$ sorgt für eine Ablenkung der Elektronen.
Die Ablenkung bewirkt eine Potentialdifferenz zwischen den 
Punkten $A$ und $B$ (vgl. \ref{}). Dadruch entsteht gleichzeitig ein
weiteres elektrisches Feld $E\ua{y}$, es wirkt der Elektronenbewegung entgegen.
Die Spannung zwischen $A$ und $B$ wächst solange an, bis es zu einem Kräftgleichgewicht zwischen der Lorentz- und Coulombkraft kommt.
Die gemessene Hallspannung ist  dann gleich

\begin{equation}
\label{eq:hall_span}
U\ua{H})E\ua{y}b=\overline{v}\ua{d}Bd=-\frac{1}{n\map{e}}\frac{BI\ua{q}}{d}
\end{equation}

Die Hallspannung kann also Auskunft über die Ladungsträgerdichte $n$ geben.

\subsection{Berechnung mikroskopischer Leitfähigkeitsparameter}
Mithilfe des Widerstandes und der \emph{Hall-Spannung} kann nun auf 
einige mikroskopischer Leitfähigkeitsparameter geschlossen werden.
Zum einen lässt sich die \emph{freie Weglänge} $\overline{l}$ bestimmen.
Dazu wird der Zusammenhang

\begin{equation}
\ref{eq:freie_weg}
\overline{l}=\overline{\tau}\left|\,v\,\right|
\end{equation}
verwendet.
Es ist $\left|\,v\,\right|$ die Totalgeschwindigkeit des Elektrons.
Sie entsteht durch die Wärmebewegung der Kristallbausteine.
Um die Totalgeschwindigkeit eines Elektrons zu bestimmen nutzt 
man die \emph{Fermie-Energie} $E\ua{F}$. Sie nimmt den Wert des energiereichsten Elektronen wegen des 
Pauli-Verbotes am absoluten Nullpunkt an.
Und wir mit dem Zusammenhang 

\begin{equation}
\label{eq:fermi_e}
E\ua{f}=\frac{h^2}{2\map{m}\ua{e}}\left[\left(\frac{3}{8\pi}\right)^2\right]^{\frac{1}{3}}
\end{equation}
berechnet. In Formel \eqref{eq:fermi_e} wird das Plancksches Wirkungsquantum $h$ benutzt.
Da kaum ein Elektron $E\ua{f}$ erreicht und somit im wesentlichen nicht 
für die elektrische Leitfähigkeit verantwortlich ist.
Kann die Totalgeschwindigkeit abgeschätz werden mit

\begin{equation}
\label{eq:total_v}
\left|\,\overline{v}\,\right|\approx\left(\frac{2E\ua{F}}{\map{m}\ua{e}}\right)^{\frac{1}{2}}
\end{equation}
\subsection{Histerese}
