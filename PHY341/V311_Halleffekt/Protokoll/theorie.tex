\section*{Zielsetzung}
Der Versuch möchte den Hall-Effekt durch Mesung der 
Hall-Spannung experiementell bestätigen.

\section{Theorie}

\subsection{Elektrische Leitfähigkeit}
Jedes Atom besitzt nach der Quantenmechanik und dem Pauli-Prinzip 
Energiebände.
Diese geben die möglichen Energiebereiche an die ein gebundenes Elektron 
erreichen kann. Auf den Bändern kann das Elektron weder 
Energie aufnehmen noch abgeben. Es ist dem Elektron also 
nicht gestattet zu beschleunigen bzw. abzubremsen (z.B. durch ein äußeres E-Feld).
Lediglich bei leitenden Elemente (z.B. Metalle) gibt es ein sogenanntes \emph{Leitfähigkeitsband}.
Auf diesem besitzen die Elektronen (\emph{Leitungselektronen}) die Möglichkeit zu Beschleunigen oder abzubremsen.
Bei Isolatoren befinden sich keine freien Elektronen auf dem \emph{Leitfähigkeitsband}.
In Metallen kommt es zu Bildung von Gitterstrukturen. 
Auf diesen verhalten sich die \emph{Leitungselektronen} wie Teilchen eines idealen Gases.
Das bedeuet es kommt untereinander zu Stößen.
Die gemittelte Zeit zwischen zwei Stößen wird als \emph{mittlere Flugzeit} $\overline{\tau}$ bezeichnet.
Wird ein Elektorn mittels $\vec{F}=\map{e}\vec{E}$ ($\map{e}\, \hat{=}$ Elementarladung) beschleunigt 
ergibt sich für \emph{Driftgeschwindigkeit}:

\begin{equation}
\label{eq:drift_v}
\vec{\overline{v}}\ua{d}=\frac{1}{2}\Delta\vec{V}=-\frac{1}{2}\frac{\map{e}}{\map{m}\ua{e}}\vec{E}\overline{\tau}
\end{equation}

Hierbei sei $\map{m}\ua{e}$ die Elektronenmasse.
Mittels der \emph{Driftgeschwindigkeit} kann nun auf die Stromdichte $j$ geschlossen werden. Dazu:

\begin{equation}
\label{eq:stromdicht}
j=-n\vec{\overline{v}}\ua{d}\map{e}=\frac{1}{2}\frac{\map{e^2}}{\map{m}\ua{e}}n\vec{E}\overline{\tau}
\end{equation}

\subsection{Histerese}
