\section{Versuchsaufbau/-durchführung}
In dem Versuch sollen zwei Proben (Kupfer und Zinn) auf den \emph{Hall-Effekt} untersucht werden.

\subsection{Hysteresemessung}
Die Hysteresemessung dient dazu die Abhängigkeit der Magnetfeldstärke 
zur Stromstärke zu bestimmen. Gleichzeitig aber auch den Hystereseeffekt 
zu bestimmen. Der Versuch wird wie in \ref{} zu sehen aufgebaut.
Mittels eines Teslameters wird nun für zehn Stromstärken die Magnetfeld bestimmt.
Die Messungen ersten $10$ Messungen werden beim hochfahren des Magneten getätigt.
Die anderen zehn beim herunterfahren.

\begin{center}
Die im Folgenden beschriebenden Messungen, werden immer zwei Mal durchgeführt.
Mit dem Unterschied, dass beim zweiten Durchgang die Spannnung umgepolt wird.
Dies dient zur Vermeidung von systematischen Fehlern (z.B. Störspannung).
\end{center}

\subsection{Bestimmung des Widerstands}
Zur Wiederstandsbestimmung wird ein Voltmeter und 
eine Spannungsquelle benötigt.
Nachdem beide an die Probe angeschlossen wurden. 
Wird die Spannungsquelle eingeschaltet und die Stromstärke variiert. 
Für jede Probe werden zu zehn vetrschiedenen Stromstärken 
die Spannung am Voltmeter notiert.
Mithilfe des Ohmschen Gesetzes kann dann auf den Widerstand geschlossen werden.

\subsection{Messung der Hallspannung}
Der Grundlegende Versuchsaufbau ist in Abbildung \ref{} dargestellt.
Zunächst werden die Abmessungen (Breite, Höhe und Länge) der Probe gemessen.
Anschließend folgt die Verkabelung nach dem in \ref{} abgebildeten Schaltbild.
Die verkabelde Probe wird nun in eine Haltevorrichtung gegeben. Diese sorgt dadfür das 
der Strommfluss durch die Probe orthogonal zum Magnetfeld steht.
Für die Messung der \emph{Hall-Spannung} gibt es nun zwei Verfahren. 
Zum einen kann die Stromstärke die an der Probe anliegt konstant gelassen werden und
die Magnetfeldstärke variiert werden. Zum Andern kann 
aber auch die Magnetfeldstärke konstant gelassen werden und die Stromstärke verändert
werden. Für beide Verfahren sollten zehn Messpunkte gewählt werden.
Bei der Regelung des Elektromagneten ist darauf zu achten, dass der 
erzeugende Strom langsam herunter gefahren wird.
