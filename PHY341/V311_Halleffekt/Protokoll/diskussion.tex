\section{Diskussion}
Im Folgenden sollen die Ergebnisse der Messungen interpretiert und in Beziehung zur Präzision des verwendeten Aufbaus gestellt werden. \\
Zunächst zur Messung der Maße der verwendeten Proben. Hier war eine präzise Messung aufgrund der deformierten Proben nur sehr schwer möglich.
Da die Proben fest auf den Kunststoffplatten befestigt sind, konnte insbesondere die Dicke der Zinkprobe mit dem Messschieber nur sehr ungenau aufgenommen werden.
Ungenauigkeiten an dieser Stelle des Versuches haben immense Auswirkungen auf die Bestimmung der mikroskopischen Größen. Etwa die Teilchenzahl pro Volumen
ist direkt Abhängig von der Dicke $d$ (siehe \eqref{eq: }). In der Literatur \cite{} findet man für den spezifischen Widerstand von Kupfer den Wert $a$. Der Vergleich %cite, a?
mit dem gefundenen Wert \eqref{eq: } (Abweichung im Mittel ) bestätigt exemplarisch die nur sehr ungenaue Bestimmung der mikroskopischen Größen.%Abweichung im Mittel
Darüber hinaus bildet die Berechnung der Magnetfeldstärke aus dem Anfangs ermittleten Zusammenhang zum Spulenstrom \eqref{eq: } eine weiter Quelle systematischer
Fehler. Das Netzgerät zeigte im Laufe der Messung sehr häufig technische Probleme auf, was auf eine Überhitzung zurück zu führen ist. \\
Davon abgesehen ermöglicht der Aufbau eine relativ genaue Bestimmung der Hallspannung, wie sich besonders bei der Versuchsereihe mit der Kupferprobe zeigt. Die Annäherung
an einen linearen Zusammenhang zwischen Querstromstärke und Hallspannung \eqref{eq: } gelingt hier mit großer Präzision. Weiterhin zeigt sich, dass die jeweiligen
Ergebnisse aus der Messung mit konstantem Spulenstrom bzw. konstantem Querstrom nur geringfügig voneinander abweichen. Dies zeigt, dass der Aufbau im Rahmen der erwähnten
Ungenauigkeiten relativ aussagekräftige Ergenisse liefert.\\
Ausgehend von der Annahme, dass Kupfer ein Elektronenleiter ist,
kann aufgrund des Vorzeichens der Hallspannung geschlossen werden, dass es sich bei Zink um einen Ionenleiter handelt. \\
