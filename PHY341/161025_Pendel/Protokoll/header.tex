\documentclass[parskip=half, bibliography=totoc, captions=tableheading]{scrartcl} %Ich habe [parskip=half] hinzugefügt

%\usepackage[calc]{picture}
\usepackage{fixltx2e}

\usepackage{tcolorbox}

%\pagestyle{headings}
\usepackage{scrpage2}
\pagestyle{scrheadings}
\ifoot[\pagemark]{\pagemark}
\ofoot[]{}
\cfoot[]{}

\usepackage{polyglossia}
\setmainlanguage{german}
\usepackage{caption}
\usepackage{amsmath}
\usepackage{amssymb}
\usepackage{mathtools}

\usepackage{fontspec}
\defaultfontfeatures{Ligatures=TeX}

\usepackage[
  math-style=ISO,
  bold-style=ISO,
  sans-style=italic,
  nabla=upright,
]{unicode-math}

\setmathfont{Latin Modern Math}
%\setmathfont[range={\mathscr, \mathbfscr}]{XITS Math}
%\setmathfont[range=\coloneq]{XITS Math}
%\setmathfont[range=\propto]{XITS Math}

\usepackage[autostyle]{csquotes}

\usepackage[
  locale=DE,                   % deutsche Einstellungen
  separate-uncertainty=true,   % Immer Fehler mit \pm
  per-mode=symbol-or-fraction, % m/s im Text, sonst Brüche
]{siunitx}
%\sisetup{math-stylemicro=\text{µ},text-micro=µ}

\usepackage{xfrac}

\usepackage[section, below]{placeins}
\usepackage[
  labelfont=bf,
  font=small,
  width=0.9\textwidth,
]{caption}

\usepackage{subcaption}

\usepackage{graphicx}

\usepackage{float}
\floatplacement{figure}{h}
\floatplacement{table}{h}

\usepackage{booktabs}

\usepackage{biblatex}
\addbibresource{lit.bib}

\usepackage{bookmark}

\usepackage[shortcuts]{extdash}

\usepackage[math]{blindtext}

\usepackage{microtype}



\usepackage{hyperref}

\usepackage{color} % Das ist Geschmacksfrage

\usepackage{makeidx} %Ich habe makeidx hinzugefügt + makeindex
\makeindex

\newcommand{\tens}[1]{\underline{\underline{#1}}} %Für Tensor mit der Ordnung zwei
\newcommand{\map}[1]{\mathup{#1}} %Befehl für mathup

\usepackage[version=3]{mhchem} % für Thermodynamik-chemische Elemente
\usepackage{enumitem} %Ich habe enumitem hinzugefügt



%\usepackage{showframe}

\author{Steven Becker und Stefan Grisad}

\title{Versuchsname}

\date{\today\\WS 2016/2017}

\newcommand{\ud}{\mathup{d}}
\newcommand{\del}{\partial}
