\section{Theorie}

Bei gekoppelten Pendel, wir die Theorie eines einzelnen Pendels benötigt.
Dieses soll im nächsten Unterkapitel besprochen werden.

\subsection{Fadenpendel}

Ein Pendel an sich zeichnet sich dadurch aus, das eine Masse $m$ an ein Seil der Länge $l$ befestigt ist.
Dabei wirkt auf die Masse die Kraft:

 \begin{equation*}
 m\ddot{x}=-mg\sin(\varphi)
\end{equation*}

Setzen wir nun die Reihenentwicklung für den Sinus
bis zu ersten Ordnung ein: $\sin(x)\approx x$.
Außerdm nutzen wir noch den Zusammenhang 
für die Bogenlänge $x=l\phi$.
Damit ergibt sich:

\begin{equation*}
\ddot{\varphi}=-\frac{g}{l}\varphi
\end{equation*}

Damit erhalten wir eine Differntialgleichung zweiter Ordnung.
Die Lösung dieser ist eine Schwingung mit der Frequenz 
$\omega=\sqrt{\frac{g}{l}}$.

\subsection{Gekopplete Schwingung}

Werden zwei Fadenpendel miteinander, zum Beispiel durch eine Feder mit der Federkonstante $k$, 
gekopplelt. So erhält man nicht nur noch eine Differentialgleichung, sondern eine Differntialgleichsungssystem der Form

\begin{align*}
\ddot{\varphi}_1+\frac{g}{l}\varphi_1-k(\varphi_1-\varphi_2)&=0\\
\ddot{\varphi}_2+\frac{g}{l}\varphi_2-k(\varphi_2-\varphi_1)&=0
\end{align*}

Die beiden Terme mit der Federkonstante $k$ beschreiben, in
diesem Zusammenhang den Kopplungsterm der beiden Systeme.
Durch geeignete Transformation kann man das Differntialgleichungssystem in ein Eigenwertproblem umwandeln.
Somit können die möglichen Eigenfrequenzen $\omega_1$ und $\omega_2$ des Sytems bestimmt werden.Außerdem erhält man auch noch die möglichen Schingungstypen.

\subsubsection{Gleichsinnige Schwingung}

Bei einer Gleichsinnigen Schwingung werden beiden 
Pendel zu Anfang in die gleiche Richtung ausgelenkt.
Dadurch schwingen beide Massen in die gleichen Richtung und verhalten sich so wie ein einziges Federpendel mit der Frequenz

\begin{equation*}
\omega_+=\sqrt{\frac{g}{l}}
\end{equation*}

Mit $T=\frac{2\pi}{\omega}$ folgt dann für die Periodenfrequenz:

\begin{equation*}
T_+=2\pi\sqrt{\frac{l}{g}}
\end{equation*}

\subsubsection{Gegensinnige Schwingung}
Bei diesem Schwingungstyp des Systems werden beide Massen, zum Anfang, entgegengsetzt ausglenkt. Das sich nun beide Massen aufsich zu bewegen, übt die Feder noch eine Kraft auf beide Pendel.
Dadurch ergibt sich einer veränderte Frequenz

\begin{equation*}
\omega_-=\sqrt{\frac{g}{l}+\frac{2K}{l}}
\end{equation*}

mit der veränderten Schwingungsdauer

\begin{equation*}
T_=2\pi\sqrt{\frac{l}{g+2K}}
\end{equation*}

Dabei sei $K$ die Kopplungskonstante der Feder.

\subsubsection{Gekoppelte Schwingung}

Bei diesem Schwingungstype wird nur eines der beiden Pendel 
am Anfang ausgelenkt, während das andere in der Ruheposition 
verharrt. Nachdem das ausgelenkte Pendel losgelassen wird,
überträgt dieses Energie auf das Ruhende. Dadurch beginnt diesese an 
zu schwingen, während das erste Pendel immer mehr zur Ruhe kommt.
Neben der Energieerhaltung ist bei diesem Versuch auch eine Schwebung beobachtbar. Die Schwebungsdauer ist die Zeit zwischen
zwei Stillständen eines Pendls.
Es ergibt sich für die Schwebungsdauer $T_s$

\begin{equation*}
T_s=\frac{T_{+} T_{-}}{T_{+}-T_{-}}
\end{equation*}

,sowie die Frequenz

\begin{equation*}
\omega_s=\omega_{+}-\omega_{-}
\end{equation*}

Die Kooplungskonstante ist festgelegt durch

\begin{equation*}
K=\frac{\omega^2_{-} -\omega^2_{+}}{\omega^2_{-} +\omega^2_{+}}
=-\frac{T^2_{+}T^2_{-}}{T^2_{+}+T^2_{-}}.
\end{equation*}



