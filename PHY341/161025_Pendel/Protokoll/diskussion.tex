\section{Diskussion}
Im Folgenden sollen die gewonnen Ergebnisse aus dem voran gegagangen Abschnitt, mit Hinblick auf die Präzision
des verwendeten Versuchsaufbaus, auf ihre Plausibilität hin überprüft werden.\\
Zunächst zur Zeitmessung. Als maßgebliche Fehlerquelle bei der Messung der einzelnen Schwingungsfrequenzen ist die
fehlende Präzision beim Starten und Stoppen der Zeitmessung zu nennen. Dies schlägt sich
insbesondere in jenen Messungen ab, bei denen Loslassen der Pendel bzw. Starten der Zeit
von zweich Personen durchgeführt wurde. Bei der Messung von $T_{S}$ kommt hinzu, dass
der Zeitpunkt des Stillstandes kaum exakt erkannt werden kann. Durch das Mitteln über
5 Schwingungsvorgänge fallen diese Ungenauigkeiten jedoch kaum ins Gewicht, was sich an
den kleinen absoluten Fehlern in den Tabellen 1-4 widerspiegelt. \\
Die beiden Kopplungsgrade aus den beiden verschiedenen Pendellängen weichen deutlich
voneinnder ab. Dies erscheint plausibel, wenn man die äquivalente Darstellung von xy
betrachtet.
\begin{equation*}
  \Kappa = \frac{D_F}{D_p + D_F}
\end{equation*}
Die Werte für $D_F$ weichen bei beiden Messungen stark dadurch voneinander ab, dass
zum einen die Pendellänge varriert wurde und zum anderen die Feder auf unterschiedliche
Höhen eingehangen wurde. Hierdurch wirken andere Drehmomente auf die Massen, was die
Ergebnisse xz erklärt. \\
In Abschnitt 3 wurden die Schwebungsfrequenzen $\omega_{S}$ berechnet. Diese sollen nun
abschließend qualitativ mit den gemessenen Werten qualitativ verglichen werden. Der berechnete
Wert der Schwebungsfrequenz im ersten Fall ($l = 0.7 m$) weicht um $+ 10 \%$ vom gemessenen Wert
ab, im zweiten Fall ($l = 0.6 m$) um $-12\%$. Im Rahmen der bereits erwähnten Messspräzision
stimmen die aus der Messung gewonnenen Erkenntnisse mit der Theorie \ref{eq: bla}
