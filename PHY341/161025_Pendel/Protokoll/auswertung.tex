\section{Auswertung}

\subsection{Messung der Schwingungsdauern}
Alle Messungen wurden für zwei verschiedene Pendellängen ($l = (0.70 \pm 0.01)m$ und $l = (0.60 \pm 0.01)m$) durchgeführt, die identische Auswertung soll an dieser Stelle
parallel geschehen. \\
Für die beiden Pendel wurden zunächst die Zeiten $5 \cdot T_1$ und $5 \cdot T_2$ für jeweils 5 Schwingungen aufgezeichnet. Die jeweiligen Mittelwerte
berechnen sich gemäß:
\begin{equation}
\bar{T}_{j}=\frac{1}{n}\sum_{i=1}t_i
\end{equation}
Selbiges wurde für die gleichsinnige($T_{+}$), gegesinnige ($T_{-}$) und gekoppelte ($T_{G}$) Schwingung durchgeführt. Zur direkten Messung der Schwebungszeit $T_{S}$ wurde lediglich
die Zeit für einen vollen Schwebungsvorgang gemessen. Die Berechnung der Standardabweichung des Mittelwerts erfolgt in allen Fällen nach der Formel:
\begin{equation}
\bar{\sigma}_{\bar{T_j}}=\sqrt{\frac{1}{n(n-1)}\sum_{i=1}^{n}(t_i-\bar{t})^2}
\end{equation}
Die entsprechenden Ergebnisse sind in Tabelle \ref{tab:mids_T_70} bzw. \ref{tab:mids_T_60}  aufgetragen. Durch den Zusammenhang $\omega = 2\pi / T$ ergeben
sich unmittelbar die dazugehörigen Frequenzen, welche in den Tabellen \ref{tab:mids_omega_70} und \ref{tab:mids_omega_60} aufgeführt sind.\\

\begin{minipage}{\textwidth}
\begin{table}{0.49	extwidth} 
 \centering 
 \begin{tabular}{S S S S S S } 
 \toprule \\ 
$T_{1}$  & $T_{2}$  & $T_{+}$  & $T_{-}$  & $T$  & $T_{S}$ \\ 
\midrule \\ 
 8.32 \pm 0.02  & 8.29 \pm 0.02  & 8.25 \pm 0.05  & 6.72 \pm 0.03  & 46.23 \pm 1.24  & 7.95 \pm 0.15\\ 
 \bottomrule 
 \end{tabular} 
 \caption{Gemittelte Werte} 
 \label{tab:mids_T} 
  \end{table}
 \hfill
\begin{minipage}{0.49\textwidth}
 \centering
 \begin{tabular}{S S}
 \toprule
$\text{}$ & $\text{Mittelwert}$ \\
 \midrule
 ${5} \cdot T_{1}$ &  7.79 \pm 0.05 \\
  ${5} \cdot T_{2}$ & 7.76 \pm 0.04 \\
${5} \cdot T_{+}$  &  7.63 \pm 0.05 \\
${5} \cdot T_{-}$  & 7.25 \pm 0.05 \\
${5} \cdot T_{K}$  & 133.94 \pm 1.97 \\
$T_{S}$ & 25.48 \pm 0.52\\
 \bottomrule
 \end{tabular}
 \captionof{table}{Gemittelte Werte für $l = 0.6m$}
 \label{tab:mids_T_60}
  \end{minipage}

\end{minipage}

\begin{minipage}{\textwidth}
\begin{minipage}{0.49\textwidth}
 \centering
 \begin{tabular}{S S}
 \toprule
$\text{}$ & $\text{Frequenz / \si{\second^{-1}}}$ \\
 \midrule
$\omega_{+}$  &  3.81 \pm 0.02 \\
$\omega_{-}$  & 4.68 \pm 0.02 \\
$\omega_{K}$  & 0.68 \pm 0.02 \\
$\omega_{S}$ & 0.79 \pm 0.01\\
 \bottomrule
 \end{tabular}
 \captionof{table}{Frequenzen für $l = 0.7m$}
 \label{tab:mids_omega_70}
  \end{minipage}

 \hfill
\begin{minipage}{0.49\textwidth}
 \centering
 \begin{tabular}{S S}
 \toprule
$\text{}$ & $\text{Frequenz / \si{\second^{-1}}}$ \\
 \midrule
$\omega_{+, 0.6}$  &  4.12 \pm 0.03 \\
$\omega_{-, 0.6}$  & 4.34 \pm 0.03 \\
$\omega_{K, 0.6}$  & 0.23 \pm 0.00 \\
$\omega_{S, 0.6}$ & 0.25 \pm 0.00\\
 \bottomrule
 \end{tabular}
 \captionof{table}{Frequenzen für $l = 0.6m$}
 \label{tab:mids_omega_60}
  \end{minipage}

\end{minipage}


Nun soll mit den gefundenen Werten der Kopplungsgrad $\Kappa$ berechnet werden:
\begin{equation}
  \Kappa = \frac{T_{+}^2 - T_{-}^2}{T_{+}^2 + T_{-}^2}
\end{equation}
Hiermit ergibt sich für den Mittelwert $\bar{\Kappa}$ als Funktion der einzelnen Mittelwerte:
\begin{equation}
\bar{\Kappa}_{0.7} = 0.20
\end{equation}
Für die erste Konfiguration ($l = 0.70m$), respektive:
\begin{equation}
\bar{\Kappa}_{0.6} = 0.05
\end{equation}
Für die zweite Konfiguration ($l = 0.60m$). Um die zugehörigen Fehler anzugeben, muss die Gaußsche-Fehlerfortpflanzung verwendet werden:
\begin{equation}
  \dots
\end{equation}
Hierzu müssen zunächst die partiellen Ableitungen nach $T_{+}$ und $T_{-}$ bestimmt werden:
\begin{align}
  \frac{\partial \Kappa}{\partial T_{+}} &= \frac{4 \cdot T_{+} \cdot T_{-}^2}{(T_{+}^2 + T_{-}^2)^2} \\
  \frac{\partial \Kappa}{\partial T_{-}} &= - \frac{4 \cdot T_{-} \cdot T_{+}^2 }{T_{+}^2 + T_{-}^2}
\end{align}
Einsetzen liefert, dass die entsprechenden Werte für den Fehler der Kopplungskonstante in der Größenordnung $10^{-3}$ liegen und somit
vernachlässigt werden können. \\
Abschließend soll nun noch der Zusammenhang
\begin{equation}
 \omega_{S} = |\omega_{+} - \omega_{-}|
 \label{eq: bla}
\end{equation}
überprüft werden. Hierzu werden die Werte für $\omega_{S}$ aus $\omega_{+}$ und $\omega_{-}$ berechnet und mit dem gemessenen Wert verglichen.
Mit den Werten aus \ref{tab: frequenzen_70} ergeben sich die Kreisfrequenzen zu:
\begin{align}
  \omega_{S, 0.7} &= (0.87 \pm 0.03) \si{\second ^{-1}} \\
  \omega_{S, 0.6} &= (0.22 \pm 0.04) \si{\second^{-1}}
\end{align}
Die entsprechenden absoluten Fehler wurden gemäß der üblichen Formel für Summen berechnet:
\begin{equation}
(\Delta f(x_1, \dots ,x_n))^2 = \sum _i^n (\Delta x_i) ^2
\end{equation}



\section{Messungen}


\subsection{Pendellänge $l = 0.70 m$}
\begin{minipage}{\textwidth}
\begin{table}
 \centering
 \caption{Schwingungsdauer linkes Pendel, $l = 0.7 m$}
 \label{tab:T_links_70}
 \begin{tabular}{S }
 \toprule \\
$5 \cdot T_1$ \\
\midrule \\
 8.50 \\
 8.35 \\
 8.32 \\
 8.26 \\
 8.26 \\ 
 8.29 \\
 8.29 \\
 8.30 \\
 8.23 \\
 8.38 \\
 \bottomrule
 \end{tabular}
 \end{table}

 \hfill
\begin{table}
 \centering
 \caption{Schwingungsdauer linkes Pendel, $l = 0.7 m$}
 \label{tab:T_rechts_70}
 \begin{tabular}{S }
 \toprule \\
$5 \cdot T_2$ \\
\midrule \\
 8.13 \\
 8.30 \\
 8.36 \\
 8.29 \\
 8.30 \\
 8.26 \\
 8.44 \\
 8.27 \\ 
 8.26 \\
 8.33 \\
 \bottomrule
 \end{tabular}
 \end{table}

\end{minipage}


\begin{minipage}{\textwidth}
\begin{minipage}[b]{0.49\textwidth}
 \centering
 \begin{tabular}{S S}
 \toprule
\multicolumn{2}{S}{${5} \cdot T_{+} / \si{\second}$} \\
\midrule
 8.12 &  8.33 \\
 8.18 &  8.16 \\
 8.23 &  8.18 \\
 8.55 &  8.10 \\
 8.18 &  8.50 \\
 \bottomrule
 \end{tabular}
 \captionof{table}{Schwingungsdauer gleichsinnig}
 \label{tab:T_gleichsinnig_70}
 \end{minipage}

 \hfill
\begin{table}
 \centering
 \caption{Schwingungsdauer gegensinnig, $l = 0.7 m$}
 \label{tab:T_gegensinnig_70}
 \begin{tabular}{S }
 \toprule \\
$5 \cdot T_{-}$ \\
\midrule \\
 6.89 \\
 6.66 \\
 6.80 \\
 6.55 \\
 6.61 \\ 
 6.76 \\
 6.73 \\
 6.69 \\
 6.84 \\
 6.64 \\
 \bottomrule
 \end{tabular}
 \end{table}

\end{minipage}

\begin{minipage}{\textwidth}
\begin{minipage}[b]{0.49\textwidth}
 \centering
 \begin{tabular}{S S}
 \toprule
 \multicolumn{2}{S}{${5} \cdot T_{K}  / \si{\second}$} \\
\midrule
 48.30 &  41.93 \\
 49.47 &  40.01 \\
 44.23 &  53.81 \\
 48.23 &  44.67 \\
 48.42 &  43.23 \\
 \bottomrule
 \end{tabular}
 \captionof{table}{Schwingungsdauer gekoppelt}
 \label{tab:T_gekoppelt_70}
 \end{minipage}

 \hfill
\begin{table}
 \centering
 \caption{Schwebungsdauer, $l = 0.7 m$}
 \label{tab:T_schwebe_70}
 \begin{tabular}{S }
 \toprule \\
$T_{S}$ \\
\midrule \\
 7.18 \\
 7.10 \\
 7.86 \\
 7.80 \\
 8.11 \\
 8.59 \\
 8.23 \\ 
 8.36 \\
 8.04 \\
 8.21 \\
 \bottomrule
 \end{tabular}
 \end{table}

\end{minipage}


\subsection{Pendellänge $l = 0,6 \si{\meter} $}
\begin{minipage}{\textwidth}
\begin{minipage}{0.49\textwidth}
 \centering
 \begin{tabular}{S }
 \toprule
${5} \cdot T_{1}/ \si{\second}$ \\
\midrule
 7.86 \\
 7.61 \\
 7.76 \\
 7.78 \\
 7.92 \\
 \bottomrule
 \end{tabular}
 \captionof{table}{linkes Pendel}
 \label{tab:T_links_60}
  \end{minipage}

 \hfill
\begin{minipage}{0.49\textwidth}
 \centering
 \begin{tabular}{S }
 \toprule
${5} \cdot T_{2}$ \\
\midrule
 7.86 \\
 7.83 \\
 7.64 \\
 7.80 \\
 7.67 \\
 \bottomrule
 \end{tabular}
 \captionof{table}{Schwingungsdauer rechtes Pendel}
 \label{tab:T_rechts_60}
  \end{minipage}

\end{minipage}


\begin{minipage}{\textwidth}
\begin{minipage}{0.49\textwidth}
 \centering
 \begin{tabular}{S }
 \toprule
${5} \cdot T_{+}/ \si{\second}$ \\
\midrule
 7.47 \\
 7.55 \\
 7.64 \\
 7.75 \\
 7.73 \\
 \bottomrule
 \end{tabular}
 \captionof{table}{Schwingunsdauer gleichsinnig}
 \label{tab:T_gleichsinnig_60}
  \end{minipage}

 \hfill
\begin{minipage}{0.49\textwidth}
 \centering
 \begin{tabular}{S }
 \toprule
${5} \cdot T_{-} / \si{\second}$ \\
\midrule
 7.15 \\
 7.21 \\
 7.38 \\
 7.13 \\
 7.36 \\
 \bottomrule
 \end{tabular}
 \captionof{table}{Schwingungsdauer gegensinnig}
 \label{tab:T_gegensinnig_60}
  \end{minipage}

\end{minipage}

\begin{minipage}{\textwidth}
\begin{minipage}{0.49\textwidth}
 \centering
 \begin{tabular}{S }
 \toprule
${5} \cdot T$ \\
\midrule 
 129.61 \\
 137.27 \\
 129.47 \\
 140.67 \\
 132.66 \\
 \bottomrule
 \end{tabular}
 \captionof{table}{Schwingungsdauer gekoppelt}
 \label{tab:T_gekoppelt_60}
  \end{minipage}

 \hfill
\begin{minipage}{0.49\textwidth}
 \centering
 \begin{tabular}{S }
 \toprule
$T_{S}/ \si{\second}$ \\
\midrule
 26.34 \\
 24.62 \\
 26.21 \\
 23.64 \\
 26.60 \\
 \bottomrule
 \end{tabular}
 \captionof{table}{Schwebungsdauer}
 \label{tab:T_schwebe_60}
  \end{minipage}

\end{minipage}
