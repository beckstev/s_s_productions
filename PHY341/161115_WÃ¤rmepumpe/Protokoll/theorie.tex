\section{Theorie}

\subsection{Prinzip der Wärmepuppe}

In der Thermodynamik fließt die Wärme immer 
vom warmen Medium $T_1$ zum kalten Medium $T_2$.
Möchte man diesen Effekt umkehren, also vom Kalten 
zum Warmen. Kommt die Wärmepumpe zum Einsatz.
Denn durch Zufuhr von Energie (z. B. mechanische Arbeit) sagt der erste 
Hauptsatz der Thermodynamik mittels

\begin{equation}
\label{eq:hst_1}
Q_1=Q_2+A
\end{equation}

,dass die im Warmen aufgenommene Wärmemenge $Q_1$ gleich 
der Summe der aus dem Kalten entnommene Wärmemenge $Q_2$
und der zugeführten Arbeit $A$ ist.

Jede Wärmepumpe besitzt eine sogenannte Güteziffer $\nu$.
Diese gibt das Verhältnis zwischen transportierter Wärmemenge 
und der dazu benötigten Arbeit $A$.
Aus der Hauptsatz ergibt sich für ideale Voraussetzung:

\begin{equation}
\label{eq:best_kennziffer}
\nu=\frac{Q_1}{A}
\end{equation}

Betrachtet man zusätzlich den zweiten Hauptsatz der Thermodynamik.
So ergibt sich ein weiterer Zusammenhang zwischen den Wärmemengen 
$Q_1$ und $Q_2$ und den Temperaturen der Medien $T_1$ und $T_2$.
Denn ändert sich die Temperatur der beiden Medien nicht während der 
Wärmeübertragung. 
So verschwindet die reduzierte Wärmemenge und es folgt

\begin{equation}
\label{eq:hst_2}
\frac{Q_1}{T_1}-\frac{Q_2}{T_2}=0
\end{equation}

Jedoch ist für \eqref{eq:hst_2} eine Voraussetzung,
dass der Prozess reversibel ist.
Das bedeutet die in einem thermodynamischen Prozess aufgenommene Wärme 
und Energie, muss bei Umkehrung des Versuches wieder zurückfließen.
In der Realität ist dies durch Verlustwärme und Reibungsprozesse
nie zu realisieren.
Dadurch stellt sich für die reale Wärmepumpe eine andere 
Günterziffe $\nu_{real}$ ein. 
Sie lässt sich mittels der idealen Güteziffer 
\begin{equation*}
\nu_{id}=\frac{T_1}{T_1-T_2}
\end{equation*}
abschaätzen zu:

\begin{equation*}
\nu_{real}<\frac{T_1}{T_1-T_2}
\end{equation*}

Das bedeutet je geringer die Differenz zwischen $T_1$ und $T_2$, 
desto höher ist die Effizienz der Wärmepumpe.

\subsection{Bestimmung der realen Güteziffer}
Aus einer von $t$ abhängigen Messreihe $T_1$ wird der Differenzenquotient $\frac{\Delta T_1}{\Delta T}$ bestimmt.
Dadurch kann dann die Wärmemenge bestimmt werden:

\begin{equation*}
\frac{\Delta Q_1}{\Delta t}=\left(m_1c_w+m_kc_k\right)\frac{\Delta T_1}{\Delta t}
\end{equation*}

Durch Bestimmung einer Ausgleichsgeraden kann der 
Differenzenquotient  durch einen Differenzialquotient
ersetzt werden.
Dies soll in allen folgenden Rechnungen gelten.

Man erhält:

\begin{equation}
\label{eq:warmemenge_1}
\frac{\mathup{d} Q_1}{\mathup{d} t}=\left(m_1c_w+m_kc_k\right)\frac{\mathup{d} T_1}{\mathup{d} t}
\end{equation}


Wobei $m_1c_w$ die Wärmekapazität des Wassers im Behälter $1$ ist und 
$m_kc_k$ die Wärmekapazität der Kupferschlange und des Eimers sind.
Mit \eqref{eq:best_kennziffer} und \eqref{eq:warmemenge_1} folgt dann

\begin{equation}
\label{eq:bestimmung_ziffer}
\nu_{real}=\frac{\mathup{d}Q_1}{\mathup{d}tN}
\end{equation}

Hierbei sei $N$ die Leistungsaufnahme des Wattmeter im Zeitraum $\mathup{d}$.

\subsection{Bestimmung des Massendurchsatzes}

Bei Betrachtung der Messreihe $T_2$ bildet man den Differenzialquotient $\frac{\mathup{d}T_2}{\mathup{d}t}$.
Damit kann dann, wie oben

\begin{equation*}
\frac{\mathup{d} Q_2}{\mathup{d} t}=\left(m_2c_w+m_kc_k\right)\frac{\mathup{d} T_2}{\mathup{d} t}
\end{equation*}Der

die Wärmemenge, die pro Zeit $\mathup{d}t$ entohmen wird, bestimmen.
Da es sich bei diesem Vorgang, um eine Aggregatzustandsänderung handelt.
Muss die Wärmemenge gleich der Verdampfungswärme $L$ pro Massenzeitstückchen $\frac{\mathup{d}m}{dt}$.
Also:

\begin{equation*}
\frac{\mathup{d} Q_2}{\mathup{d} t}=L\frac{\mathup{d}m}{dt}
\end{equation*}

\subsection{Bestimmung der mechanischen Kompressorleistung}

Der Kompressor benötigt, wenn ein Gas vom Volumen $V_a$ auf das Volumen $V_b$ verringert
die Arbeit:

\begin{equation}
\label{eq:mechanische_arbeit}
A_m=-\int_{V_a}^{V_b}p\mathup{d}V
\end{equation}
Hier sei $p$ der Druck.

Gehen wir davon aus, dass der Kompressor adiabatisch arbeitet,
d. h. ohne Wärme an die Umgebung abzugeben. 
So kann mithilfe der Poisson Gleichung

\begin{equation*}
p_aV^{\kappa}_a=p_bV^{\kappa}_b=pV^{\kappa} \quad \kappa>1
\end{equation*}

die Gleichung \eqref{eq:mechanische_arbeit} umgeschrieben.
$\kappa$ gibt das Verhältnis der Molwärme $C_p$ und $C_V$ a.
Es folgt:

\begin{equation*}
A_m=-p_aV_a^{\kappa}\int_{V_a}^{V_b}V^{-\kappa}\mathup{d}V=\frac{1}{\kappa-1}\left(p_b\left(\frac{p_a}{p_b}\right)^{\frac{1}{k}}-p_a\right)V_a
\end{equation*}

Und damit direkt für die Kompressorleistung

\begin{equation*}
N_{mech}=\frac{\mathup{d}A_m}{\mathup{d}t}=\frac{1}{\kappa-1}\left(p_b\left(\frac{p_a}{p_b}\right)^{\frac{1}{\kappa}}-p_a\right)\frac{1}{\rho}\frac{\mathup{d}m}{\mathup{d}t}
\end{equation*}

Hierbei ist $\rho$ die Dichte des Transportmediums im gasförmigen Zustand, also beim Druck $p_a$.



