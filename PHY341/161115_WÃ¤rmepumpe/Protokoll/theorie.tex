\section{Theorie}

\subsection{Prinzip der Wärmepuppe}

In der Thermodynamik fließt die Wärme immer 
vom warme Medium $T_1$ zum kalte Medium $T_2$.
Möchte man diesen Effekt umkehren, also vom Kalten 
zum Warmen. Kommt die Wärmepumpe zum einsatz.
Denn durch Zufuhr von Energie (z.B. mechanische Arbeit) sagt der erste 
Hauptsatz der Thermodynamik mititels

\begin{equation}
\label{eq:hst_1}
Q_1=Q_2+A
\end{equation}

,dass die im Warmen aufgenommene Wärmemenge $Q_1$ gleich 
der Summe der aus dem Kalten entnommene Wärmemenge $Q_2$
und der zugeführten Arbeit $A$ ist.

Jede Wärmepumpe besitzt eine sogennante Güteziffer $\nu$.
Diese gibt das Verhältnis zwischen transportierten Wärmemenge 
und die dazu benötigte Arbeit $A$.
Aus der Hauptsatz ergibt sich für ideale Voraussetzung:

\begin{equation}
\label{eq:best_kennziffer}
\nu=\frac{Q_1}{A}
\end{equation}

Betrachtet man zusätlich den zweiten Hauptsatz der Thermodynamik.
So ergibt sich ein weiterer Zusammenhang zwischen den Wärmemengen 
$Q_1$ und $Q_2$ und den Temperaturen der Medien $T_1$ und $T_2$.
Denn ändert sich die Temperatur der beiden Medien nicht während der 
Wärmeübertragung. 
So verschwindet die reduzierte Wärmemenge und es folgt

\begin{equation}
\label{eq:hst_2}
\frac{Q_1}{T_1}-\frac{Q_2}{T_2}=0
\end{equation}

Jedoch ist für \eqref{eq:hst_2} eine Vorraussetzung,
das der Prozess reversibel ist.
Das bedeutet die in einem thermodynamischen Prozess aufgenommene Wärme 
und Energie, muss bei Umkehrung des Versuches wieder zurückfließen.
In der Realität ist dies durch Verlustwärme und Reibungsprozesse
nie zu realisisieren.
Dadurch stellt sich für die reale Wärmepumpe eine andere 
Günterziffe $\nu_{real}$ ein. 
Sie lässt sich mittels der idelaen Güteziffer 
\begin{equation*}
\nu_{id}=\frac{T_1}{T_1-T_2}
\end{equation*}
abschaätzen zu:

\begin{equation*}
\nu_{real}<\frac{T_1}{T_1-T_2}
\end{equation*}

Das bedeutet je geringer die Differenz zwischen $T_1$ und $T_2$, 
desto höher ist die Effizienz der Wärmepumpe.

\subsection{Bestimmung der realen Güteziffer}
Aus einer von $t$ abhängigen Messreihe $T_1$ wird der Differezenqoutient $\frac{\Delta T_1}{\Delta T}$ bestimmt.
Dadurch kann dann die Wärmemenge bestimmt werden:

\begin{equation*}
\frac{\Delta Q_1}{\Delta t}=\left(m_1c_w+m_kc_k\right)\frac{\Delta T_1}{\Delta T}
\end{equation*}

Durch Bestimmung einer Ausgleichsgeraden, kann der 
Differenezenqoutient durch ein Differnezialqoutient
ersetzt werden.
Dies soll in allen folgenden Rechnungen gelten.

man erhält:

\begin{equation}
\label{eq:warmemenge_1}
\frac{\mathup{d} Q_1}{\mathup{d} t}=\left(m_1c_w+m_kc_k\right)\frac{\mathup{d} T_1}{\mathup{d} T}
\end{equation}


Wobei $m_1c_w$ die Wärmekapazität des Wassers im Behälter $1$ ist und 
$m_kc_k$ die Wärmekapazität der Kupferschlange und des Eimers sind.
Mit \eqref{eq:best_kennziffer} und \eqref{eq:warmemenge_1} folgt dann

\begin{equation}
\label{eq:bestimmung_ziffer}
\nu_{real}=\frac{\mathup{d}Q_1}{\mathup{d}tN}
\end{equation}

Hierbei sei $N$ die leistungsaufnahme des Wattneter im Zeitraum $\mathup{d}$.



