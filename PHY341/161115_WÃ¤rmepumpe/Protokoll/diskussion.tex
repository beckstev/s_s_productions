\section{Diskussion}
In diesem Abschnitt sollen die gewonnen Erkenntnisse der Auswertung diskutiert werden. Zunächst zur Temperaturmessung.
Hierbei zeigte sich direkt zum Beginn der Messung, dass das Thermometer zur Messung der Temperatur $T_2$ nicht korrekt
funktionierte. Während die Temperatur in Reservoir 1 anstieg, wurde in den ersten 5 Minuten eine nahezu konstante Temperatur
in Reservoir 2 gemessen (siehe \ref{tab: tempdruck}). Dies widerspricht der Physik, weswegen in der nachfolgenden Auswertung
eine Betrachtung des Zeitraumes nach $t_0 = 360\si{\second}$ vorgezogen wurde. Insgesamt ist jedoch durch diese Beobachtung
die Aussagekraft der Temperaturmessung als sehr gering einzustufen. \\
Der Vergleich der realen und idealen Güteziffer (\ref{tab: dTdt}) bestätigt deutlich den Zusammenhang $\nu_{real} < \nu_{ideal}$. Ebenfalls
zu erkennen ist der Effekt, dass die Güteziffer kleiner wird mit dem Anstieg der Temperaturdifferenz. Als Ursachen für die deutlich
geringere reale Güteziffer sind immense Energieverluste des Systems zu nennen. Dies bestätigt sich vorallem in der Tatsache, dass die tatsächlich
aufgebrachte elektrische Leistung (im Mittel $P = 201.9 \si{\watt}$) deutlich höher ist als die genutze mechanische Leistung (etwa
für $t = 720 \si{\second}$: $N_{mech} = 43,84 \si{\watt}$). Zur Berechnung der mechanischen Leistung sei noch anzumerken, dass der errechnete
Fehler (siehe \ref{tab: dmdtNmech}) sehr hoch ausfällt und die somit die Signifikanz dieser Messung als relativ gering zu bewerten ist. Dennoch %ref überprüfen
erlauben die gewonnen Erkenntnisse der Auswertug die Feststellung, dass reale und ideale Effektiviät einer Wärmepumpe stark voneinander abweichen.
