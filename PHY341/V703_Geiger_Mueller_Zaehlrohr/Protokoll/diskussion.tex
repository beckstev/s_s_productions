\section{Diskussion}
Abschhließend sollen an dieser Stelle die gewonnenen Ergebnisse der Auswertung interpretiert und
in Verbindung mit den in der Theorie aufgeführten Zusammenhängen gesetzt werden.\\
Die in Abbildung \ref{fig: zählrate_ges} dargestellte Zählrohrstatistik zeigt den erwarteten Verlauf. Insbesondere
die Betrachtung des Plateaubereichs (Abbildung \ref{fig: plateau}) bestätigt eine annähernd konstante Impulsrate,
deren Steigung von $m = \SI{0.041 \pm 0.009}{\per\second\per\volt}$ auf Nachentladungen zurückzuführen ist.\\
Die Bestimmung der Totzeit ist im Falle beider Methoden als unpräzise einzustufen. Das Ablesen des relevanten
Intervalls auf dem Oszilloskops ist nur grob möglich. Der Einsatz eines digitalen Oszilloskops würde an dieser
Stelle eine größere Signifikanz liefern. Gleiches gilt im übrigen für das Ablesen der Erholungszeit.
Die Berechnung der Totzeit aus den Messungen der Zwei-Quellen-Methode liefert eine experimentelle Größe, deren Bestwert etwa um
$100\%$ von dem Ergebnnis der Oszilloskopmessung abweicht. Darüber hinaus ist der statistische Fehler so groß
[siehe \eqref{}], dass eine präzise Aussage über die Totzeit nicht getroffen werden.\\
Die Graphische Darstellung des Zusammenhangs zwischen Betriebsspannung und ausgelöster Ladungsmenge $\Delta Q$ zeigt
einen linearen Verlauf auf. Auch dies geht aus der Theorie hervor; innerhalb des Auslösebereichs ist die
Ladung am Anodendraht unabhängig von der Primärionisation und damit unter konstantem Zählrohrvolumen nur noch
von der Spannung $U$ abhängig. \\
Die Ergebnisse des Versuchs zeigen, dass das verwendete Geiger-Müller-Zählrohr plausible Aussagen über Strahlungsintensitäten
zulässt. Präzise Messungen, etwa der Totzeit sind hingegen unter den gegebenen Voraussetzungen nicht möglich.
