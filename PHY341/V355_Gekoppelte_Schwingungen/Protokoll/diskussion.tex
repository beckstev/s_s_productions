\section{Diskussion}
Im folgenden Abschnitt soll die Aussagekfaft der Ergebnisse im Bezug auf
die Rahmenbedingungen des Versuches diskutiert werden.
Es sei zunächst die Annahme von fehlerfreien Bauelementen (vgl. \eqref{eq: bauteile})
bemängelt werden. Die Annahme stimmt mit der Realität keines Falls überein. %bemängelt passt nicht
Die Qualität der Ergebnisse wird auf diese Weise verringert.
Hiermit lässt sich begründen, wieso die Abweichung zwischen Theorie und Praxis teilweise
relativ groß ist (vgl. dazu die ersten Werte in den Tabellen \ref{fig:teilb_schwingungen_prak_theo} und \ref{fig:teilc_schwingungen_prak_theo}).
Die in den Tabellen \ref{fig:teilb_schwingungen_prak_theo} und \ref{fig:teilc_schwingungen_prak_theo} gewählten Fehler sind darauf zu
zurückzuführen, dass sich die Frequenz $\nu_+$, bei Variation der Kapazität $C\ua{k}$, nicht ändern sollte.
Es soll mit der Abweichung sichergestellt werden, dass alle $\nu_+$ in derselben Menge liegen. %ungünstig formuliert
Der in Tabelle \ref{fig:teila_n_ck} definierte Fehler, wurde aufgrund des  Oszilloskop eingeführt.
Denn die schlechte Skala bzw. Auflösung des Oszilloskops, erschwert das genaue Ablesen. %Satzbau, mach einen Satz mit HS und NS draus
Die Ungenauigkeit des Oszilloskop, trübt außerdem  die Ergebnisse der anderen Teilversuche. %kein Komma, trübt passt nicht

Dennoch liefert der Versuch eine relativ hohe Übereinstimmung mit den Theoriewerten.
Denn bei Betrachtung von Tab. \ref{fig:teilb_schwingungen_prak_theo} und Tab. \ref{fig:teilc_schwingungen_prak_theo} erkennt man,  %Sätze nicht mit denn anfangen lassen
dass in der betrachteten Signifikanz, kein Unterschied zwischen
Theorie und Experiment feststellbar ist.
Es scheint, als ob eine Erhöhung der Kapazität $C\ua{k}$ eine höhere Genauigkeit des Versuches bewirkt (vgl. Abb. \ref{fig: plot}).
