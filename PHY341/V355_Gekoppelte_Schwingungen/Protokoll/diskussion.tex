\section{Diskussion}
Im folgenden Abschnitt sollen die Aussagekfraft der Ergebnisse im Bezug auf 
die Rahmenbedingungen des Versuches diskutiert werden.
Es sei hierbei zunächste die Annahme von fehlerfreien Bauelementen (vgl. \eqref{eq:bau}) 
bemängelt werden. Die Annahme stimmt mit der Realität keines falls überein.
Hierdurch wird die Qualität der Ergebnise beeinträchtigt.
Hierdurch lässt sich, aber wahrscheinlich erklärenw warum die Abweichung zwischen Theorie und Praxis
relativ groß war (vgl. dazu die Tabellen \ref{fig:teilb_schwingungen_prak_theo} und\ref{fig:teilc_schwingungen_prak_theo}).
Die in den Tabellen \ref{fig:teilb_schwingungen_prak_theo} und \ref{fig:teilc_schwingungen_prak_theo} angenommen Fehler sind darauf zu 
zurückzuführen, dass sich die Frequenz $\nu_+$, bei Änderung der Kapazität $C\ua{k}$, nicht ändern sollte.
Mit der Abweichung soll also sichergestellt werden, dass alle $\nu_+$ in der selben Menge liegen.
Den in Tabelle \ref{fig:teila_n_ck} definierte Fehler, wurde dshalb eingeführt da das Ozilloskop eine weitere Fehlerquelle
wiederspiegelt. Die Skala bzw. die Auflösung des Ozilloskop schränkte, zusätzlich das genaue ablesen, soweit ein das
ein großer Fehler in \ref{fig:teila_n_ck} realistisch scheint.
Die Ungenauigkeit des Ozilloskop, trübt auch die Ergebnisse der andern Teilversuche.

