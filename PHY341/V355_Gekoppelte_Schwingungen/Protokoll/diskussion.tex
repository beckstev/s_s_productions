\section{Diskussion}
Im folgenden Abschnitt soll die Aussagekfraft der Ergebnisse im Bezug auf 
die Rahmenbedingungen des Versuches diskutiert werden.
Es sei zunächste die Annahme von fehlerfreien Bauelementen (vgl. \eqref{eq: bauteile}) 
bemängelt werden. Die Annahme stimmt mit der Realität keines falls überein.
Die Qualität der Ergebnise wird auf diese Weise verringert.
Hiermit läasst sich begründen, wieso die Abweichung zwischen Theorie und Praxis teilweise
relativ groß ist (vgl. dazu die ersten Werte in den Tabellen \ref{fig:teilb_schwingungen_prak_theo} und \ref{fig:teilc_schwingungen_prak_theo}).
Die in den Tabellen \ref{fig:teilb_schwingungen_prak_theo} und \ref{fig:teilc_schwingungen_prak_theo} gewählten Fehler sind darauf zu 
zurückzuführen, dass sich die Frequenz $\nu_+$, bei Variation der Kapazität $C\ua{k}$, nicht ändern sollte.
Es soll mit der Abweichung sichergestellt werden, dass alle $\nu_+$ in der selben Menge liegen.
Der in Tabelle \ref{fig:teila_n_ck} definierte Fehler, wurde auf Grund des  Ozilloskop eingeführt.
Denn die schlechte Skala bzw. Auflösung des Ozilloskops, erschwert das genaue Ablesen. 
Die Ungenauigkeit des Ozilloskop, trübt auch die Ergebnisse der anderen Teilversuche.

Dennoch liefert der Versuch eine relativ hohe Übereinstimmung mit den Theoriewerten.
Denn bei Betrachtung von Tab. \ref{fig:teilb_schwingungen_prak_theo} und Tab. \ref{fig:teilc_schwingungen_prak_theo} erkennt man, 
das in der betrachteten Signifikanz, kein Unterschied zwischen 
Theorie und Experiment festellbar ist. 
Es scheint, als ob eine Erhöhung der Kapazität $C\ua{k}$ eine höhere Genauigkeit des Versuches bewirkt (vgl. Abb. \ref{fig: plot}).


