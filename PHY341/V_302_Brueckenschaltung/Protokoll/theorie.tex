\setcounter{page}{1}
\section*{Zielsetzung}
Im Versuch 302 sollen, unter Verwendung sogenannter Brückenschaltungen, physikalische Größen
in elektrischen Schaltkreisen experimentell ermittelt werden.
\section{Theorie}\label{abs: theo}
Zur Berechnung unbekannter Größen in einem elektrischen Stromkreis werden die Kirchhoffschen Regeln
benötigt. Die erste Kirchhoffsche Regel (Knotenregel) sagt aus, dass die Summe aller zu- und abfließenden
Ströme in einem Verzweigungspunkt verschwindet. Der Zusammenhang ergibt sich unmittelbar aus der Ladungserhaltung.
\begin{equation}
  \sum_k I_k = 0
  \label{eq: knoten}
\end{equation}

Die zweite Kirchhoffsche Regel besagt, dass in einem in sich geschlossenen Stromkreis
die Summe aller Quellspannungen der Summe aller abfallenden
Spannungen enspricht. Die Regel ist ein Resultat aus der Energieerhaltung.
\begin{equation}
  \sum_k U_{\symup{Q},k} = \sum_k I_k \cdot R_k
  \label{eq: maschen}
\end{equation}
\begin{figure}
  \centering
  \fbox{\includegraphics[width = 6cm]{pics/prinz_brückenschaltung.png}}
  \caption{Prinzipielle Brückenschaltung\cite{anleitung302}}
  \label{fig: prinzbrücke}
\end{figure}
Der grundsätzliche Aufbau einer Brückenschaltung ist in Abbildung  dargestellt. Regelt man die Brückenspannung $U_{\symup{Br}}$
zwischen den Punkten A und B auf null (abgeglichene Brücke), so ergeben sich mit \eqref{eq: knoten} und \eqref{eq: maschen}
für allgemeine komplexe Widerstände $Z_i = X_i + j \cdot Y_i$ die
beiden Bedingungen:
\begin{align}
  \begin{aligned}
    X_1 X_4 - Y_1 Y_4 &= X_2 X_3 - Y_2 Y_3 \\
    X_1 Y_4 + X_4 Y_1 &= X_2 Y_3 + X_3 Y_2
  \end{aligned}
  \label{eq: widerstandsbedingungen}
\end{align}
Die spezielleren Formeln für Ausführungen der allgemeinen Brückenschaltung \ref{fig: prinzbrücke} werden im Aufbau aufgeführt.\\
Im Versuch werden drei Arten von Widerständen verwendet. Der rein ohmsche Widerstand
$Z_{\symup{R}}$ verfügt lediglich über einen Wirkwiderstand:
\begin{equation}
 Z_{\symup{R}} = R \quad \Leftrightarrow \quad Y_{\symup{R}} = 0
\end{equation}
Eine reale Kapazität $C$ setzt einen Teil der in ihr gespeicherten elektrischen Energie
in Wärme um. Der Widerstand $Z_{\symup{C}}$ setzt sich also aus einem Wirk- und einem Blindanteil zusammen. Mit der Kreisfrequenz $\omega$ gilt:
\begin{equation}
  Z_{\symup{C}} = R - j \frac{1}{\omega C}
\end{equation}
Für die reale Induktivität $L$, die einen Teil der in ihr
gespeicherten magnetischen Feldenergie in Wärme umsetzt, gilt entsprechend:
\begin{equation}
  Z_{\symup{L}} = R + j \omega L
\end{equation}
Über die Bestimmung von Real- und Wirkwiderständen hinaus, soll im Versuch die Güte einer Generatorspannung überprüft werden. Hierbei
macht man sich Brückenschaltungen zur Nutze, die nur für eine bestimmte Spannungsfrequenz $\nu_0$ theoretisch abgeglichen sind. In der
Praxis erzeugt ein Generator neben der Grundschwingung mit Amplitude $U_1$ unerwünschte Oberschwinungen mit entsprechenden
Amplituden $U_i$ ($i\ge 2$). Der sogenannte Klirrfaktor $k$ als Maß für den Oberwellengehalt im Verhältnis zur Grundwelle ist definiert durch:
\begin{equation}
  k = \frac{1}{U_1} \left( \sum_{i\ge2} U_i ^2 \right) ^{\frac{1}{2}}
  \label{eq: klirr}
\end{equation}
