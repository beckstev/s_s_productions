\section{Auswertung}

Die im Versuch gemessenen Wiederstände, Kapazitäten und Induktivitäten
wurden alle als fehlerfrei angenommen.
Und mit einer Genauigkeit von drei Signifikatenstellen angegeben.%Überprüfen

\subsection{Bestimmung zweier Widerstände mithilfe der Whetonschen Brückenschaltung}

Im ersten Teil des Veruches sollten zwei unbekannte Widerstäne $R_{\map{12}}$ und $R_{\map{13}}$ ausgemessen werden.
Die dazu eingestellen Widerstände am Potentiometer \\ (Gesamtwiderstand 
$R_{G}=\SI{1000}{\ohm}$) werden wie folgt umgerechnet:

\begin{equation*}
R\ua{4}=R\ua{G}-R_{\map{3}}
\end{equation*}

Anschließen kann dann mithilfe von Formel \eqref{eq: Rx} die 
unbekannten Wiederstände $R_{\map{12}}$ und $R_{\map{13}}$ bestimmt werden. 

Die sich mit den Messwerten ergebene Widerstände sind in Tabelle \ref{tab:wid_r12} und
\ref{tab:wid_r13} abgebildet. 

\begin{figure}
\begin{subfigure}{0.49\textwidth}
\centering
\begin{tabular}{S S S S}
      \toprule
    {$R\ua{2}$ in $\si{\ohm}$} &  {$R\ua{3}$ in $\si{\ohm}$} & {$R\ua{4}$ in $\si{\ohm}$} &{$R\ua{12}$ in $\si{\ohm}$}  \\
    \midrule
     {$\num{332}$} & {$\num{541}$} &  {$\num{459}$}& {$\num{391}$}  \\
     {$\num{500}$} & {$\num{439}$}  & {$\num{561}$} & {$\num{391}$}  \\
     {$\num{1000}$} & {$\num{281}$}  & {$\num{719}$} & {$\num{390}$}  \\
     \bottomrule
	\end{tabular}
  \caption{Gemessene Widerstände für $R\ua{12}$}
  \label{tab:wid_r12}
\end{subfigure}
\begin{subfigure}{0.49\textwidth}
\centering
\begin{tabular}{S S S S}
      \toprule
    {$R\ua{2}$ in $\si{\ohm}$} &  {$R\ua{3}$ in $\si{\ohm}$} & {$R\ua{4}$ in $\si{\ohm}$} &{$R\ua{13}$ in $\si{\ohm}$}  \\
    \midrule
     {$\num{332}$} & {$\num{420}$}  & {$\num{580}$} & {$\num{240}$}  \\
     {$\num{500}$} & {$\num{324}$}  & {$\num{676}$} & {$\num{234}$}  \\
     {$\num{1000}$} & {$\num{192}$} &  {$\num{808}$}& {$\num{238}$}  \\
     \bottomrule
  \end{tabular}
  \caption{Gemessene Widerstände für $R\ua{13}$}
  \label{tab:wid_r13}
\end{subfigure}
\label{fig:tabellen_rx_a}
\end{figure}

Die Mittlungen und Standartabweichungem werden im folgenden immer 
mit den bekannten Zusammenhöngen bestimmt.

Für $R_{\map{12}}$ und $R_{\map{13}}$ ergeben sich 

\begin{equation}
\label{eq:wert_r12_r_13}
\overline{R}_{\map{12}}=\SI{391\pm0.13}{\ohm}\quad \overline{R}_{\map{13}}=\SI{239\pm0.68}{\ohm}
\end{equation}
als Mittelwert und dazugehöriger Abweichung.

\subsection{Kapazitätsmessung mithilfe der Kapazitätmessbrücke}

Bei den folgenden Messungen werden einmal 
ideale Kondensatoren angenommen und einmal ideale.

\subsubsection{Idealer Kondensator}

Um die unbekannte Kapazität der Kondensatoren $C\ua{1}$ und $C\ua{3}$
zu bestimmen.
