\section{Diskussion}
In diesem Abschnitt soll die Aussagekraft der Ergebnisse, bezogen auf den
Rahmbedingungen des Versuches diskutiert werden.

Hierbei sei direkt die Annahme der fehlerfreien Bauelemente zu erwähnen.
Da eine fehlerfreie Produktion in der Realität nicht gewährleistet ist, 
sinkt dadurch die Qualität der Ergebnisse.
Des Weiteren ist das ungenaue ablesen am Oszilloskop und Multimeter eine 
weitere Fehlerquelle.
Insbesondere das Oszilloskop beeinträchtigt die Genauigkeit des Versuches.
Denn es war nicht möglich, aufgrund eines Defekts, die genauste 
Skala zu nutzen.
Dennoch zeigt der Versuch relativ gute Ergebnisse. 
Die in Auswertung bestimmten Werte haben, bis auf einen (vgl. \eqref{eq:wert_r16_l_16}), eine geringe Abweichung.
Die Messung mit Brückenschaltung erlauben anscheinend relativ Genaues ausmessen von Bauelementen.
Innerhalb der Brückenschaltung gibt es weitere Unterschiede.
Beim Vergleich der Ergebnise \eqref{eq:wert_r16_l_16} mit \eqref{eq:wert_lr_16_max}, 
fällt der große Fehler der Kapazitätsmessbrücke bei der Berechnung des Widerstands
auf. Es empfiehlt sich also eher, die Maxwell-Brücke zu verwenden.
Bei der Wien-Robinson-Brücke ist zu erwähnen, dass der in Abbildung \ref{fig: plot}
zu sehende Abfall (im Plot rechts) wahrscheinlich auf den Tiefpass zurückgeführt werden
kann.

Die abschließende Betrachtung der Ergebnisse zeigt, unter Berücksichtigung der Ungenauigkeit, 
dass Brückenschaltungen eine gute Möglichkeit bietet um elektrische Bauelemente 
auszumessen.