\section{Auswertung}
\subsection{Bestimmung der Relativeschwindigkeit}

In diesem Teil des Versuches wird bestimmt, mit welcher Geschwindigkeit der vom Synchronmotor angetrieben Wagen sich befindet.
%Tabelle
Dazu wurde für jede Getriebestufe, eine Messreihe mit $n=5$ Messwerten aufgenommen.
Die Messdaten wurden in \textbf{Tabelle 1} aufgelistet.

Um die Geschwindigkeit zu berechnen, wurde das Gesetz $v=\frac{s}{t}$
genutzt. Dazu wurde der Weg $l$ auf $\num{13e-2}\si{\meter}$ gemessen mit einer Fehlerabschätzung von $\pm\num{1e-3}\si{\meter}$. 
Der Mittelwert der Zeitintervalle $t$ wurde mit der Formel

\begin{equation*}
\bar{t}=\frac{1}{n}\sum_{i=1}t_i
\end{equation*}
bestimmt. Durch Anwendung von 

\begin{equation*}
\bar{\sigma}_{\bar{t}}=\sqrt{\frac{1}{n(n-1)}\sum_{i=1}^{n}(t_i-\bar{t})^2}
\end{equation*}

konnte die Abweichung des Mittelwertes angeben werden.
Die Tabelle mit den gemessen Werten befindet sich im Anhang des Protokolls.

\subsection{Ruhefrequenzmessung}

In diesem Versuch sollte die Größe $\nu_0$ bestimmt werden. 
Diese Größe wurde mittels $\nu_0=t*N$ bestimmt. Dabei sei $t$ der 
festgelegte Messzeitraum und $N$ die gemessene Anzahl an Phasendurchläufen. 

Es ergibt sich folgende Frequenz:

\textbf{Hier Tabelle einfügen}

\subsection{Bestimmung der Wellenlänge} 
Für die Berechnung der Wellenlänge wurde immer der Abstand von 
zwei Phasen gemessen. 
Damit ergaben sich, verschiedene Wellenlängen von den dann 
der Mittelwert berechnet wurde:

\textbf{Wert der Wellenlänge einfügen}

\subsection{Ermittlung der Schallgeschwindigkeit}
Die Schallgeschwindigkeit wurde mit dem Zusammenhang

\begin{equation}
c=\lambda \nu
\end{equation}
,bestimmt. Dieser ist gültig bei einer Messung in Luft und bei Raumtemperatur.
Dabei wurde für $\nu$ der Mittelwert der Ruhefrequenz eingesetzt.
Aus der Theorie sei zu vermuten, dass es ein Unterschied $\nu_s$ und $\nu_e$ gibt.
Doch im Versuch ist eine relevante Differenz nicht festzustellen. 
Denn bei der Betrachtung der Reihenenwticklung von \textbf{Hier Formmelnummer einfügen},
wird deutlich, dass die quadratischen Terme schon so klein sind, dass sie nicht mehr in
das Gewicht fallen. Ein Ziel des Versuches war es die Größe $\zeta=\frac{v_0}{c}=\frac{1}{\lambda}$
zu bestimmen.
Durch Verwendung der gemittelten Wellenlänge ergibt sich:

\textbf{Wert einfügen}

\subsection{Messung des Dopplereffekts 1}
Die durch den Dopplereffekt eintretende Frequenzänderung, wurde 
mit $\Delta \nu=\nu_0-\nu_l$ berechnet. Hierbei sein $\nu_l$ die 
gemessenen Werte, die im Anhang eingesehen werden können.
Es ergab sich für die verschiedenen Geschwindigkeiten folgender Zusammenhang:

\textbf{Hier Tabbelle einfügen}

Des Weiteren befindet sich im Anhang die grafische Auftragung 
von $v$ zu $\Delta \nu$. 
Die oben erwähnte Größe $\zeta$ sollte dabei ungefähr der 
Steigung der Ausgleichsgeraden der Messwerte betragen.
Mittels der aus Regressionsrechnung genutzte Gleichung

\begin{equation*}
m=\frac{\bar{xy}-\bar{x}\bar{y}}{\bar{x^2}-\bar{x}^2}
\end{equation*}

und dem dazugehörigen Fehler
%Woher kommt die nochmal?
%Sigma?
\begin{equation*}
o_m=\sqrt{\frac{\sigma^2}{N(\bar{x^2}-\bar{x}^2)}}
\end{equation*}

ergibt sich für die Steigung der Wert:

\textbf{Den nochmal mit zusamengefassten Bereichen berechnen}


\subsection{Messung des Dopplereffekts - Schwebungsmethode}
Der Doppler-Effekt sollte auch einmal mit der Schwebungsmethode 
bestimmt werden.
Dabei wurden folgende Werte für die Frequenzänderung 
bestimmt:

\textbf{Tabelle einfügen}

Des Weitern ist im Anhang noch ein Plot von $v$ zu $\Delta \nu$ 
zu finden. Auch hier ist es sinnvoll die Steigung der Ausgleichsgerade, mit 
Regressionsrechnung, zu bestimmen, um sie anschließend mit dem 
Faktor $\zeta$ zu vergleichen.
Nach der Regressionsrechnung ergibt sich:

\subsection{Students-t-Faktor}