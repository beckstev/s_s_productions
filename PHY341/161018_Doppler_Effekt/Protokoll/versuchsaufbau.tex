
\section{Versuchsdurchführung}

Der Versuch Doppler-Effekt bestand an sich aus vier einzelnen Versuchen.
In jedem dieser sollte entweder der Doppler-Effekt selber gemessen oder eine
für die Auswertung entscheidene Größe bestimmt werden.

\subsection{Messung der Relativgeschwindigkeit}

%Die Messung der Realtivgeschwindigkeit, wird benötigt um nacher die Frequenz
%bzw. die Wellenlängenverschiebung des Doppler-Effekts zu berechnen.
%
%Zur Messung benötigt man einen Wagen, auf dem der Lautsprecher montiert ist,
%zwei Lichtschranken, logische Schaltung und ein Antriebsmotor (10 Gang).
%
%Die Lichtschranken sollen Anfangs- und Endpunkt festlegen.
%Erreicht der Wagen, angetrieben durch den Synchronmotor, die erste
%Lichtschranke, so aktiviert diese.

Der Verrsuchsaufbau ist in Abbildung $1$ zu entnehmen.
Der Wagen wurde von einem zehngängigen Synchronmotor angetrieben.
Dabei sollen die Lichtschranken Start- und Endpunkt der Messung festlegen.
Um somit die Zeit zu bestimmen, die der Wagen benöigt um die beiden
Lichtschranken zu passieren.



\subsection{Messung der Schallgeschwindigkeit}

Der Versusaufbau zur Messung der Schallgeschwindigkeit ist
Abbildung \textbf{ NUMMER EINFÜGEN} zu entnehmen.

Im Versuch werden das vom Quarzgenrator gelieferte Ton und der
vom Mikrofon gemessene Ton an einem Ozilliskop angelegt.
Dabei werden die Signale so eingestelt Lissajou-Figuren entstehen.
Diese werden genutzt um den auf dem Lautsprecher montierten Schlitten, so
einzustellen, dass beide Signale in Phase (Lissajou Figur entspricht einer Geraden) sind.
Wenn nun immer der Abstand abgelesen wird, zwischen zwei
Signalen gleicher Phase, kann auf die Wellenlänge geschlossen werden.

\subsection{Frequenzmessung}

Im Anhang \textbf{Nummer einfügen} wird der Versuchsaufbau dargestellt.
Beim Versuch bewegt sich der, auf dem Wagen montierte, Lautsprecher entweder
in Richtung des Mikrofons oder in gegengesetzer Richtung vom Mikrofons.
Dadurch entsteht eine Relativbewegung und somit auch der
Doppler-Effekt.
Nachdem der Wagen mittels eine Lichtschranke den Messzeitraum startet.
Das Mikrofon misst solange, jede angekommene Phase, bis der Untersetzer
die Messung beendet.
Der Versuch wird mit verschiedenen Wagengeschwindigkeit
wiederholt, um die unerschieldiche Stärke des Doppler-Effekts
festzustellen.

\subsection{Frequenzmessung mit der Schwebungsmethode}

Der Versuchsaufbau ist im Anhang \textbf{Nummer einfügen}
skizziert.
Im wesentlich besteht hier der Unterschied zur voherigen
Frequenzmessung, das sich bei dieser Messung Mikrofon und
Lautsprecher statisch sind.
Lediglich ein Reflektor auf einem Wagen bewegt sich.
Der Lautsprecher, der nebem dem Mikrofon plaziert ist, schallt auf
den Reflektor. Dadurch kommt es zu einer Überlagerung von zwei Schwingungen
(1. Ruhefrequenz vom Lautsprecher und 2. vom Reflektor reflektierte Schwingung).
Dadurch entsteht eine Schwebung die vom Mikrofon gemessen wird.
Anschließend Filter ein Tiefpass die vom Dopplereffekt verursachte
Frequenzänderung raus.
