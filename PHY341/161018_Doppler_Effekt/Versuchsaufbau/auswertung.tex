\documentclass[parskip=half]{scrartcl} %Ich habe [parskip=half] hinzugefügt

%\usepackage[calc]{picture}
\usepackage{fixltx2e}

\usepackage{tcolorbox}

%\pagestyle{headings}
\usepackage{scrpage2}
\pagestyle{scrheadings}
\ifoot[\pagemark]{\pagemark}
\ofoot[]{}
\cfoot[]{}

\usepackage{polyglossia}
\setmainlanguage{german}

\usepackage{amsmath}
\usepackage{amssymb}
\usepackage{mathtools}

\usepackage{fontspec}
\defaultfontfeatures{Ligatures=TeX}

\usepackage[
  math-style=ISO,
  bold-style=ISO,
  sans-style=italic,
  nabla=upright,
]{unicode-math}

\setmathfont{Latin Modern Math}
%\setmathfont[range={\mathscr, \mathbfscr}]{XITS Math}
%\setmathfont[range=\coloneq]{XITS Math}
%\setmathfont[range=\propto]{XITS Math}

\usepackage[autostyle]{csquotes}

\usepackage[
  locale=DE,                   % deutsche Einstellungen
  separate-uncertainty=true,   % Immer Fehler mit \pm
  per-mode=symbol-or-fraction, % m/s im Text, sonst Brüche
]{siunitx}
\sisetup{math-stylemicro=\text{µ},text-micro=µ}

\usepackage{xfrac}

\usepackage[section, below]{placeins}
\usepackage[
  labelfont=bf,
  font=small,
  width=0.9\textwidth,
]{caption}

\usepackage{subcaption}

\usepackage{graphicx}

\usepackage{float}
\floatplacement{figure}{h}
\floatplacement{table}{h}

\usepackage{booktabs}

\usepackage{biblatex}
\addbibresource{lit.bib}

\usepackage{bookmark}

\usepackage[shortcuts]{extdash}

\usepackage[math]{blindtext}

\usepackage{microtype}



\usepackage{hyperref}

\usepackage{color} % Das ist Geschmacksfrage

\usepackage{makeidx} %Ich habe makeidx hinzugefügt + makeindex
\makeindex

\newcommand{\tens}[1]{\underline{\underline{#1}}} %Für Tensor mit der Ordnung zwei
\newcommand{\map}[1]{\mathup{#1}} %Befehl für mathup

\usepackage[version=3]{mhchem} % für Thermodynamik-chemische Elemente
\usepackage{enumitem} %Ich habe enumitem hinzugefügt



%\usepackage{showframe}

\author{Steven Becker und Stefan Grisad}

\title{Doppler-Effekt}

\date{\today\\WS 2016/2017}

\newcommand{\ud}{\mathup{d}}
\newcommand{\del}{\partial}

\begin{document}
\maketitle
%\input{""}
\section{Auswertung}
\subsection{Bestimmung der Relativeschwindigkeit}

In diesem Teil dess Versuches wird bestimmt, mit welcher Geschwindigkeit der vom Synchronmotor angetriebe Wagen sich befindet.
%Tabelle
Dazu wurde für jede Getriebestufe, eine Messreihe mit $n=5$ Messwerten aufgenomen.
Die Messdaten wurden in \textbf{Tabelle 1} aufgelistet.

Um die Geschwindigkeit zu berechnen wurde das Gestz $v=\frac{s}{t}$
genutzt. Dabei wurde der Weg $s$ auf $\num{13e-2}$ gemessen mit einer Fehlerabschätzung von $\pm\num{1e-3}$. 
Der Mittelwert der Zeitintervalle $t$ wurde mit der Formel

\begin{equation*}
\bar{t}=\frac{1}{n}\sum_{i=1}t_i
\end{equation*}
bestimmt. Durch Anwendung von 

\begin{equation*}
\bar{\sigma}_{\bar{t}}=\sqrt{\frac{1}{n(n-1)}\sum_{i=1}^{n}(t_i-\bar{t})^2}
\end{equation*}

konnte die Abweichung des Mittelwertes angeben werden.
Die Tabelle mit den gemessen Werten befindet sich im Anhang des Protokolls.

\subsection{Ruhefrequenzmessung}

In diesem Versuch sollte die Größe $\nu_0$ bestimmt werden. 
Diese Größe wurde mittels $\nu_0=t*N$ bestimmt. Dabei sei $t$ der 
festgelegte Messzeitraum und $N$ die gemessene Anzahl an Phasendurchläufen. 

Es eribt sich folgende Frequenz:

\textbf{Hier Tabelle einfügen}

\subsection{Bestimmung der Wellenlänge} 
Für die Berechnung der Wellenlänge wurden immer der Abstand von 
zwei Phasen gemessen. 
Damit ergaben sich verschiedne Wellenlängen von den dann 
der Mittelwert berechnet wurde:

\textbf{Wert der Wellenlänge einfügen}

\subsection{Ermittlung der Schallgeschwindigkeit}
Die Schallgeschwindigkeit wurde mit dem Zusammenhang

\begin{equation}
c=\lambda \nu
\end{equation}
,bestimmt. Dieser ist gültig bei einer Messung in Luft und bei Raumtemperatur.
Dabei wurde für $\nu$ der Mittelwert der Ruhefrequenz eingesetzt.
Aus der Theorie sei zu vermuten das es ein Unterschied $\nu_s$ und $\nu_e$ gibt.
Doch im Versuch ist eine relevante Differenz nicht festzustellen. 
Denn bei dei Betrachtung der Reihenenwticklung von \textbf{Hier Formmelnummer einfügen},
wird deutlch das die quadratischen Terme schon so klein sind, dass sie nicht mehr in
das Gewicht fallen. Ein Ziel des Versuches war es die Größe $\zeta=\frac{v_0}{c}=\frac{1}{\lambda}$
zu bestimmen.
Durch Verwendung der gemittelten Wellenlänge ergibt sich:

\textbf{Wert einfügen}

\subsection{Messung des Dopplereffekts 1}
Die durch den Dopplereffekt eintretende Frequenzänderung, wurde 
mit $\Delta \nu=\nu_0-\nu_l$ berechnet. Hierbei sein $\nu_l$ die 
gemessenen Werte die im Anhang eingesehen werden können.
Es ergab sich für die verschiedenen Geschwindigkeiten folgender Zusammenhang:

\textbf{Hier Tabbelle einfügen}

Desweiteren befindet sich im Anhang die graphische Auftragung 
von $v$ zu $\Delta \nu$. 
Die oben erwähnte Größe $\zeta$ sollte dabei ungefähr der 
Steigung der Ausgleichsgeraden der Messwerte betragen.
Mittels der aus Regressionsrechnung genutze Gleichung

\begin{equation*}
m=\frac{\bar{xy}-\bar{x}\bar{y}}{\bar{x^2}-\bar{x}^2}
\end{equation*}

und dem dazugehörigen Fehler
%Woher kommt die nochmal?
%Sigma?
\begin{equation*}
o_m=\sqrt{\frac{\sigma^2}{N(\bar{x^2}-\bar{x}^2)}}
\end{equation*}

ergibt sich für die Steigung der Wert:

\texbf{Den nochmal mit zusamengefassten Bereichen berechnen}


\subsection{Messung des Dopplereffekts - Schwebungsmethode}
Der Doppler-Effekt sollte auch einmal mit der Schwebungsmethode 
bestimmt werden.
Dabei wurden folgende Werte für die Frequenzänderung 
bestimmt:

\textbf{Tabelle einfügen}

Desweitern ist im Anhang noch ein Plot von $v$ zu $\Delta \nu$ 
zu finden. Auch hier ist es sinnvol die Steigung der Ausgleichsgerade, mit 
Regressionsrechnung, zu bestimmen, um sie anschließend mit dem 
Faktor $\zeta$ zu vergleichen.
Nach der Regressionsrechung ergibt sich:

\subsection{Students-t-Faktor}
\printbibliography
\printindex
\end{document}