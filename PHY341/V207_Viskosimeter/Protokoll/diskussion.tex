\section{Diskussion}
In diesem Kapitel sollen die gewonnen Ergebnisse
aus der Auswertung diskutiert werden.
Bei dem Versuch besitzt die Zeitmessung eine
signifikante Unsicherheit. %was meinst du damit?
Durch die Mittlung der Zeit wird, aber die Aussagekraft des
Experiements sichergestellt.% direkter
Eine weitere Ungenauigkeit entsteht durch den Versuchsaufbau selber. %1
Denn eine genaue Temperaturmessung des Wassers im Viskosimeter
ist kaum möglich.
Da die Temperatur des destillierten Wassers nicht direkt gemessen wird,
sondern indirekt durch die Temperatur des Wasserbades. %2 aus 1 bis 2 besser einen Satz machen
Die Unsicherheit der Messergebnisse wird weiter erhöht dadurch, dass
die angenommenen Literaturwerte ohne Fehler angegeben sind. %ausdruck %was meinst du damit?
Bei dem Vergleich der Messergebnisse mit de Literatur in Abbildung
\ref{fig:t_v_v} und \ref{fig:t_v_l_v} wird eine y-Achsenverschiebung deutlich.
Diese lässt sich eventuell auf die ungenaue Zeitmessung bzw. Temperaturmessung zurückführen.
In Abbildung \ref{fig:t_v_l_v} ist gut zu erkennen, dass die Steigung beider
Kurven gleich ist. %exakt glech? %du meinst nicht gleich oder?
Dies lässt sich auch auf die ungenaue Temperatur- und Zeitmessung zurückführen.%Oder ?

Abschließend bleibt zu sagen, dass der Versuch die Theorie widerspiegelt. %noch sowas wie: im Rahmen der angesprochenen Präzision des Aufbaus...
Die Messergebnisse sind also als plausibel einzustufen.
