\section*{Zielsetzung}
Im Versuch $207$ soll das Verhalten der sogenannten \emph{dynamischen Viskosität} von destilliertem Wasser untersucht werden. Hierzu wird das Kugelfall-Viskosimeter verwendet,
das auf eine Erfindung von Fritz Höppler aus dem Jahre 1932 zurück geht.

\section{Theorie}
Die Strömung einer Flüssigkeit wird als \emph{laminar} bezeichnet, wenn die \emph{innere Reibung} eine Kraft $F_R$ verursacht, die wesentlich größer ist
als die strömungserzeugende Kraft $F_B$. Die Strömungslinien verlaufen ohne Verwirbelungen nebeneinander.
In diesem Fall ergibt sich der Betrag der Reibungskraft auf eine Kugel mit Radius $R$, die sich mit einer Geschwindigkeit
$v$ durch die Flüssigkeit bewegt, mittels des empirisch gefundenen \emph{Stokesschen Gesetzes}:
\begin{equation}
  F_R = 6 \pi \cdot \eta \cdot v \cdot R
  \label{eq: stokes}
\end{equation}
Hierbei handelt es sich bei $\eta$ um die anfangs erwähnte \emph{dynamische Viskosität}, deren Abhängigkeit von der Temperatur $T$ durch eine Funktion der Form
\begin{equation}
  \eta(T) = \mathup{A} \cdot \exp\left(\frac{B}{T}\right) \quad \mathup{A}, \mathup{B} \in \mathbb{R}
  \label{eq: andrad}
\end{equation}
beschrieben werden kann (\emph{Andredesche Funktion}). \\
Fällt ein Körper in einer viskosen Flüssigkeit nach unten, so wirkt neben der Gewichts- und Reibungskraft ebenfalls eine nach oben gerichtete Auftriebskraft $F_A$, deren
Betrag sich aus dem \emph{Prinzip des Archimedes} ergibt.
\begin{equation}
  F_A = \rho_{Fl} \cdot V_K \cdot g
\end{equation}
Hierbei ist $\rho_{Fl}$ die Dichte der Flüssigkeit und $V_K$ das Volumen des Körpers (hier: Kugel). Durch das Ansteigen der Reibungskraft mit der Geschwindigkeit,
stellt sich nach kurzer Zeit im Kräftgleichgewicht eine konstante Geschwindigkeit $v_0$ ein.
\begin{align}
  \begin{aligned}
  6 \pi \cdot \eta \cdot v_0 \cdot R_K  &= (\rho_K - \rho_{Fl}) \cdot \frac{4}{3}\pi R_K^3 \cdot g  \\
  \hfill
  \Leftrightarrow \quad \eta &= \frac{2 (\rho_K - \rho_{Fl}) \cdot R_K^2 \cdot  g}{9 v_0}
\end{aligned}
\end{align}
Mit der Dichte des Körpers $\rho_K$. Durch Ersetzen der Geschwindigkeit $v_0 = s / t$ ergibt sich:
\begin{equation}
  \eta = \frac{2 (\rho_K - \rho_{Fl}) R_K^2 \cdot  g \cdot t}{9  s} := K (\rho_K - \rho_{Fl}) \cdot t
  \label{eq: eta}
\end{equation}
Hierbei ist $K$ nun eine Konstante, die lediglich von der Geometrie der Kugel und der Fallstrecke abhängt. Sie dient der einfachen Handhabung im späteren Verlauf.\\

Um den Fluss verschiedener Medien in ähnlichen Körpern qualitativ vergleichen zu können, definiert man in der Strömungslehre die sogenannte \emph{Reynoldsche Zahl} $R_e$. Für
Zylinder mit Durchmesser $d$ ergibt sie sich zu:
\begin{equation}
  R_e = \frac{\rho_{Fl} \cdot v_0 \cdot d}{\eta}
  \label{eq: reynolds}
\end{equation}
Ist die \emph{Reynoldsche Zahl} zweier Anordnungen (z.B. Fluss von Honig in einem Rohr mit Durchmesser $1 \si{\meter}$ und Fluss von Wasser in einem Rohr mit Durchmesser $1\si{cm}$) gleich, so
ist das Strömungs- bzw. Turbulenzverhalten der Theorie nach identisch. Aus diesem Grund kann durch das Bestimmen der \emph{Reynoldschen Zahl} beurteilt werden, ob eine Strömung laminar ist oder nicht.
Als Grenzwert für laminare Strömungen findet man etwa in einem Forschungsbericht des Max-Planck-Instituts für Dynamik und Selbstorganisation \cite{rey}:
\begin{equation}
  \text{Strömung laminar} \quad \Leftrightarrow \quad R_e < 2040
  \label{eq: laminar}
\end{equation}
