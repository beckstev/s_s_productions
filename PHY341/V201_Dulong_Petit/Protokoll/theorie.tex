\section*{Zielsetzung}
Der Versuch soll eine Aussage darüber treffen, ob das Dulong-Petitsche Gesetz, der klassichen Physik genügt, oder
eine quantenmechanische Beschreibung von Nöten ist.

\section{Theorie}

\subsection{Definition der spezifischen Wärmekapazität}

Findet an einem Körper eine Temperaturänderung $\Delta T$ statt, ohne das an ihm 
Arbeit verrichet wird. So kommt es bei ihm zu einer Wärmeaufnahme oder -abgabe $\Delta Q$.
Mit dem ersten Hauptsatz der Thermodynamik ergibt sich der Zusammenhang

\begin{equation*}
\Delta Q=m c \Delta T.
\end{equation*}

Dabei sei $m$ die Masse und $c$ die Wärmekapazität 
des Körpers.
Wird zusätzlich die Masseneinheit berücksichtig
so wird $c$ als spezifische Wärmekapazität bezeichnet.

Weiter ist die Molwärme $C$ eine für den Versuch relevante Größe.
Sie gitb an was für eine Wärmemenge $\map{d}Q$ benötigt wird,
um ein Mol eines Stoffes um $\map{d}T$ zu erwärmen.

Der Betrag der Molwärme hängt davon ab unter welchen Bedingungen 
die Temperaturänderung hervor gerufen wurde.
Zum Beispiel wird zwischen der spezfischen Wärmekapazität beo konstantem
Volumen $C_{V}$ und bei kosntantem Druck $C_{p}$.


\subsection{Dulong-Petitsch-Gestz}

Das Dulong-Petitsche-Gesetz sagt aus, dass die Atomwärme bei 
konstantem Volumen unabhängig von den chemischen Eigenschaften eines Elementes 
und gleich $3R$, hier sei $R$ die allgemeine Gaskonstante.
In der Herleitung des Gesetztes wird sich die Gitterstrucktur eines 
Festkörper zu nutze gemacht.






\subsection{Thermoelemnte und ihre Funktionsweise}

Ein Thermoelemt wird häufig eingesetzt um eine 
Temperatur mir einer hohen Einstellungsschwindigkeit zu messen.
Es besteht aus zwei Metallen, die sich durch zwei 
verschiedene Wärmeleitkoeffizienten auszeichnen.
Am Ende des Themoelemts befindet sich eine Berührungstelle zwischen 
den beiden Elementen.
An dieser kommt es später zu einem Elektronenfluss.

