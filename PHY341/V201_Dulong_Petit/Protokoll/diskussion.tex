\section{Diskussion}
In diesem Abschnitt sollen nun die gewonnen Ergebnisse abschließend diskutiert und in
Verhältnis mit der Präzision des verwendeten Aufbaus gesetzt werden. \\
Zunächst zur Temperaturmessung. Die Annäherung des Spannung-Temperatur-Verhältnisses
an eine Gerade \eqref{eq: UtoTemp} stellt eine starke Vereinfachung dar. Ein systematischer
Fehler an dieser Stelle der Durchführung hat immense Auswirkungen auf die folgenden Berechnungen.
Aus diesem Grund ist die Signifikanz der gesamten Ergebnisse als relativ gering einzustufen.
Der Vergleich der spezifischen Wärmekapazitäten mit den Literaturwerten im Fall Aluminium und Graphit bestätigt diese Vermutung (siehe
Tabelle \ref{tab: comp}). Als weitere Fehlerquelle ist die Unvollkommenheit des Aufbaus zu nennen.
Es ist beispielsweise zu keinem Zeitpunkt gewährleistet, dass Wärme nicht auch an die Umgebung abgegeben werden kann.
Dennoch erlauben die gefundenen Werte die Feststellung, dass der aus der klassichen Mechanik hergeleitete Zusammenhang $C_V = 3R$ allgemein
nicht gültig ist. Insbesondere die spezifische Wärmekapazität von Graphit weicht deutlich von diesem Wert ab. Das Dulong-Petit-Gesetz liefert %lieber: weicht vom Literatturwert ab
einen Wert, der zwar in der richtigen Größenordnung liegt, aber nur bei einigen Stoffen wie Zinn eine sinnvolle Modellvorstellung darstellt. Allgemein
müssen offenbar quantenmechanische Überlegungen zur Beschreibung der Prozesse in einem Festkörper heran gezogen werden.
