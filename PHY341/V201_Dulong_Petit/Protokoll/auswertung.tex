\section{Auswertung}
\subsection{Eichung des Thermoelements}
Der Zusammenhang zwischen dem Kontaktpotential des Kalorimeters und der Temperatur soll im späteren Verlauf
durch einen linearen Zusammenhang angenähert werden. Hierzu wurde eine Kontakstelle in Eiswasser($0 \si{\celsius}$)
und eine in kochendes Wasser ($100 \si{\celsius}$) gehalten. Hierbei wurde folgender Wert am Kalorimeter abgelesen:
\begin{equation}
  U_{eich} = 3.98 \si{\milli\volt}
\end{equation}
Die Steigung der Geraden ergibt sich zu:
\begin{equation}
  m = \frac{100}{U_{eich}}
\end{equation}
Hiermit lässt sich die Temperatur in Kelvin über den Zusammenhang
\begin{equation}
  T(U) = m \cdot U + 275.13
\end{equation}
berechnen.

\subsection{Bestimmung der Wärmekapazität des Kalorimeters}
