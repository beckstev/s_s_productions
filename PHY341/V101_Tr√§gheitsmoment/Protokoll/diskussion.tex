\section{Diskussion}
In diesem Kapitel soll eine Art Plausibilitätsprüfung in Bezug auf die
gewonnen Ergebnise folgen.
Dabei beginnen wir zunächst bei der Messung der Schwingungsdauer.
Hier sei direkt der Mensch und seine Reaktionszeit als Fehlerquelle anzumerken.
Denn eine exakte Festlegung von Anfangs- und Endpunkt einer jeden Messung ist
nicht möglich. % Denn es ist, extrem schwierig Anfangs- und Endpunkt einer jeden Zeitmessung exakt
Doch eine Mittelung
über 5 Messwerte sorgt dafür, dass der so entstehende Fehler vergleichsweise klein wird, wie in den Tabellen
\ref{tab: winkelricht} und \ref{tab: winkel_dynamisc} zu sehen ist.

Der entstandene Unterschied zwischen der statisch und dynamisch gemessenen Winkelrichtgröße
lässt sich auf einige Faktoren zurückführen.
Zum einen ist das Ablesen der Kraft bei der statischen Methode
durch die grobe Skalierung ungenau.
Zweitens ist das Positionieren der Federwaage orthogonal zum Radius
nicht ganz genau.
Bei der dynamischen Methode ist auch das oben angesprochene Messen der Schwingungsdauer
eine Fehlerquelle.
Dazu kommt die Messungenauigkeit der Waage bei der Gewichtmessung.
Betrachtet man die Messung des Eigenträgheitsmomentes der Drillachse, so fällt sofort auf,
dass experimentelle Größen mit theoretisch bestimmten Größen (Trägheitsmoment der beiden Zylinder)
gemischt worden sind.
Dies vermindert die Aussagekraft der Messergebnisse.
Hinzu kommt noch, dass die Höhen und Radien der Zylinder nicht ganz genau bestimmt werden können.
Da die Ausrichtung der Massen auf dem Stab nicht hundertprozentig genau erfolgen kann, ist
eine vollkommen symmetrische Anordnung nicht möglich.
Hieraus folgt somit eine weitere Unsicherheit die in das Resultat einfließt.

Die Messung der Trägheitsmomente von Kugel und Zylinder fiel unterschiedlich aus.
Bei der Kugel war eine Abweichung von $+13\%$ zur Theorie festzustellen,
dies lässt sich wahrscheinlich auf die ungenaue Größe $I_S$ zurückführen.
Ist aber dennoch als ein positives Ergebnis einzustufen.
Hingegen war die Abweichung bei dem Zylinder $\pm 0\%$. Dies zeigt,
wie genau mechanische Messungen von Hand getätigt werden können.
Die wohl größte Ungenauigkeit fand man bei der Messung des Trägheitsmomentes der Puppe.
Denn hier lag die Abweichung zwischen Theorie und Praxis bei bis zu
$+1200\%$ (Position 1). Dieses Ergebnis lässt sich auf die
ungenaue theoretische Berechnung zurückführen.
Denn zum einen wurde die Puppe stark vereinfacht siehe \ref{fig:approx_puppe}, als auch
die Befestigung der Puppe vernachlässigt.  %kann man das so schreiben ??

Abschließend ist zu sagen, dass sich der Zusammenhang zwischen der Theorie und den experimentellen Befunden
gerade bei dem Zylinder und der Kugel bestätigt haben.
Lediglich bei der Puppe ist eine qualitative Aussage nicht möglich.
