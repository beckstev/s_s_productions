\section{Diskussion}
Im folgenden Kapitel soll die Aussagekraft der Ergebnisse in Bezug auf die
Genauigkeit des Versuchsaufbau unterscht werden.
Bei der Betrachtung der Messergebnisse (vgl. Tabelle \ref{tab:messergebnisse})

\begin{table}
  \centering
  \caption{Messergebnisse für $RC$.}
  \label{tab:messergebnisse}
  \begin{tabular}{cccc}
    \toprule
    {Entladekurve} & {frequenzabhängige} & { Phasenverschiebung  } & {Phasenverschiebung} \\
    & { Kondensatorspannung} & {ohne Korrektur} & {mit Korrektur} \\
    \midrule
    $\SI{1.12\pm0.1}{\milli\second} $ & $\SI{1.49\pm0.02}{\milli\second}$ &
    $\SI{1.7\pm0.5}{\milli\second}$ & $\SI{1.49\pm0.07}{\milli\second}$ \\
    \bottomrule
  \end{tabular}
\end{table}

ist insbesondere die Korrektur bei der Phasenverschienungs-Messmethode anzumerken.
Es scheint, dass es hier zu einem systematischen Fehler z.\,B. zu Ungenaues ablesen gekommen ist.
Das Messergebnis mit der Korrektur zeigt dennoch eine Übereinstimmung mit den
aus den anderen Methoden erzielten Werten für $RC$.
Die Abweichung bei der Untersuchung der Entladekurve lässt sich auf das
niedrig aufgelöste Bild des Oszilloskops zurückführen.

Abschließend lässt sich sagen, dass der Versuch unter Berücksichtigung
der Rahmenbedingung gute Messergebnisse liefert.
