\section{Auswertung}


\subsection{Bestimmung der Kugeldichten}

In Tabelle \ref{tab:messwerte_kugel} sind die gemessenen Radien und Masses der Kugel 
einzusehen.

\begin{table}
\centering
\begin{tabular} {ccccccc}
	\toprule
  & \multicolumn{3}{c}{Radius in $\si{\meter}\cdot \num{e-3}$}  & \multicolumn{3}{c}{Masse in $\si{\kilogram}\cdot \num{e-3}$} \\
\midrule \\
Kugel 1 & $\num{7.565} $&  $\num{7.5625} $ & $\num{7.565} $  & $\num{4.45}$ & $\num{4.44} $ & $\num{4.44} $ \\
Kugel 2  & $\num{7.65} $&  $\num{7.65} $ & $\num{7.65} $ & $\num{4.6}$ & $\num{4.6} $ & $\num{4.61} $ \\
\bottomrule
\end{tabular}
\caption{Abmessung Kugeln}
\label{tab:messwerte_kugel}

Der Mittelwert der Messreihen wird mittels

\begin{equation}
\label{eq:mittel}
\bar{x}=\frac{1}{n}\sum_{i=1}x_i
\end{equation}

berechnet. Dabei wird der zugehörige Fehler
durch 
\begin{equation}
\label{eq:stand_ab}
\bar{\sigma}_{\bar{x}}=\sqrt{\frac{1}{n(n-1)}\sum_{i=1}^{n}(x_i-\bar{x})^2}.
\end{equation}

bestimmt.

Für die Messungen aus \ref{tab:messwerte_kugel} ergeben sich 
folgende gemittelten Werte:

\begin{align}
\label{eq:abmessungen_kugel}
\begin{aligned}
\overline{m}_{1}&= \left(\num{0.0044}\pm\num{e-6}\right) \si{\kilogram} \\
\overline{m}_{2}&= \left(\num{0.0046}\pm\num{e-6}\right) \si{\kilogram} \\
\hfill \\
\overline{r}_{1}&= \left(\num{0.0076}\pm\num{e-7}\right) \si{\meter} \\
\overline{r}_{2}&= \left(\num{0.0077}\pm\num{0}\right) \si{\meter} \\
\end{aligned}
\end{align}



\end{table}
