\section{Diskussion}
Im Folgenden sollen die Messergebnisse in Bezug auf die Messgenauigkeit des
Versuchsaufbaus diskutiert werden.
Zunächst ist die Empfindlichkeit des Versuchsaufbaus gegenüber Erschütterungen %zu aller erst umgangsspracjlich, Erschütterungen
zu erwähnen. Selbst kleinste Erschütterungen sorgen für falsche Registrierungen von %Begleiter oder plural
Maxima. Die Empfindlichkeit ist Grund für eine signifikante Ungenauigkeit in den
Messergebnissen. Mit Hilfe zum Beispiel einer Dämpfung, ist es möglich die %schreib besser einfach dämpfung
Erschütterungsempfindlichkeit erheblich zu verringern. %zu
Eine weitere Fehlerquelle ist die Fotodiode, da diese bei einem zu schnellen
Wechsel von Maxima und Minima, keinen Impuls an den Verstärker überträgt.

Die verursachten Ungenauigkeiten wirken sich besonders bei der experimtellen bestimmten
Wellenlänge \eqref{eq:wellenlaenge}, des roten Läsers, aus.
In der Theorie liegt rotes Licht in einen Wellenlängenbereich von
$\lambda \in \left[640,770\right]\,\si{\nano\meter}$\cite{tafelwerk}.
Das macht eine Abweichungsintervall von $\Delta \lambda \in \left[-6, -22 \right]\,\%$.

Abschließend ist zu sagen, dass die Messergebnisse, trotz der Empfindlichkeiten
des Versuchsaufbaues, eine qualitative Aussage liefern.

%Diskussion ist zu schwammig. Bezieh dich auf die Ergebnisse.
